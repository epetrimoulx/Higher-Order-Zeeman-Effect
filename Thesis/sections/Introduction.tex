\chapter{Introduction}
    \section{Background and Motivation}
        This dissertation investigates hydrogenic systems subjected to an external magnetic field, and plays a key role in the investigation of the $g_\mu - 2$ anomaly. The Zeeman effect as a whole dates back to before the conception of quantum mechanics, and was one of many key experiments that motivated its discovery. The study of hydrogenic states in quantum mechanics provides a basis for the analysis of all the elements, ions, and isotopes throughout the universe and because they consist only of a nucleus and a single electron, have analytical solutions.
    \section{Research Significance}
        The potential impact of this research spans across many branches of physics. Not only is this research highly relevant in high precision magnetometry, but is also of interest in high energy physics and astrophysics. The ability to accurately measure the spectral line splitting when an atom is subjected to an external magnetic field will help serve as a test to the SM via the measurement of $g_\mu - 2$. This work will provide corrections to the high-precision $^3$He$^+$ magnetometer present at the Max Planck institute in Germany, with hopes of further lowering their uncertainty, allowing for an even more precise measurement of the $^3$He$^+$ magnetic moment. The direct determination of the nuclear magnetic moment of $^3$He$^+$ is of key importance to the independant calibration of $^3$He NMR probes. While these corrections are small, they play a significant role in the high-field limit, eventually becoming a dominant term that dictates the Zeeman splitting. The behaviour of high energy systems with large magnetic field changes drastically due to these contributions, allowing for the further study of the nature of neutron stars and other astrophysical objects.
    \section{Dissertation Structure}
        The main body of this disseration includes four chapters, of which this section concludes the first: the Introduction. The next chapter introduces theoretical methods and concepts that are key to understanding the later sections and main results of this thesis. It will cover concepts ranging from Atomic Units and the Schr$\ddot{o}$dinger equation, to double perturbation theory. Chapter \ref{sec:Zeeman-Effect} covers the main topic of the dissertation, the Zeeman effect, where the known effects are introduced and combined to give the higher-order Zeeman effect for hydrogenic systems. Afterwards, chapter \ref{sec:Conclusion} will give an overall conclusion to the dissertation and postulate areas of future study.\\

        Several apendices are included at the end of this thesis, which cover extra material which may be helpful to the reader. These sections are either provided as a convenience to the reader for a reminder of mathematical properties and tools, or display full derivations which are summarized in the relevant sections of the work.
