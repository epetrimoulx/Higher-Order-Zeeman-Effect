\chapter{Introduction}
    \section{Background and Significance}
        This dissertation investigates hydrogenic systems subjected to an external magnetic field, and plays a key role in the investigation of the $g_\mu - 2$ anomaly. According to the Dirac equation, a spin-$\frac{1}{2}$ particle such as the muon should possess a magnetic dipole moment with a g-factor of exactly 2. Deviations from this value arise from quantum corrections, which can be expressed in terms of the anomalous magnetic moment $a_\mu = \frac{g_\mu-2}{2}$. As long as a particle's g-factor remains close to 2 and consistent with theoretical expectations, it suggests a lack of internal structure. This result is expressed in terms of the g-factor $g_\mu$ and so long that the g-factor for a spin-$\frac{1}{2}$ particle is 2, indicates that the particle has no internal structure. The $g_\mu - 2$ anomaly is a discrepancy between experiment and QED, which can be partially corrected for with quantum field theory, but is yet to be properly explained to this day.
        \begin{align}
            a_\mu^{\text{SM}} = a_\mu^{\text{QED}} + a_\mu^{\text{EW}} + a_\mu^{\text{hadron}}\label{eq:g-2_Anomaly}
        \end{align}
        The first term in equation \eqref{eq:g-2_Anomaly} is the standard prediction from Dirac, and the second term is a correction from the electro-weak force. Both of these terms can be derived from first principles. The third term arises from the contribution of the strong nuclear force, in which there are theoretical predictions, but they contain a high amount of uncertainty. Because of this, the g-factor of the muon as well as for any atomic system remains a quantity that can only be determined via experiment. Further understanding the $g_\mu - 2$ result provides insight to the struture of the muon, and is currently an active area of research for new physics beyond the standard model. Current research being conducted at the Max Planck institute in Germany aims to measure the helion magnetic moment to a high degree of accuracy, in hopes of further understanding this anomaly. Their work involves a high precision magnetometer for helion, in which higher order corrections to the Zeeman effect contribute to the accuracy of their measurement \cite{Schneider_Sikora_Dickopf_Müller_Oreshkina_Rischka_Valuev_Ulmer_Walz_Harman_et_al._2022}.
        The Zeeman effect as a whole dates back to before the conception of quantum mechanics, and was one of many key experiments that motivated its discovery. The study of hydrogenic states in quantum mechanics provides a basis for the analysis of all the elements, ions, and isotopes throughout the universe and because they consist only of a nucleus and a single electron, have analytical solutions.\\

        The potential impact of this research spans across many branches of physics. Not only is this research highly relevant in high precision magnetometry, but is also of interest in high energy physics and astrophysics. The ability to accurately measure the spectral line splitting when an atom is subjected to an external magnetic field will help serve as a test to the SM via the measurement of $g_\mu - 2$. This work will provide corrections to the high-precision $^3$He$^+$ magnetometer present at the Max Planck institute in Germany, with hopes of further lowering their uncertainty. The magnetometer used in their experiment achieves exceptional sensitivity, approaching the sub-part-per-billion level, currently having a measurement accuracy of $\pm 0.26$Hz \cite{Schneider_Sikora_Dickopf_Müller_Oreshkina_Rischka_Valuev_Ulmer_Walz_Harman_et_al._2022}. At such high precision, corrections beyond the first-order Zeeman effect (including relativistic and second-order perturbative terms) become experimentally observable. This dissertation provides theoretical corrections that aim to reduce the systematic uncertainty of this magnetometer, allowing for a more accurate determination of the $^3$He$^+$ nuclear magnetic moment. The results formulated from this experiment will aid other key experiments in magnetometry, such as the independent calibration of $^3$He-based nuclear magnetic resonance (NMR) probes \cite{Farooq_Chupp_Grange_Tewsley-Booth_Flay_Kawall_Sachdeva_Winter_2020}. Such magnetometers exhibit a wide range of applications from MRI machines to fundamental physics experiments. While these higher order corrections are small, they play a significant role in the high-field limit, eventually becoming a dominant term that dictates the Zeeman splitting. The behaviour of high energy systems with large magnetic field changes drastically due to these contributions, allowing for the further study of the nature of neutron stars and other astrophysical objects. These astrophyical objects are capable of producing magnetic fields that can reach strengths exceeding $10^9$-$10^{11}$ Tesla. Under these conditions, nonlinear and relativistic corrections to the Zeeman interaction dominate the behavior of atomic systems. Accurate modeling of spectral line splitting in such regimes allows astronomers to infer magnetic field strengths and compositions of these objects with higher precision. The work presented here contributes to a more complete understanding of how matter behaves under some of the most extreme conditions in the universe.
    \section{Dissertation Structure}
        The main body of this disseration includes four chapters, of which this section concludes the first: the Introduction. The next chapter introduces theoretical methods and concepts that are key to understanding the later sections and main results of this thesis. It will cover concepts ranging from Atomic Units and the Schr$\ddot{o}$dinger equation, to double perturbation theory. Chapter \ref{sec:Zeeman-Effect} covers the main topic of the dissertation, the Zeeman effect, where the known effects are introduced and combined to give the higher-order Zeeman effect for hydrogenic systems. Afterwards, chapter \ref{sec:Conclusion} will give an overall conclusion to the dissertation and postulate areas of future study.\\

        Several apendices are included at the end of this thesis, which cover extra material which may be helpful to the reader. These sections are either provided as a convenience to the reader for a reminder of mathematical properties and tools, or display full derivations which are summarized in the relevant sections of the work.
