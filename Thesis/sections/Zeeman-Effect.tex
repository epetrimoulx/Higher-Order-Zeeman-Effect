\chapter{The Higher Order Zeeman Effect}\label{sec:Zeeman-Effect}
    \section{Overview}
        Now that the proper theory has been introduced, we now move to the main section of the thesis. Here I will introduce the Zeeman Effect, talk about the motivation and direct application of my work, and then move on to the higher order Zeeman Effect. The main focus of this chapter is to show the effect of the quadratic Zeeman Effect, and show how using the magnetic dipole operator in conjunction with the relativistic corrections to $^3$He$^+$ can yield a Cubic Zeeman Effect. The effects of both the quadratic and cubic corrections will be discussed in great detail, and the impact of the effect on high precision measurements will be displayed for various magnetic field strengths.\\

        % PUT SOMETHING HERE FOR APPLICATIONS IN AMO
        I will begin in Sec.~\ref{sec:history} where I will discuss the history of the Zeeman effect, its origins and discovery. Afterwards I will move towards the motivation for the project in Sec.~\ref{sec:motivation}. Here I will discuss some current experiments in the field such as the $g-2$ experiment being conducted at the Max Planck Institute by Klaus Blaum's research group as well as the applications to high-precision magnetometry. I will then discuss further some additional applications in the field of atomic physics such as $\dots$. I will then discuss in Sec.~\ref{sec:Ordinary_Zeeman} the ordinary Zeeman effect, introducing its theory and application to atomic systems such as $^3$He$^+$. After introducing the ordinary Zeeman effect I will move forward to discuss the quadratic Zeeman effect, where I will derive it using the canonical momentum and give a description of its impact on an atom subjected to a magnetic field. Moving towards higher order systems, I will then introduce the main topic of this thesis, the cubic Zeeman effect. Starting with the effects that contribute to the cubic Zeeman effect such as the magnetic dipole operator in Sec.~\ref{sec:magnetic_dipole_operator} and relativistic corrections to the $^3$He$^+$ ion in Sec.~\ref{sec:Relativistic_Correction} I will show how these effects can be combined to yield a $B^3$ contribution to the energy splitting within the presence of an external magnetic field.
    \section{History}\label{sec:history}
    \section{Motivation}\label{sec:motivation}
        \subsection{The $g-2$ experiment}\label{sec:g-2}
        \subsection{High-precision magnetometry}\label{sec:High-Precision-magnetometry}
        \subsection{Connection to Atomic Physics}\label{sec:Connection_To_AMO}
    \section{The Zeeman effect in Hydrogen}\label{sec:Ordinary_Zeeman}
    \section{The Zeeman effect in Helium-3}\label{sec:Ordinary_Zeeman_He3}
    \section{The higher-order Zeeman effect}\label{sec:Higher_Zeeman}
        \subsection{The quadratic Zeeman effect}\label{sec:quadratic_zeeman}
            The non-relativistic Hamiltonian for an $n$ electron system in atomic units can be expressed as follows
            \begin{align}
                \hat{H} = \sum_{i = 1}^n \frac{\left(\vec{p}_i - e\vec{A}_i \right)^2}{2m} + V
            \end{align}

            \noindent using the cannonical momentum instead of the classical momentum is essential to account for electromagnetic interactions. The vector potential $\vec{A}$ can be described in terms of the magnetic field

            \begin{align}
                \vec{A} = \frac{B}{2} \left( y\hat{x} - x\hat{y} \right)
            \end{align}

            \noindent which when expanded gives us the operator corresponding to the quadratic Zeeman effect

            \begin{align}
                \hat{H}_Z^{(2)} = \frac{B^2 e^2}{8m} \sum_{i = 1}^n (x_i^2 + y_i^2)
            \end{align}

            \noindent This operator can be expressed in spherical coordinates, written in terms of Legendre Polynomials

            \begin{align}
                \hat{H}^{(2)}_Z = \frac{B^2 e^2}{12m} \sum_{i = 1}^n r_i^2 \left( P_0(\cos \theta) - P_2(\cos \theta) \right)
            \end{align}

            \noindent $^3$He$^+$ is a system which contains only one electron, so we can drop the summation which accounts for all electrons to get our final quadratic Zeeman operator

            \begin{align}
                \hat{H}^{(2)}_Z(^3\text{He}^+) = \frac{B^2 e^2}{12m} r^2 \left( P_0(\cos \theta) - P_2(\cos \theta) \right)
            \end{align}

            % NOTE: Talk about here why the quadratic Zeeman effect provides a shift to the energy for both possible values of magnetic quantum number evenly, so the overall energy is changed slightly, but the difference between the energy split states remains the same.
        \subsection{The magnetic dipole moment operator}\label{sec:magnetic_dipole_operator}
            The magnetic dipole moment operator represents the interaction of a magnetic dipole moment with an external magnetic field. It is described via the following relation

            \begin{align}
                Q_{M1} = \mu_B \left( 1 - \frac{2p^2}{3m^2 c^2} + \frac{Ze^2}{3mc^2r} \right) \vec{\sigma} \cdot \vec{B}
            \end{align}

            \noindent Where $\mu_B$ is the Bohr magneton
            
            \begin{align}
                \mu_B = \frac{e \hbar}{2mc}
            \end{align}

            \noindent The second term in the brackets of the magnetic dipole moment operator accounts for the relativistic correction to the kinetic energy of the electron, and the third term is the potential energy due to the Coulomb interaction between the electron and the nucleus. The first term corresponds to the ordinary Zeeman Effect, which does not contribute to the sum over states due to orthogonality.

            The ordinary Zeeman effect contributes to $Q_{M1}$ in $^3$He$^+$ because it has non-zero spin due to the missing electron. For systems such as $^3$He, the ordinary Zeeman effect will not contribute.

        \subsection{The relativistic correction to $^3$He$^+$}\label{sec:Relativistic_Correction}
            Combining the magnetic dipole moment with the quadratic Zeeman operator, we can write down the relativistic
            corrections for $^3$He$^+$. Written in terms of pseudostates the relativistic correction is

            \begin{align}
                C_{\text{rel}}^{(2)} =\sum_{\substack{n = -\infty \\ n \neq 0}}^{\infty}
                \frac{\langle \psi_0 \vert H_Z^{(2)} \vert \psi_n \rangle \langle \psi_n \vert Q_{M1} \vert \psi_0 \rangle}{E_0 - E_n}
            \end{align}

            % NOTE: Ask drake again how the relativistic correction relates to the overall higher order Zeeman Effect. Description needs to be here

    \section{Results}





