% FIX GRAPHS NOT APPEARING
\chapter{Derivation of the Recursion Relations}
    The aim of this section is to guide the reader through a more in depth derivation of the recursion relations for the quadratic zeeman effect as well as the relativistic magnetic dipole operator perturbations to the Hamiltonian. Starting with the first order perturbation equation
    \begin{align}
        \left(H^0 - E^0 \right) \vert \psi^1 \rangle = -\left( V - E^1  \right) \vert \psi^0 \rangle\;,
    \end{align}
    \noindent $E^1$ can be found by multiplying through by $\langle \psi^0 \vert$
    \begin{align}
        \langle \psi^0 \vert H^0 - E^0 \vert \psi^1 \rangle &= -\langle \psi^0 \vert V - E^1 \vert \psi^0 \rangle \;.
    \end{align}
    \noindent The first term is $0$ due to the orthogonality of states which is imposed and discussed in Sec.~\ref{sec:Zeeman-Effect}. The expression now reads 
    \begin{align*}
        &-\langle \psi^0 \vert V - E^1 \vert \psi^0 \rangle = 0 \\
        &\langle \psi^0 \vert V \vert \psi^0 \rangle =  \langle \psi^0 \vert E^1 \vert \psi^0 \rangle\\
        &\langle \psi^0 \vert V \vert \psi^0 \rangle =  E^1 \langle \psi^0 \vert \psi^0 \rangle
    \end{align*}
    \noindent So 
    \begin{align}
        E^1 = \langle \psi^0 \vert V \vert \psi^0 \rangle\;.
    \end{align}
    \noindent For the $r^2$ perturbation, one gets 
    \begin{align*}
        E^1 &= \int^\infty_0\int_0^\pi  \int_0^{2\pi} r^2 \sin\theta \frac{2Z^{\frac{3}{2}} e^{-Zr}}{\sqrt{4\pi}} \; \frac{B^2e^2}{12m}r^2\;  \frac{2Z^{\frac{3}{2}} e^{-Zr}}{\sqrt{4\pi}} \;dr\; d\theta\; d\phi\\ \nonumber\\
        E^1 &= {4\pi} \int_0^\infty \frac{Z^3 e^{-2Zr}}{\pi} \; \frac{B^2e^2}{12m}r^4 \;dr\\ \nonumber\\
        E^1 &= \frac{Z^3 B^2 e^2}{3m} \frac{4!}{(2Z)^5} \\ \nonumber \\
        E^1 &= \frac{B^2 e^2}{4Z^2m}
    \end{align*}
    \noindent Here, we let $\frac{B^2 e^2}{m} \equiv \gamma^2$ \cite{Killingbeck_1979}, and we end up with the final result for $E^1$ to be 
    \begin{align}
        E^1 = \frac{1}{4Z^2} \gamma^2\;.
    \end{align}
    \noindent The perturbing potential $V$ also is written in terms of $\gamma^2$
    \begin{align}
        V = \frac{B^2 e^2}{12m} r^2 = \frac{1}{12} \gamma^2 r^2
    \end{align}
    \noindent Assuming that $\vert \psi^1 \rangle$ is of the form of a power series 
    \begin{align}
        \vert \psi^1 \rangle = \sum_{j = 0}^\infty Z^{\frac{3}{2}} a_j r^j e^{-Zr}
    \end{align}
    \noindent and substituting both $E^1$ and $\vert \psi^1 \rangle$ into the perturbation equation
    \begin{align}
        \left(H^0 - E^0 \right) \sum_{j = 0}^\infty Z^{\frac{3}{2}} a_j r^j e^{-Zr} = \left( \frac{1}{12} \gamma^2 r^2 - \frac{1}{4Z^2} \gamma^2 \right) \frac{Z^{\frac{3}{2}} e^{-Zr}}{\sqrt{\pi}}\;.
    \end{align}
    \noindent Substituting the Hamiltonian in atomic units 
    \begin{align*}
        \left(\frac{1}{2}\nabla^2 + \frac{Z}{r} - E_0\right) \sum_{j = 0}^\infty a_j r^j e^{-Zr} = \left( \frac{1}{12} \gamma^2 r^2 - \frac{1}{4Z^2} \gamma^2 \right) \frac{Z^{\frac{3}{2}} e^{-Zr}}{\sqrt{\pi}}
    \end{align*}
    \begin{align*}
        \frac{1}{2} \frac{1}{r^2} \frac{\partial}{\partial r} \left(r^2 \frac{\partial}{\partial r} \right) \sum_{j = 0}^\infty a_j r^j e^{-Zr} + Z \sum_{j = 0}^\infty a_j r^{j - 1} e^{-Zr} - E^0 \sum_{j = 0}^\infty a_j r^j e^{-Zr} \nonumber\\ = \left( \frac{1}{12} \gamma^2 r^2 - \frac{1}{4Z^2} \gamma^2 \right) \frac{Z^{\frac{3}{2}} e^{-Zr}}{\sqrt{\pi}}\;.
    \end{align*}
    \begin{align*}
        \left( \frac{Z^2}{2} - E^0\right) \sum_{j = 0}^\infty a_j r^j - Zr \sum_{j = 0}^\infty a_j r^{j -2} (j + 1) + \sum_{j = 0}^\infty \frac{j(j+1)}{2} a_j r^{j - 2} \nonumber\\+ Z \sum_{j = 0}^\infty a_j r^{j -1} - E^0 \sum_{j = 0}^\infty a_j r^j = \frac{1}{12} \gamma^2 \left(r^2 - \frac{3}{Z^2} \right) \frac{1}{\sqrt{\pi}}.
    \end{align*}
    \noindent Since $E^0 = \frac{Z^2}{2}$, the first term is zero and the equation becomes 
    \begin{align*}
        -Zr \sum_{j = 0}^\infty a_j r^{j - 2} (j + 1) + \sum_{j = 0}^\infty \frac{j(j+1)}{2} a_j r^{j - 2} + Z \sum_{j = 0}^\infty a_j r^{j-1} = \frac{1}{12\sqrt{\pi}} \gamma^2 \left(r^2 - \frac{3}{Z^2} \right).
    \end{align*}
    \noindent Performing a shift of summation index to group all of the summations together 
    \begin{align}
        \sum_{j = 0}^\infty \left[ Z(j-1) a_{j-1} - \frac{j(j+1)}{2} a_j \right] r^{j-2} = -\frac{1}{12} \gamma^2 \left(r^2 - \frac{3}{Z^2} \right) \frac{1}{\sqrt{\pi}}.
    \end{align}
    \noindent Which is equation \eqref{eq:Result_r^2_sum} in Sec.~\ref{sec:quadratic_zeeman}. Writing out the first few recursion relations by grouping the powers of $r$ together 
    \begin{align*}
        j &= 0 & -Za_{-1} = 0& &a_{-1} = 0\\ \nonumber
        j &= 1 & -a_1 = 0& &a_{1} = 0\\ \nonumber
        j &= 2 & Za_1 - 3a_2 = \frac{1}{12} \gamma^2 \frac{3}{Z^2}\frac{1}{\sqrt{\pi}} & &a_2 = -\frac{1}{12} \gamma^2 \frac{1}{Z^2 \sqrt{\pi}}\\ \nonumber
        j &= 3 & 2Za_2 - 6a_3 = 0&& a_3 = -\frac{1}{32Z\sqrt{\pi}} \gamma^2\\ \nonumber
        j &= 4 & 3Za_3 - 10a_4 = -\frac{1}{12} \gamma^2 \frac{1}{\sqrt{\pi}} && a_4 = 0\\ \nonumber
        j &= 5 & 4Z_4 - 15a_5 = 0 && a_5 = 0 \\ \nonumber
        & \vdots & \vdots && \vdots \\
    \end{align*}
    \noindent It is clearly seen above that all other cases of $a_j$ must be zero after $j = 3$. The recursion relation has been solved and a closed form solution for $\vert \psi^1 \rangle$ can be written in terms of a power series in $r$ using $a_0$, $a_2$, and $a_3$. $a_0$ is an undetermined coefficient here that does not appear in the recursion relation. it is determined by imposing the normalization condition which is shown in Sec.~\ref{sec:quadratic_zeeman}.\\

    The identical process is carried out for the relativistic magnetic dipole moment operator perturbation, with the only difference being that the perturbing potential has changed. No operators act on the perturbing potential throughout the perturbation expansion, so the result will remain the same as in the example above. Applying the same process the first order correction to the wavefunction for the relativistic magnetic dipole moment operator is 
    \begin{align}
        \sum_{j = 0}^\infty \left[ Z(j-1)a_{j-1} - \frac{j(j+1)}{2} a_j \right] r^{j-2} = -\left(- \mu_B Z \alpha^2 a_0 \vec{\sigma} \cdot \vec{B} \frac{1}{r} - E^1 \right) \frac{1}{\sqrt{\pi}}\;.
    \end{align}
    \noindent $E^1$ is determined in the standard way\footnote{Note that $a_0$ here is for the Bohr radius. $a_0$ is also the zeroeth term in the summation, but the form of the summation here dictates that there is no $a_0$ term present.}
    \begin{align*}
        E^1 &= \langle \psi^0 \vert - \mu_B Z \alpha^2 a_0 \vec{\sigma} \cdot \vec{B} \frac{1}{r} \vert \psi^0 \rangle \\ \nonumber \\
        E^1 &= -\int_0^\infty \int_0^{2\pi} \int_0^\pi r^2 \sin \theta \frac{2Z^{\frac{3}{2}} e^{-Zr}}{\sqrt{4\pi}} \mu_B Z \alpha^2 a_0 \vec{\sigma} \cdot \vec{B} \frac{1}{r} \frac{2Z^{\frac{3}{2}} e^{-Zr}}{\sqrt{4\pi}}\;dr\;d\theta\;d\phi  \\\nonumber \\
        E^1 &= -4\pi \int_0^\infty r \sin\theta \frac{Z^3 e^{-2Zr}}{\pi}  \mu_B Z \alpha^2 a_0 \vec{\sigma} \cdot \vec{B} \;dr\;d\theta\;d\phi\\\nonumber\\
        E^1 &= - 4\pi Z^4 \alpha^2 a_0 \mu_B \vec{\sigma} \cdot \vec{B} \int_0^\infty r e^{-2Zr} dr \\\nonumber \\
        E^1 &= - 4\pi Z^4 \alpha^2 a_0 \mu_B \vec{\sigma} \cdot \vec{B} \frac{1!}{(2Z)^2} \\\nonumber \\
        E^1 &= - Z^2 \alpha^2 a_0 \mu_B \vec{\sigma} \cdot \vec{B}
    \end{align*}
    \noindent So the final expression for the recursion relations for the relativistic magnetic dipole moment operator is 
    \begin{align}
        \sum_{j = 0}^\infty \left[ Z(j-1)a_{j-1} - \frac{j(j+1)}{2} a_j \right] r^{j-2} = \alpha^2 Z a_0 \mu_B \left(\frac{1}{r} - Z \right) \frac{1}{\sqrt{\pi}}\; \vec{\sigma} \cdot \vec{B}\;.
    \end{align}
    \noindent The series converges after $j = 1$, as shown below.
    \begin{align*}
        j &= 0 & -Za_{-1} = 0& &a_{-1} = 0\\ \nonumber
        j &= 1 & -a_1 = \alpha^2 Z a_0 \mu_B  \frac{1}{\sqrt{\pi}}\; \vec{\sigma} \cdot \vec{B} & &a_{1} = - \alpha^2 Z a_0 \mu_B \frac{1}{\sqrt{\pi}}\; \vec{\sigma} \cdot \vec{B}\\ \nonumber
        j &= 2 & Za_1 - 3a_2 = - \alpha^2 Z^2 a_0 \mu_B \frac{1}{\sqrt{\pi}}\; \vec{\sigma} \cdot \vec{B} && a_2 = 0\\ \nonumber
        j &= 3 & 2Za_2 - 6a_3 = 0&& a_3 = 0\\ \nonumber
        j &= 4 & 3Za_3 - 10a_4 =0 && a_4 = 0\\ \nonumber
        j &= 5 & 4Z_4 - 15a_5 = 0 && a_5 = 0 \\ \nonumber
        & \vdots & \vdots && \vdots \\
    \end{align*}
    \noindent Thus the recursion relations for the relativistic magnetic dipole moment operator and the quadratic Zeeman operator have been solved, and thier respective first order corrections to the hydrogenic wavefunction have been calculated.

\chapter{Left-acting Operators} \label{sec:Hermitian_Left}
    This section contains the proof for the ability of hermitian operators to act to the left as well as to the right. The specific case discussed here uses the Hamiltonian, but this can be similarly shown for any hermitian operator. Given the matrix element
    \begin{align}
        \langle \psi \vert \hat{H} \vert \phi \rangle &= \int^\infty_{-\infty} \psi^*(x) \hat{H} \phi(x) dx\\
        \langle \psi \vert \hat{H} \vert \phi \rangle &= \int^\infty_{-\infty} \psi^*(x) -i\hbar\frac{d \phi(x)}{dx} dx\;,
    \end{align}
    \noindent applying integration by parts gives 
    \begin{align}
        \langle \psi \vert \hat{H} \vert \phi \rangle &= \left[\psi^*(x) \phi(x) \right]^\infty_{-\infty} - \int_{-\infty}^\infty -i\hbar \frac{d\psi^*(x)}{dx} \phi(x) dx\\
        \langle \psi \vert \hat{H} \vert \phi \rangle &= \int_{-\infty}^\infty i\hbar \frac{d\psi^*(x)}{dx} \phi(x) dx\;.
    \end{align}
    \noindent Which is the same as writing
    \begin{align}
        \langle \psi \vert \hat{H} \vert \phi \rangle = \int^\infty_{-\infty} \phi(x) \hat{H} \psi^*(x) dx
    \end{align}
    \noindent So it is shown that the Hamiltonian is capable of acting to the right or to the left, and it is up to the user which way they want to apply the operation.
\chapter{Special Functions}
    \section{Overview}
        This section of the appendix discusses the special functions used throughout this thesis, providing their definitions and properties in greater detail than discussed in the main body of this paper. The introduction to these functions is brief, and are placed here for convenience of the reader. Sec~.\ref{sec:Gamma_Function} introduces the gamma function and its properties, and its use in providing a closed solution to many of the integrals present in the matrix elements discussed in Sec~.\ref{sec:Integration_Techniques} is highlighted. Sections \ref{sec:Laguerre_Polynomial} and \ref{sec:Spherical_Harmonics} introduce the Laguerre polynomials and the spherical harmonic function respectively, which are key to the solution of hydrogenic wavefunctions used in this thesis. Additionally, the confluent hypergeometric function is discussed in Sec~.\ref{sec:Confluent_Hypergeometric_Function}, where an alternative formulation of the radial wavefunctions for hydrogen is given. Finally, the chapter concludes with the description of Legendre polynomials, which can be used in a similar manner to the spherical harmonics due to their property of forming a complete set of orthogonal functions.
    \section{The gamma function} \label{sec:Gamma_Function}
        The gamma function $\Gamma(z)$ is an extension of the factorial function into the complex plane. 
        \begin{align}
            \Gamma(z) \in \mathbb{C} 
        \end{align}
        For the case where the input parameter $z \in \mathbb{Z}^+$, the function is equal to \cite{Arfken_Weber_Arfken_Weber_2008}
        \begin{align}
            \Gamma(n) = (n - 1)!\;.
        \end{align}
        \noindent This is a relation of key importance in Sec.~\ref{sec:The_Radial_Part}, where the radial integral is replaced with a factorial function for positive integers of $j$ in the sum.\\

        The gamma function also has an integral definition, which is known as the Euler integral
        \begin{align}
            \Gamma(z) = \int_0^\infty e^{-t} t^{z-1} dt, \hspace{1cm} \mathfrak{R}e(z) > 0\;.
        \end{align}
        \noindent This integral is defined as long as the real part of $z$ is greater than zero. This integral closely resembles the radial integrals present in Sec.~\ref{sec:The_Radial_Part}, which are of the form
        \begin{align}
            I_{\text{radial}} = \int_0^\infty r^j e^{-\alpha r} dr
        \end{align}
        \noindent To fit the gamma function, perform a change of variables $t = \alpha r$, $dt = \alpha dr$.
        \begin{align}
            I_{\text{radial}} &= \int_0^\infty \left( \frac{t}{\alpha} \right)^j e^{-t} \frac{1}{\alpha} dt
        \end{align}
        \begin{align}
            I_{\text{radial}} &= \frac{1}{\alpha^{j+1}}\int_0^\infty {t}^j e^{-t} dt
        \end{align}
        \noindent So the solution to the radial integral is shown to be 
        \begin{align}
            I_{\text{radial}} = \frac{\Gamma(j+1)}{\alpha^{j+1}}
        \end{align}
        \noindent if $j \in \mathbb{Z}^+$, the relationship between the gamma function and the factorial function can be used. The values of $j$ are the summation indices for each power of $r$\footnote{See sections \ref{sec:quadratic_zeeman} and \ref{sec:magnetic_dipole_operator} for the summation expressions for each.}. Since the summation indices are constrained to a set of integers from 0 to infinity, it can be said that 
        \begin{align}
            \forall j, j \in Z^{0+}
        \end{align}
        \noindent Additionally, the $0^{\text{th}}$ power of $r$ (corresponding to $j = 0$) does not appear in the recursion relations of the expanded sum in \ref{sec:magnetic_dipole_operator}, and is zero in \ref{sec:quadratic_zeeman}. This further restricts the set of $j$ indices in both problems to 
        \begin{align}
            \forall j, j\in \mathbb{N}
        \end{align}
        \noindent Since the set of natural numbers only contains positive integers, the euler integral only evaluates values for positive integer $j$. The gamma function by its original definition is then replaced by the factorial function, giving the solution to the radial integrals used in sections \ref{sec:The_Radial_Part}, \ref{sec:quadratic_zeeman} and \ref{sec:magnetic_dipole_operator}. 
        \begin{figure}[ht!]
            \centering
            \resizebox{0.8\linewidth}{!}{%% Creator: Matplotlib, PGF backend
%%
%% To include the figure in your LaTeX document, write
%%   \input{<filename>.pgf}
%%
%% Make sure the required packages are loaded in your preamble
%%   \usepackage{pgf}
%%
%% Also ensure that all the required font packages are loaded; for instance,
%% the lmodern package is sometimes necessary when using math font.
%%   \usepackage{lmodern}
%%
%% Figures using additional raster images can only be included by \input if
%% they are in the same directory as the main LaTeX file. For loading figures
%% from other directories you can use the `import` package
%%   \usepackage{import}
%%
%% and then include the figures with
%%   \import{<path to file>}{<filename>.pgf}
%%
%% Matplotlib used the following preamble
%%   \def\mathdefault#1{#1}
%%   \everymath=\expandafter{\the\everymath\displaystyle}
%%   
%%   \ifdefined\pdftexversion\else  % non-pdftex case.
%%     \usepackage{fontspec}
%%     \setmainfont{DejaVuSerif.ttf}[Path=\detokenize{/Users/evanpetrimoulx/.pyenv/versions/3.12.6/lib/python3.12/site-packages/matplotlib/mpl-data/fonts/ttf/}]
%%     \setsansfont{Arial.ttf}[Path=\detokenize{/System/Library/Fonts/Supplemental/}]
%%     \setmonofont{DejaVuSansMono.ttf}[Path=\detokenize{/Users/evanpetrimoulx/.pyenv/versions/3.12.6/lib/python3.12/site-packages/matplotlib/mpl-data/fonts/ttf/}]
%%   \fi
%%   \makeatletter\@ifpackageloaded{underscore}{}{\usepackage[strings]{underscore}}\makeatother
%%
\begingroup%
\makeatletter%
\begin{pgfpicture}%
\pgfpathrectangle{\pgfpointorigin}{\pgfqpoint{6.400000in}{4.800000in}}%
\pgfusepath{use as bounding box, clip}%
\begin{pgfscope}%
\pgfsetbuttcap%
\pgfsetmiterjoin%
\definecolor{currentfill}{rgb}{1.000000,1.000000,1.000000}%
\pgfsetfillcolor{currentfill}%
\pgfsetlinewidth{0.000000pt}%
\definecolor{currentstroke}{rgb}{1.000000,1.000000,1.000000}%
\pgfsetstrokecolor{currentstroke}%
\pgfsetdash{}{0pt}%
\pgfpathmoveto{\pgfqpoint{0.000000in}{0.000000in}}%
\pgfpathlineto{\pgfqpoint{6.400000in}{0.000000in}}%
\pgfpathlineto{\pgfqpoint{6.400000in}{4.800000in}}%
\pgfpathlineto{\pgfqpoint{0.000000in}{4.800000in}}%
\pgfpathlineto{\pgfqpoint{0.000000in}{0.000000in}}%
\pgfpathclose%
\pgfusepath{fill}%
\end{pgfscope}%
\end{pgfpicture}%
\makeatother%
\endgroup%
}
            \caption{Graph of the gamma function compared to the factorial function.}
            \label{img:Gamma_fn_graph}
        \end{figure}
    \section{Legendre Polynomials} \label{sec:Legendre_Polynomial}
        The Legendre equation is 
        \begin{align}
            (1 - x^2)P^{\prime \prime}(x) -2xP^\prime(x) + \lambda P(x) = 0\;.
        \end{align}
        \noindent This differential equation can be solved with a series solution about $x = 0$ and has a radius of convergence of 1. If $\lambda$ has the value $l(l+1)$, where $l \in \mathbb{Z}$, the series truncates after $x^{l}$. The series solution to this ODE is 
        \begin{align}
            g(x, t) = \sum_{l = 0}^\infty P_l(x) t^l 
        \end{align}
        \noindent Where $g(x, t)$ is the generating function which produces the possible solutions to the Legendre Equation. Each solution to the Legendre equation generated by $g(x, t)$ is called a Legendre polynomial. These polynomials can be produced with the following equation
        \begin{align}
            P_l(x) = \sum_{n = 0}^{\lfloor \frac{l}{2} \rfloor} (-1)^k \frac{(2l - 2k)!}{2^n k! (n - k)!(n - 2k)!}x^{n-2k}
        \end{align}
        \noindent where $\lfloor x \rfloor$ is the floor function, which rounds down to the nearest integer value. This formulation is of key importance for programmed implementation, since a discrete formula is simple to compute. A more helpful definition of the Legendre polynomial comes from the Rodriguez formula \cite{Arfken_Weber_Arfken_Weber_2008}
        \begin{align}
            P_n(x) = \frac{1}{2^n n!} \left( \frac{d}{dx} \right)^n (x^2 - 1)^n
        \end{align}
        \noindent This formulation is much more useful for analytical work. The first few Legendre Polynomials are given below
        \begin{table}
            \centering
            \begin{tabular}{l}
                Legendre Polynomials\\
                \hline
                $P_0(x) = 1$\\
                $P_1(x) = x$\\
                $P_2(x) = \frac{1}{2}\left(3x^2 - 1\right)$\\
                $P_3(x) = \frac{1}{2} \left(5x^3 - 3x\right)$ \\
                $P_4(x) = \frac{1}{8} \left(35x^4 - 30x^2 + 3 \right)$\\
                \hline
            \end{tabular}
            \caption{Table of the first few Legendre Polynomials}
            \label{tab:Legendre_Polynomials}
        \end{table}
        \noindent Another key feature of the Legendre polynomials is that they form a complete set of orthogonal basis functions, meaning that for each value of $l$, the associated Legendre polynomial is orthogonal to all other Legendre polynomials with different $l$.
        \begin{align}
            \int_{-1}^1 P_l(x) P_l^\prime(x) = \frac{2}{2l + 1}\delta_{l,l^\prime}
        \end{align}
        \begin{figure}[ht!]
            \centering
            \resizebox{0.8\linewidth}{!}{%% Creator: Matplotlib, PGF backend
%%
%% To include the figure in your LaTeX document, write
%%   \input{<filename>.pgf}
%%
%% Make sure the required packages are loaded in your preamble
%%   \usepackage{pgf}
%%
%% Also ensure that all the required font packages are loaded; for instance,
%% the lmodern package is sometimes necessary when using math font.
%%   \usepackage{lmodern}
%%
%% Figures using additional raster images can only be included by \input if
%% they are in the same directory as the main LaTeX file. For loading figures
%% from other directories you can use the `import` package
%%   \usepackage{import}
%%
%% and then include the figures with
%%   \import{<path to file>}{<filename>.pgf}
%%
%% Matplotlib used the following preamble
%%   \def\mathdefault#1{#1}
%%   \everymath=\expandafter{\the\everymath\displaystyle}
%%   
%%   \ifdefined\pdftexversion\else  % non-pdftex case.
%%     \usepackage{fontspec}
%%     \setmainfont{DejaVuSerif.ttf}[Path=\detokenize{/Users/evanpetrimoulx/.pyenv/versions/3.12.6/lib/python3.12/site-packages/matplotlib/mpl-data/fonts/ttf/}]
%%     \setsansfont{Arial.ttf}[Path=\detokenize{/System/Library/Fonts/Supplemental/}]
%%     \setmonofont{DejaVuSansMono.ttf}[Path=\detokenize{/Users/evanpetrimoulx/.pyenv/versions/3.12.6/lib/python3.12/site-packages/matplotlib/mpl-data/fonts/ttf/}]
%%   \fi
%%   \makeatletter\@ifpackageloaded{underscore}{}{\usepackage[strings]{underscore}}\makeatother
%%
\begingroup%
\makeatletter%
\begin{pgfpicture}%
\pgfpathrectangle{\pgfpointorigin}{\pgfqpoint{12.000000in}{8.000000in}}%
\pgfusepath{use as bounding box, clip}%
\begin{pgfscope}%
\pgfsetbuttcap%
\pgfsetmiterjoin%
\definecolor{currentfill}{rgb}{1.000000,1.000000,1.000000}%
\pgfsetfillcolor{currentfill}%
\pgfsetlinewidth{0.000000pt}%
\definecolor{currentstroke}{rgb}{1.000000,1.000000,1.000000}%
\pgfsetstrokecolor{currentstroke}%
\pgfsetdash{}{0pt}%
\pgfpathmoveto{\pgfqpoint{0.000000in}{0.000000in}}%
\pgfpathlineto{\pgfqpoint{12.000000in}{0.000000in}}%
\pgfpathlineto{\pgfqpoint{12.000000in}{8.000000in}}%
\pgfpathlineto{\pgfqpoint{0.000000in}{8.000000in}}%
\pgfpathlineto{\pgfqpoint{0.000000in}{0.000000in}}%
\pgfpathclose%
\pgfusepath{fill}%
\end{pgfscope}%
\begin{pgfscope}%
\pgfsetbuttcap%
\pgfsetmiterjoin%
\definecolor{currentfill}{rgb}{1.000000,1.000000,1.000000}%
\pgfsetfillcolor{currentfill}%
\pgfsetlinewidth{0.000000pt}%
\definecolor{currentstroke}{rgb}{0.000000,0.000000,0.000000}%
\pgfsetstrokecolor{currentstroke}%
\pgfsetstrokeopacity{0.000000}%
\pgfsetdash{}{0pt}%
\pgfpathmoveto{\pgfqpoint{1.500000in}{0.880000in}}%
\pgfpathlineto{\pgfqpoint{10.800000in}{0.880000in}}%
\pgfpathlineto{\pgfqpoint{10.800000in}{7.040000in}}%
\pgfpathlineto{\pgfqpoint{1.500000in}{7.040000in}}%
\pgfpathlineto{\pgfqpoint{1.500000in}{0.880000in}}%
\pgfpathclose%
\pgfusepath{fill}%
\end{pgfscope}%
\begin{pgfscope}%
\pgfpathrectangle{\pgfqpoint{1.500000in}{0.880000in}}{\pgfqpoint{9.300000in}{6.160000in}}%
\pgfusepath{clip}%
\pgfsetroundcap%
\pgfsetroundjoin%
\pgfsetlinewidth{0.803000pt}%
\definecolor{currentstroke}{rgb}{0.800000,0.800000,0.800000}%
\pgfsetstrokecolor{currentstroke}%
\pgfsetdash{}{0pt}%
\pgfpathmoveto{\pgfqpoint{1.922727in}{0.880000in}}%
\pgfpathlineto{\pgfqpoint{1.922727in}{7.040000in}}%
\pgfusepath{stroke}%
\end{pgfscope}%
\begin{pgfscope}%
\definecolor{textcolor}{rgb}{0.150000,0.150000,0.150000}%
\pgfsetstrokecolor{textcolor}%
\pgfsetfillcolor{textcolor}%
\pgftext[x=1.922727in,y=0.782778in,,top]{\color{textcolor}{\sffamily\fontsize{10.000000}{12.000000}\selectfont\catcode`\^=\active\def^{\ifmmode\sp\else\^{}\fi}\catcode`\%=\active\def%{\%}\ensuremath{-}1.00}}%
\end{pgfscope}%
\begin{pgfscope}%
\pgfpathrectangle{\pgfqpoint{1.500000in}{0.880000in}}{\pgfqpoint{9.300000in}{6.160000in}}%
\pgfusepath{clip}%
\pgfsetroundcap%
\pgfsetroundjoin%
\pgfsetlinewidth{0.803000pt}%
\definecolor{currentstroke}{rgb}{0.800000,0.800000,0.800000}%
\pgfsetstrokecolor{currentstroke}%
\pgfsetdash{}{0pt}%
\pgfpathmoveto{\pgfqpoint{2.979545in}{0.880000in}}%
\pgfpathlineto{\pgfqpoint{2.979545in}{7.040000in}}%
\pgfusepath{stroke}%
\end{pgfscope}%
\begin{pgfscope}%
\definecolor{textcolor}{rgb}{0.150000,0.150000,0.150000}%
\pgfsetstrokecolor{textcolor}%
\pgfsetfillcolor{textcolor}%
\pgftext[x=2.979545in,y=0.782778in,,top]{\color{textcolor}{\sffamily\fontsize{10.000000}{12.000000}\selectfont\catcode`\^=\active\def^{\ifmmode\sp\else\^{}\fi}\catcode`\%=\active\def%{\%}\ensuremath{-}0.75}}%
\end{pgfscope}%
\begin{pgfscope}%
\pgfpathrectangle{\pgfqpoint{1.500000in}{0.880000in}}{\pgfqpoint{9.300000in}{6.160000in}}%
\pgfusepath{clip}%
\pgfsetroundcap%
\pgfsetroundjoin%
\pgfsetlinewidth{0.803000pt}%
\definecolor{currentstroke}{rgb}{0.800000,0.800000,0.800000}%
\pgfsetstrokecolor{currentstroke}%
\pgfsetdash{}{0pt}%
\pgfpathmoveto{\pgfqpoint{4.036364in}{0.880000in}}%
\pgfpathlineto{\pgfqpoint{4.036364in}{7.040000in}}%
\pgfusepath{stroke}%
\end{pgfscope}%
\begin{pgfscope}%
\definecolor{textcolor}{rgb}{0.150000,0.150000,0.150000}%
\pgfsetstrokecolor{textcolor}%
\pgfsetfillcolor{textcolor}%
\pgftext[x=4.036364in,y=0.782778in,,top]{\color{textcolor}{\sffamily\fontsize{10.000000}{12.000000}\selectfont\catcode`\^=\active\def^{\ifmmode\sp\else\^{}\fi}\catcode`\%=\active\def%{\%}\ensuremath{-}0.50}}%
\end{pgfscope}%
\begin{pgfscope}%
\pgfpathrectangle{\pgfqpoint{1.500000in}{0.880000in}}{\pgfqpoint{9.300000in}{6.160000in}}%
\pgfusepath{clip}%
\pgfsetroundcap%
\pgfsetroundjoin%
\pgfsetlinewidth{0.803000pt}%
\definecolor{currentstroke}{rgb}{0.800000,0.800000,0.800000}%
\pgfsetstrokecolor{currentstroke}%
\pgfsetdash{}{0pt}%
\pgfpathmoveto{\pgfqpoint{5.093182in}{0.880000in}}%
\pgfpathlineto{\pgfqpoint{5.093182in}{7.040000in}}%
\pgfusepath{stroke}%
\end{pgfscope}%
\begin{pgfscope}%
\definecolor{textcolor}{rgb}{0.150000,0.150000,0.150000}%
\pgfsetstrokecolor{textcolor}%
\pgfsetfillcolor{textcolor}%
\pgftext[x=5.093182in,y=0.782778in,,top]{\color{textcolor}{\sffamily\fontsize{10.000000}{12.000000}\selectfont\catcode`\^=\active\def^{\ifmmode\sp\else\^{}\fi}\catcode`\%=\active\def%{\%}\ensuremath{-}0.25}}%
\end{pgfscope}%
\begin{pgfscope}%
\pgfpathrectangle{\pgfqpoint{1.500000in}{0.880000in}}{\pgfqpoint{9.300000in}{6.160000in}}%
\pgfusepath{clip}%
\pgfsetroundcap%
\pgfsetroundjoin%
\pgfsetlinewidth{0.803000pt}%
\definecolor{currentstroke}{rgb}{0.800000,0.800000,0.800000}%
\pgfsetstrokecolor{currentstroke}%
\pgfsetdash{}{0pt}%
\pgfpathmoveto{\pgfqpoint{6.150000in}{0.880000in}}%
\pgfpathlineto{\pgfqpoint{6.150000in}{7.040000in}}%
\pgfusepath{stroke}%
\end{pgfscope}%
\begin{pgfscope}%
\definecolor{textcolor}{rgb}{0.150000,0.150000,0.150000}%
\pgfsetstrokecolor{textcolor}%
\pgfsetfillcolor{textcolor}%
\pgftext[x=6.150000in,y=0.782778in,,top]{\color{textcolor}{\sffamily\fontsize{10.000000}{12.000000}\selectfont\catcode`\^=\active\def^{\ifmmode\sp\else\^{}\fi}\catcode`\%=\active\def%{\%}0.00}}%
\end{pgfscope}%
\begin{pgfscope}%
\pgfpathrectangle{\pgfqpoint{1.500000in}{0.880000in}}{\pgfqpoint{9.300000in}{6.160000in}}%
\pgfusepath{clip}%
\pgfsetroundcap%
\pgfsetroundjoin%
\pgfsetlinewidth{0.803000pt}%
\definecolor{currentstroke}{rgb}{0.800000,0.800000,0.800000}%
\pgfsetstrokecolor{currentstroke}%
\pgfsetdash{}{0pt}%
\pgfpathmoveto{\pgfqpoint{7.206818in}{0.880000in}}%
\pgfpathlineto{\pgfqpoint{7.206818in}{7.040000in}}%
\pgfusepath{stroke}%
\end{pgfscope}%
\begin{pgfscope}%
\definecolor{textcolor}{rgb}{0.150000,0.150000,0.150000}%
\pgfsetstrokecolor{textcolor}%
\pgfsetfillcolor{textcolor}%
\pgftext[x=7.206818in,y=0.782778in,,top]{\color{textcolor}{\sffamily\fontsize{10.000000}{12.000000}\selectfont\catcode`\^=\active\def^{\ifmmode\sp\else\^{}\fi}\catcode`\%=\active\def%{\%}0.25}}%
\end{pgfscope}%
\begin{pgfscope}%
\pgfpathrectangle{\pgfqpoint{1.500000in}{0.880000in}}{\pgfqpoint{9.300000in}{6.160000in}}%
\pgfusepath{clip}%
\pgfsetroundcap%
\pgfsetroundjoin%
\pgfsetlinewidth{0.803000pt}%
\definecolor{currentstroke}{rgb}{0.800000,0.800000,0.800000}%
\pgfsetstrokecolor{currentstroke}%
\pgfsetdash{}{0pt}%
\pgfpathmoveto{\pgfqpoint{8.263636in}{0.880000in}}%
\pgfpathlineto{\pgfqpoint{8.263636in}{7.040000in}}%
\pgfusepath{stroke}%
\end{pgfscope}%
\begin{pgfscope}%
\definecolor{textcolor}{rgb}{0.150000,0.150000,0.150000}%
\pgfsetstrokecolor{textcolor}%
\pgfsetfillcolor{textcolor}%
\pgftext[x=8.263636in,y=0.782778in,,top]{\color{textcolor}{\sffamily\fontsize{10.000000}{12.000000}\selectfont\catcode`\^=\active\def^{\ifmmode\sp\else\^{}\fi}\catcode`\%=\active\def%{\%}0.50}}%
\end{pgfscope}%
\begin{pgfscope}%
\pgfpathrectangle{\pgfqpoint{1.500000in}{0.880000in}}{\pgfqpoint{9.300000in}{6.160000in}}%
\pgfusepath{clip}%
\pgfsetroundcap%
\pgfsetroundjoin%
\pgfsetlinewidth{0.803000pt}%
\definecolor{currentstroke}{rgb}{0.800000,0.800000,0.800000}%
\pgfsetstrokecolor{currentstroke}%
\pgfsetdash{}{0pt}%
\pgfpathmoveto{\pgfqpoint{9.320455in}{0.880000in}}%
\pgfpathlineto{\pgfqpoint{9.320455in}{7.040000in}}%
\pgfusepath{stroke}%
\end{pgfscope}%
\begin{pgfscope}%
\definecolor{textcolor}{rgb}{0.150000,0.150000,0.150000}%
\pgfsetstrokecolor{textcolor}%
\pgfsetfillcolor{textcolor}%
\pgftext[x=9.320455in,y=0.782778in,,top]{\color{textcolor}{\sffamily\fontsize{10.000000}{12.000000}\selectfont\catcode`\^=\active\def^{\ifmmode\sp\else\^{}\fi}\catcode`\%=\active\def%{\%}0.75}}%
\end{pgfscope}%
\begin{pgfscope}%
\pgfpathrectangle{\pgfqpoint{1.500000in}{0.880000in}}{\pgfqpoint{9.300000in}{6.160000in}}%
\pgfusepath{clip}%
\pgfsetroundcap%
\pgfsetroundjoin%
\pgfsetlinewidth{0.803000pt}%
\definecolor{currentstroke}{rgb}{0.800000,0.800000,0.800000}%
\pgfsetstrokecolor{currentstroke}%
\pgfsetdash{}{0pt}%
\pgfpathmoveto{\pgfqpoint{10.377273in}{0.880000in}}%
\pgfpathlineto{\pgfqpoint{10.377273in}{7.040000in}}%
\pgfusepath{stroke}%
\end{pgfscope}%
\begin{pgfscope}%
\definecolor{textcolor}{rgb}{0.150000,0.150000,0.150000}%
\pgfsetstrokecolor{textcolor}%
\pgfsetfillcolor{textcolor}%
\pgftext[x=10.377273in,y=0.782778in,,top]{\color{textcolor}{\sffamily\fontsize{10.000000}{12.000000}\selectfont\catcode`\^=\active\def^{\ifmmode\sp\else\^{}\fi}\catcode`\%=\active\def%{\%}1.00}}%
\end{pgfscope}%
\begin{pgfscope}%
\definecolor{textcolor}{rgb}{0.150000,0.150000,0.150000}%
\pgfsetstrokecolor{textcolor}%
\pgfsetfillcolor{textcolor}%
\pgftext[x=6.150000in,y=0.600202in,,top]{\color{textcolor}{\sffamily\fontsize{14.000000}{16.800000}\selectfont\catcode`\^=\active\def^{\ifmmode\sp\else\^{}\fi}\catcode`\%=\active\def%{\%}x}}%
\end{pgfscope}%
\begin{pgfscope}%
\pgfpathrectangle{\pgfqpoint{1.500000in}{0.880000in}}{\pgfqpoint{9.300000in}{6.160000in}}%
\pgfusepath{clip}%
\pgfsetroundcap%
\pgfsetroundjoin%
\pgfsetlinewidth{0.803000pt}%
\definecolor{currentstroke}{rgb}{0.800000,0.800000,0.800000}%
\pgfsetstrokecolor{currentstroke}%
\pgfsetdash{}{0pt}%
\pgfpathmoveto{\pgfqpoint{1.500000in}{1.160000in}}%
\pgfpathlineto{\pgfqpoint{10.800000in}{1.160000in}}%
\pgfusepath{stroke}%
\end{pgfscope}%
\begin{pgfscope}%
\definecolor{textcolor}{rgb}{0.150000,0.150000,0.150000}%
\pgfsetstrokecolor{textcolor}%
\pgfsetfillcolor{textcolor}%
\pgftext[x=1.024435in, y=1.110290in, left, base]{\color{textcolor}{\sffamily\fontsize{10.000000}{12.000000}\selectfont\catcode`\^=\active\def^{\ifmmode\sp\else\^{}\fi}\catcode`\%=\active\def%{\%}\ensuremath{-}1.00}}%
\end{pgfscope}%
\begin{pgfscope}%
\pgfpathrectangle{\pgfqpoint{1.500000in}{0.880000in}}{\pgfqpoint{9.300000in}{6.160000in}}%
\pgfusepath{clip}%
\pgfsetroundcap%
\pgfsetroundjoin%
\pgfsetlinewidth{0.803000pt}%
\definecolor{currentstroke}{rgb}{0.800000,0.800000,0.800000}%
\pgfsetstrokecolor{currentstroke}%
\pgfsetdash{}{0pt}%
\pgfpathmoveto{\pgfqpoint{1.500000in}{1.860000in}}%
\pgfpathlineto{\pgfqpoint{10.800000in}{1.860000in}}%
\pgfusepath{stroke}%
\end{pgfscope}%
\begin{pgfscope}%
\definecolor{textcolor}{rgb}{0.150000,0.150000,0.150000}%
\pgfsetstrokecolor{textcolor}%
\pgfsetfillcolor{textcolor}%
\pgftext[x=1.024435in, y=1.810290in, left, base]{\color{textcolor}{\sffamily\fontsize{10.000000}{12.000000}\selectfont\catcode`\^=\active\def^{\ifmmode\sp\else\^{}\fi}\catcode`\%=\active\def%{\%}\ensuremath{-}0.75}}%
\end{pgfscope}%
\begin{pgfscope}%
\pgfpathrectangle{\pgfqpoint{1.500000in}{0.880000in}}{\pgfqpoint{9.300000in}{6.160000in}}%
\pgfusepath{clip}%
\pgfsetroundcap%
\pgfsetroundjoin%
\pgfsetlinewidth{0.803000pt}%
\definecolor{currentstroke}{rgb}{0.800000,0.800000,0.800000}%
\pgfsetstrokecolor{currentstroke}%
\pgfsetdash{}{0pt}%
\pgfpathmoveto{\pgfqpoint{1.500000in}{2.560000in}}%
\pgfpathlineto{\pgfqpoint{10.800000in}{2.560000in}}%
\pgfusepath{stroke}%
\end{pgfscope}%
\begin{pgfscope}%
\definecolor{textcolor}{rgb}{0.150000,0.150000,0.150000}%
\pgfsetstrokecolor{textcolor}%
\pgfsetfillcolor{textcolor}%
\pgftext[x=1.024435in, y=2.510290in, left, base]{\color{textcolor}{\sffamily\fontsize{10.000000}{12.000000}\selectfont\catcode`\^=\active\def^{\ifmmode\sp\else\^{}\fi}\catcode`\%=\active\def%{\%}\ensuremath{-}0.50}}%
\end{pgfscope}%
\begin{pgfscope}%
\pgfpathrectangle{\pgfqpoint{1.500000in}{0.880000in}}{\pgfqpoint{9.300000in}{6.160000in}}%
\pgfusepath{clip}%
\pgfsetroundcap%
\pgfsetroundjoin%
\pgfsetlinewidth{0.803000pt}%
\definecolor{currentstroke}{rgb}{0.800000,0.800000,0.800000}%
\pgfsetstrokecolor{currentstroke}%
\pgfsetdash{}{0pt}%
\pgfpathmoveto{\pgfqpoint{1.500000in}{3.260000in}}%
\pgfpathlineto{\pgfqpoint{10.800000in}{3.260000in}}%
\pgfusepath{stroke}%
\end{pgfscope}%
\begin{pgfscope}%
\definecolor{textcolor}{rgb}{0.150000,0.150000,0.150000}%
\pgfsetstrokecolor{textcolor}%
\pgfsetfillcolor{textcolor}%
\pgftext[x=1.024435in, y=3.210290in, left, base]{\color{textcolor}{\sffamily\fontsize{10.000000}{12.000000}\selectfont\catcode`\^=\active\def^{\ifmmode\sp\else\^{}\fi}\catcode`\%=\active\def%{\%}\ensuremath{-}0.25}}%
\end{pgfscope}%
\begin{pgfscope}%
\pgfpathrectangle{\pgfqpoint{1.500000in}{0.880000in}}{\pgfqpoint{9.300000in}{6.160000in}}%
\pgfusepath{clip}%
\pgfsetroundcap%
\pgfsetroundjoin%
\pgfsetlinewidth{0.803000pt}%
\definecolor{currentstroke}{rgb}{0.800000,0.800000,0.800000}%
\pgfsetstrokecolor{currentstroke}%
\pgfsetdash{}{0pt}%
\pgfpathmoveto{\pgfqpoint{1.500000in}{3.960000in}}%
\pgfpathlineto{\pgfqpoint{10.800000in}{3.960000in}}%
\pgfusepath{stroke}%
\end{pgfscope}%
\begin{pgfscope}%
\definecolor{textcolor}{rgb}{0.150000,0.150000,0.150000}%
\pgfsetstrokecolor{textcolor}%
\pgfsetfillcolor{textcolor}%
\pgftext[x=1.132460in, y=3.910290in, left, base]{\color{textcolor}{\sffamily\fontsize{10.000000}{12.000000}\selectfont\catcode`\^=\active\def^{\ifmmode\sp\else\^{}\fi}\catcode`\%=\active\def%{\%}0.00}}%
\end{pgfscope}%
\begin{pgfscope}%
\pgfpathrectangle{\pgfqpoint{1.500000in}{0.880000in}}{\pgfqpoint{9.300000in}{6.160000in}}%
\pgfusepath{clip}%
\pgfsetroundcap%
\pgfsetroundjoin%
\pgfsetlinewidth{0.803000pt}%
\definecolor{currentstroke}{rgb}{0.800000,0.800000,0.800000}%
\pgfsetstrokecolor{currentstroke}%
\pgfsetdash{}{0pt}%
\pgfpathmoveto{\pgfqpoint{1.500000in}{4.660000in}}%
\pgfpathlineto{\pgfqpoint{10.800000in}{4.660000in}}%
\pgfusepath{stroke}%
\end{pgfscope}%
\begin{pgfscope}%
\definecolor{textcolor}{rgb}{0.150000,0.150000,0.150000}%
\pgfsetstrokecolor{textcolor}%
\pgfsetfillcolor{textcolor}%
\pgftext[x=1.132460in, y=4.610290in, left, base]{\color{textcolor}{\sffamily\fontsize{10.000000}{12.000000}\selectfont\catcode`\^=\active\def^{\ifmmode\sp\else\^{}\fi}\catcode`\%=\active\def%{\%}0.25}}%
\end{pgfscope}%
\begin{pgfscope}%
\pgfpathrectangle{\pgfqpoint{1.500000in}{0.880000in}}{\pgfqpoint{9.300000in}{6.160000in}}%
\pgfusepath{clip}%
\pgfsetroundcap%
\pgfsetroundjoin%
\pgfsetlinewidth{0.803000pt}%
\definecolor{currentstroke}{rgb}{0.800000,0.800000,0.800000}%
\pgfsetstrokecolor{currentstroke}%
\pgfsetdash{}{0pt}%
\pgfpathmoveto{\pgfqpoint{1.500000in}{5.360000in}}%
\pgfpathlineto{\pgfqpoint{10.800000in}{5.360000in}}%
\pgfusepath{stroke}%
\end{pgfscope}%
\begin{pgfscope}%
\definecolor{textcolor}{rgb}{0.150000,0.150000,0.150000}%
\pgfsetstrokecolor{textcolor}%
\pgfsetfillcolor{textcolor}%
\pgftext[x=1.132460in, y=5.310290in, left, base]{\color{textcolor}{\sffamily\fontsize{10.000000}{12.000000}\selectfont\catcode`\^=\active\def^{\ifmmode\sp\else\^{}\fi}\catcode`\%=\active\def%{\%}0.50}}%
\end{pgfscope}%
\begin{pgfscope}%
\pgfpathrectangle{\pgfqpoint{1.500000in}{0.880000in}}{\pgfqpoint{9.300000in}{6.160000in}}%
\pgfusepath{clip}%
\pgfsetroundcap%
\pgfsetroundjoin%
\pgfsetlinewidth{0.803000pt}%
\definecolor{currentstroke}{rgb}{0.800000,0.800000,0.800000}%
\pgfsetstrokecolor{currentstroke}%
\pgfsetdash{}{0pt}%
\pgfpathmoveto{\pgfqpoint{1.500000in}{6.060000in}}%
\pgfpathlineto{\pgfqpoint{10.800000in}{6.060000in}}%
\pgfusepath{stroke}%
\end{pgfscope}%
\begin{pgfscope}%
\definecolor{textcolor}{rgb}{0.150000,0.150000,0.150000}%
\pgfsetstrokecolor{textcolor}%
\pgfsetfillcolor{textcolor}%
\pgftext[x=1.132460in, y=6.010290in, left, base]{\color{textcolor}{\sffamily\fontsize{10.000000}{12.000000}\selectfont\catcode`\^=\active\def^{\ifmmode\sp\else\^{}\fi}\catcode`\%=\active\def%{\%}0.75}}%
\end{pgfscope}%
\begin{pgfscope}%
\pgfpathrectangle{\pgfqpoint{1.500000in}{0.880000in}}{\pgfqpoint{9.300000in}{6.160000in}}%
\pgfusepath{clip}%
\pgfsetroundcap%
\pgfsetroundjoin%
\pgfsetlinewidth{0.803000pt}%
\definecolor{currentstroke}{rgb}{0.800000,0.800000,0.800000}%
\pgfsetstrokecolor{currentstroke}%
\pgfsetdash{}{0pt}%
\pgfpathmoveto{\pgfqpoint{1.500000in}{6.760000in}}%
\pgfpathlineto{\pgfqpoint{10.800000in}{6.760000in}}%
\pgfusepath{stroke}%
\end{pgfscope}%
\begin{pgfscope}%
\definecolor{textcolor}{rgb}{0.150000,0.150000,0.150000}%
\pgfsetstrokecolor{textcolor}%
\pgfsetfillcolor{textcolor}%
\pgftext[x=1.132460in, y=6.710290in, left, base]{\color{textcolor}{\sffamily\fontsize{10.000000}{12.000000}\selectfont\catcode`\^=\active\def^{\ifmmode\sp\else\^{}\fi}\catcode`\%=\active\def%{\%}1.00}}%
\end{pgfscope}%
\begin{pgfscope}%
\definecolor{textcolor}{rgb}{0.150000,0.150000,0.150000}%
\pgfsetstrokecolor{textcolor}%
\pgfsetfillcolor{textcolor}%
\pgftext[x=0.968879in,y=3.960000in,,bottom,rotate=90.000000]{\color{textcolor}{\sffamily\fontsize{14.000000}{16.800000}\selectfont\catcode`\^=\active\def^{\ifmmode\sp\else\^{}\fi}\catcode`\%=\active\def%{\%}$P_l(x)$}}%
\end{pgfscope}%
\begin{pgfscope}%
\pgfpathrectangle{\pgfqpoint{1.500000in}{0.880000in}}{\pgfqpoint{9.300000in}{6.160000in}}%
\pgfusepath{clip}%
\pgfsetroundcap%
\pgfsetroundjoin%
\pgfsetlinewidth{2.007500pt}%
\definecolor{currentstroke}{rgb}{0.000000,0.000000,1.000000}%
\pgfsetstrokecolor{currentstroke}%
\pgfsetdash{}{0pt}%
\pgfpathmoveto{\pgfqpoint{1.922727in}{6.760000in}}%
\pgfpathlineto{\pgfqpoint{10.377273in}{6.760000in}}%
\pgfpathlineto{\pgfqpoint{10.377273in}{6.760000in}}%
\pgfusepath{stroke}%
\end{pgfscope}%
\begin{pgfscope}%
\pgfpathrectangle{\pgfqpoint{1.500000in}{0.880000in}}{\pgfqpoint{9.300000in}{6.160000in}}%
\pgfusepath{clip}%
\pgfsetroundcap%
\pgfsetroundjoin%
\pgfsetlinewidth{2.007500pt}%
\definecolor{currentstroke}{rgb}{1.000000,0.000000,0.000000}%
\pgfsetstrokecolor{currentstroke}%
\pgfsetdash{}{0pt}%
\pgfpathmoveto{\pgfqpoint{1.922727in}{1.160000in}}%
\pgfpathlineto{\pgfqpoint{10.377273in}{6.760000in}}%
\pgfpathlineto{\pgfqpoint{10.377273in}{6.760000in}}%
\pgfusepath{stroke}%
\end{pgfscope}%
\begin{pgfscope}%
\pgfpathrectangle{\pgfqpoint{1.500000in}{0.880000in}}{\pgfqpoint{9.300000in}{6.160000in}}%
\pgfusepath{clip}%
\pgfsetroundcap%
\pgfsetroundjoin%
\pgfsetlinewidth{2.007500pt}%
\definecolor{currentstroke}{rgb}{0.000000,0.501961,0.000000}%
\pgfsetstrokecolor{currentstroke}%
\pgfsetdash{}{0pt}%
\pgfpathmoveto{\pgfqpoint{1.922727in}{6.760000in}}%
\pgfpathlineto{\pgfqpoint{1.988908in}{6.629523in}}%
\pgfpathlineto{\pgfqpoint{2.055088in}{6.501104in}}%
\pgfpathlineto{\pgfqpoint{2.121268in}{6.374744in}}%
\pgfpathlineto{\pgfqpoint{2.187449in}{6.250443in}}%
\pgfpathlineto{\pgfqpoint{2.253629in}{6.128201in}}%
\pgfpathlineto{\pgfqpoint{2.319810in}{6.008018in}}%
\pgfpathlineto{\pgfqpoint{2.385990in}{5.889893in}}%
\pgfpathlineto{\pgfqpoint{2.452170in}{5.773827in}}%
\pgfpathlineto{\pgfqpoint{2.518351in}{5.659820in}}%
\pgfpathlineto{\pgfqpoint{2.584531in}{5.547872in}}%
\pgfpathlineto{\pgfqpoint{2.650712in}{5.437983in}}%
\pgfpathlineto{\pgfqpoint{2.716892in}{5.330153in}}%
\pgfpathlineto{\pgfqpoint{2.783072in}{5.224381in}}%
\pgfpathlineto{\pgfqpoint{2.849253in}{5.120668in}}%
\pgfpathlineto{\pgfqpoint{2.915433in}{5.019014in}}%
\pgfpathlineto{\pgfqpoint{2.981614in}{4.919419in}}%
\pgfpathlineto{\pgfqpoint{3.047794in}{4.821882in}}%
\pgfpathlineto{\pgfqpoint{3.113974in}{4.726405in}}%
\pgfpathlineto{\pgfqpoint{3.180155in}{4.632986in}}%
\pgfpathlineto{\pgfqpoint{3.246335in}{4.541626in}}%
\pgfpathlineto{\pgfqpoint{3.312516in}{4.452325in}}%
\pgfpathlineto{\pgfqpoint{3.378696in}{4.365083in}}%
\pgfpathlineto{\pgfqpoint{3.444876in}{4.279899in}}%
\pgfpathlineto{\pgfqpoint{3.511057in}{4.196775in}}%
\pgfpathlineto{\pgfqpoint{3.577237in}{4.115709in}}%
\pgfpathlineto{\pgfqpoint{3.643418in}{4.036702in}}%
\pgfpathlineto{\pgfqpoint{3.709598in}{3.959753in}}%
\pgfpathlineto{\pgfqpoint{3.775778in}{3.884864in}}%
\pgfpathlineto{\pgfqpoint{3.841959in}{3.812033in}}%
\pgfpathlineto{\pgfqpoint{3.908139in}{3.741262in}}%
\pgfpathlineto{\pgfqpoint{3.974320in}{3.672549in}}%
\pgfpathlineto{\pgfqpoint{4.040500in}{3.605894in}}%
\pgfpathlineto{\pgfqpoint{4.106680in}{3.541299in}}%
\pgfpathlineto{\pgfqpoint{4.156316in}{3.494204in}}%
\pgfpathlineto{\pgfqpoint{4.205951in}{3.448266in}}%
\pgfpathlineto{\pgfqpoint{4.255586in}{3.403487in}}%
\pgfpathlineto{\pgfqpoint{4.305221in}{3.359866in}}%
\pgfpathlineto{\pgfqpoint{4.354857in}{3.317403in}}%
\pgfpathlineto{\pgfqpoint{4.404492in}{3.276098in}}%
\pgfpathlineto{\pgfqpoint{4.454127in}{3.235951in}}%
\pgfpathlineto{\pgfqpoint{4.503763in}{3.196962in}}%
\pgfpathlineto{\pgfqpoint{4.553398in}{3.159131in}}%
\pgfpathlineto{\pgfqpoint{4.603033in}{3.122459in}}%
\pgfpathlineto{\pgfqpoint{4.652669in}{3.086944in}}%
\pgfpathlineto{\pgfqpoint{4.702304in}{3.052588in}}%
\pgfpathlineto{\pgfqpoint{4.751939in}{3.019389in}}%
\pgfpathlineto{\pgfqpoint{4.801574in}{2.987349in}}%
\pgfpathlineto{\pgfqpoint{4.851210in}{2.956467in}}%
\pgfpathlineto{\pgfqpoint{4.900845in}{2.926743in}}%
\pgfpathlineto{\pgfqpoint{4.950480in}{2.898177in}}%
\pgfpathlineto{\pgfqpoint{5.000116in}{2.870769in}}%
\pgfpathlineto{\pgfqpoint{5.049751in}{2.844519in}}%
\pgfpathlineto{\pgfqpoint{5.099386in}{2.819427in}}%
\pgfpathlineto{\pgfqpoint{5.149022in}{2.795493in}}%
\pgfpathlineto{\pgfqpoint{5.198657in}{2.772717in}}%
\pgfpathlineto{\pgfqpoint{5.248292in}{2.751100in}}%
\pgfpathlineto{\pgfqpoint{5.297927in}{2.730640in}}%
\pgfpathlineto{\pgfqpoint{5.347563in}{2.711339in}}%
\pgfpathlineto{\pgfqpoint{5.397198in}{2.693196in}}%
\pgfpathlineto{\pgfqpoint{5.446833in}{2.676210in}}%
\pgfpathlineto{\pgfqpoint{5.496469in}{2.660383in}}%
\pgfpathlineto{\pgfqpoint{5.546104in}{2.645714in}}%
\pgfpathlineto{\pgfqpoint{5.595739in}{2.632203in}}%
\pgfpathlineto{\pgfqpoint{5.645374in}{2.619850in}}%
\pgfpathlineto{\pgfqpoint{5.695010in}{2.608656in}}%
\pgfpathlineto{\pgfqpoint{5.744645in}{2.598619in}}%
\pgfpathlineto{\pgfqpoint{5.794280in}{2.589740in}}%
\pgfpathlineto{\pgfqpoint{5.843916in}{2.582020in}}%
\pgfpathlineto{\pgfqpoint{5.893551in}{2.575457in}}%
\pgfpathlineto{\pgfqpoint{5.943186in}{2.570053in}}%
\pgfpathlineto{\pgfqpoint{5.992822in}{2.565807in}}%
\pgfpathlineto{\pgfqpoint{6.042457in}{2.562718in}}%
\pgfpathlineto{\pgfqpoint{6.092092in}{2.560788in}}%
\pgfpathlineto{\pgfqpoint{6.141727in}{2.560016in}}%
\pgfpathlineto{\pgfqpoint{6.191363in}{2.560402in}}%
\pgfpathlineto{\pgfqpoint{6.240998in}{2.561946in}}%
\pgfpathlineto{\pgfqpoint{6.290633in}{2.564648in}}%
\pgfpathlineto{\pgfqpoint{6.340269in}{2.568509in}}%
\pgfpathlineto{\pgfqpoint{6.389904in}{2.573527in}}%
\pgfpathlineto{\pgfqpoint{6.439539in}{2.579704in}}%
\pgfpathlineto{\pgfqpoint{6.489175in}{2.587038in}}%
\pgfpathlineto{\pgfqpoint{6.538810in}{2.595531in}}%
\pgfpathlineto{\pgfqpoint{6.588445in}{2.605181in}}%
\pgfpathlineto{\pgfqpoint{6.638080in}{2.615990in}}%
\pgfpathlineto{\pgfqpoint{6.687716in}{2.627957in}}%
\pgfpathlineto{\pgfqpoint{6.737351in}{2.641082in}}%
\pgfpathlineto{\pgfqpoint{6.786986in}{2.655365in}}%
\pgfpathlineto{\pgfqpoint{6.836622in}{2.670806in}}%
\pgfpathlineto{\pgfqpoint{6.886257in}{2.687405in}}%
\pgfpathlineto{\pgfqpoint{6.935892in}{2.705163in}}%
\pgfpathlineto{\pgfqpoint{6.985527in}{2.724078in}}%
\pgfpathlineto{\pgfqpoint{7.035163in}{2.744151in}}%
\pgfpathlineto{\pgfqpoint{7.084798in}{2.765383in}}%
\pgfpathlineto{\pgfqpoint{7.134433in}{2.787773in}}%
\pgfpathlineto{\pgfqpoint{7.184069in}{2.811320in}}%
\pgfpathlineto{\pgfqpoint{7.233704in}{2.836026in}}%
\pgfpathlineto{\pgfqpoint{7.283339in}{2.861890in}}%
\pgfpathlineto{\pgfqpoint{7.332975in}{2.888912in}}%
\pgfpathlineto{\pgfqpoint{7.382610in}{2.917092in}}%
\pgfpathlineto{\pgfqpoint{7.432245in}{2.946430in}}%
\pgfpathlineto{\pgfqpoint{7.481880in}{2.976926in}}%
\pgfpathlineto{\pgfqpoint{7.531516in}{3.008581in}}%
\pgfpathlineto{\pgfqpoint{7.581151in}{3.041393in}}%
\pgfpathlineto{\pgfqpoint{7.630786in}{3.075363in}}%
\pgfpathlineto{\pgfqpoint{7.680422in}{3.110492in}}%
\pgfpathlineto{\pgfqpoint{7.730057in}{3.146779in}}%
\pgfpathlineto{\pgfqpoint{7.779692in}{3.184223in}}%
\pgfpathlineto{\pgfqpoint{7.829328in}{3.222826in}}%
\pgfpathlineto{\pgfqpoint{7.878963in}{3.262587in}}%
\pgfpathlineto{\pgfqpoint{7.928598in}{3.303506in}}%
\pgfpathlineto{\pgfqpoint{7.978233in}{3.345583in}}%
\pgfpathlineto{\pgfqpoint{8.027869in}{3.388818in}}%
\pgfpathlineto{\pgfqpoint{8.077504in}{3.433211in}}%
\pgfpathlineto{\pgfqpoint{8.127139in}{3.478763in}}%
\pgfpathlineto{\pgfqpoint{8.176775in}{3.525472in}}%
\pgfpathlineto{\pgfqpoint{8.226410in}{3.573339in}}%
\pgfpathlineto{\pgfqpoint{8.292590in}{3.638964in}}%
\pgfpathlineto{\pgfqpoint{8.358771in}{3.706648in}}%
\pgfpathlineto{\pgfqpoint{8.424951in}{3.776390in}}%
\pgfpathlineto{\pgfqpoint{8.491131in}{3.848191in}}%
\pgfpathlineto{\pgfqpoint{8.557312in}{3.922051in}}%
\pgfpathlineto{\pgfqpoint{8.623492in}{3.997970in}}%
\pgfpathlineto{\pgfqpoint{8.689673in}{4.075948in}}%
\pgfpathlineto{\pgfqpoint{8.755853in}{4.155984in}}%
\pgfpathlineto{\pgfqpoint{8.822033in}{4.238080in}}%
\pgfpathlineto{\pgfqpoint{8.888214in}{4.322234in}}%
\pgfpathlineto{\pgfqpoint{8.954394in}{4.408447in}}%
\pgfpathlineto{\pgfqpoint{9.020575in}{4.496718in}}%
\pgfpathlineto{\pgfqpoint{9.086755in}{4.587049in}}%
\pgfpathlineto{\pgfqpoint{9.152935in}{4.679438in}}%
\pgfpathlineto{\pgfqpoint{9.219116in}{4.773886in}}%
\pgfpathlineto{\pgfqpoint{9.285296in}{4.870393in}}%
\pgfpathlineto{\pgfqpoint{9.351477in}{4.968959in}}%
\pgfpathlineto{\pgfqpoint{9.417657in}{5.069584in}}%
\pgfpathlineto{\pgfqpoint{9.483837in}{5.172267in}}%
\pgfpathlineto{\pgfqpoint{9.550018in}{5.277009in}}%
\pgfpathlineto{\pgfqpoint{9.616198in}{5.383810in}}%
\pgfpathlineto{\pgfqpoint{9.682379in}{5.492670in}}%
\pgfpathlineto{\pgfqpoint{9.748559in}{5.603589in}}%
\pgfpathlineto{\pgfqpoint{9.814739in}{5.716566in}}%
\pgfpathlineto{\pgfqpoint{9.880920in}{5.831603in}}%
\pgfpathlineto{\pgfqpoint{9.947100in}{5.948698in}}%
\pgfpathlineto{\pgfqpoint{10.013281in}{6.067852in}}%
\pgfpathlineto{\pgfqpoint{10.079461in}{6.189065in}}%
\pgfpathlineto{\pgfqpoint{10.145641in}{6.312336in}}%
\pgfpathlineto{\pgfqpoint{10.211822in}{6.437667in}}%
\pgfpathlineto{\pgfqpoint{10.278002in}{6.565056in}}%
\pgfpathlineto{\pgfqpoint{10.344183in}{6.694504in}}%
\pgfpathlineto{\pgfqpoint{10.377273in}{6.760000in}}%
\pgfpathlineto{\pgfqpoint{10.377273in}{6.760000in}}%
\pgfusepath{stroke}%
\end{pgfscope}%
\begin{pgfscope}%
\pgfpathrectangle{\pgfqpoint{1.500000in}{0.880000in}}{\pgfqpoint{9.300000in}{6.160000in}}%
\pgfusepath{clip}%
\pgfsetroundcap%
\pgfsetroundjoin%
\pgfsetlinewidth{2.007500pt}%
\definecolor{currentstroke}{rgb}{0.000000,0.000000,0.000000}%
\pgfsetstrokecolor{currentstroke}%
\pgfsetdash{}{0pt}%
\pgfpathmoveto{\pgfqpoint{1.922727in}{1.160000in}}%
\pgfpathlineto{\pgfqpoint{1.972363in}{1.354376in}}%
\pgfpathlineto{\pgfqpoint{2.021998in}{1.543030in}}%
\pgfpathlineto{\pgfqpoint{2.071633in}{1.726030in}}%
\pgfpathlineto{\pgfqpoint{2.121268in}{1.903443in}}%
\pgfpathlineto{\pgfqpoint{2.170904in}{2.075338in}}%
\pgfpathlineto{\pgfqpoint{2.220539in}{2.241782in}}%
\pgfpathlineto{\pgfqpoint{2.270174in}{2.402843in}}%
\pgfpathlineto{\pgfqpoint{2.319810in}{2.558591in}}%
\pgfpathlineto{\pgfqpoint{2.369445in}{2.709091in}}%
\pgfpathlineto{\pgfqpoint{2.419080in}{2.854413in}}%
\pgfpathlineto{\pgfqpoint{2.452170in}{2.948451in}}%
\pgfpathlineto{\pgfqpoint{2.485261in}{3.040238in}}%
\pgfpathlineto{\pgfqpoint{2.518351in}{3.129794in}}%
\pgfpathlineto{\pgfqpoint{2.551441in}{3.217139in}}%
\pgfpathlineto{\pgfqpoint{2.584531in}{3.302293in}}%
\pgfpathlineto{\pgfqpoint{2.617621in}{3.385277in}}%
\pgfpathlineto{\pgfqpoint{2.650712in}{3.466110in}}%
\pgfpathlineto{\pgfqpoint{2.683802in}{3.544812in}}%
\pgfpathlineto{\pgfqpoint{2.716892in}{3.621405in}}%
\pgfpathlineto{\pgfqpoint{2.749982in}{3.695907in}}%
\pgfpathlineto{\pgfqpoint{2.783072in}{3.768340in}}%
\pgfpathlineto{\pgfqpoint{2.816163in}{3.838723in}}%
\pgfpathlineto{\pgfqpoint{2.849253in}{3.907076in}}%
\pgfpathlineto{\pgfqpoint{2.882343in}{3.973420in}}%
\pgfpathlineto{\pgfqpoint{2.915433in}{4.037774in}}%
\pgfpathlineto{\pgfqpoint{2.948523in}{4.100159in}}%
\pgfpathlineto{\pgfqpoint{2.981614in}{4.160596in}}%
\pgfpathlineto{\pgfqpoint{3.014704in}{4.219103in}}%
\pgfpathlineto{\pgfqpoint{3.047794in}{4.275702in}}%
\pgfpathlineto{\pgfqpoint{3.080884in}{4.330412in}}%
\pgfpathlineto{\pgfqpoint{3.113974in}{4.383253in}}%
\pgfpathlineto{\pgfqpoint{3.147065in}{4.434246in}}%
\pgfpathlineto{\pgfqpoint{3.180155in}{4.483412in}}%
\pgfpathlineto{\pgfqpoint{3.213245in}{4.530769in}}%
\pgfpathlineto{\pgfqpoint{3.246335in}{4.576338in}}%
\pgfpathlineto{\pgfqpoint{3.279425in}{4.620140in}}%
\pgfpathlineto{\pgfqpoint{3.312516in}{4.662193in}}%
\pgfpathlineto{\pgfqpoint{3.345606in}{4.702520in}}%
\pgfpathlineto{\pgfqpoint{3.378696in}{4.741139in}}%
\pgfpathlineto{\pgfqpoint{3.411786in}{4.778071in}}%
\pgfpathlineto{\pgfqpoint{3.444876in}{4.813336in}}%
\pgfpathlineto{\pgfqpoint{3.477967in}{4.846955in}}%
\pgfpathlineto{\pgfqpoint{3.511057in}{4.878946in}}%
\pgfpathlineto{\pgfqpoint{3.544147in}{4.909331in}}%
\pgfpathlineto{\pgfqpoint{3.577237in}{4.938130in}}%
\pgfpathlineto{\pgfqpoint{3.610327in}{4.965362in}}%
\pgfpathlineto{\pgfqpoint{3.643418in}{4.991048in}}%
\pgfpathlineto{\pgfqpoint{3.676508in}{5.015209in}}%
\pgfpathlineto{\pgfqpoint{3.709598in}{5.037863in}}%
\pgfpathlineto{\pgfqpoint{3.742688in}{5.059032in}}%
\pgfpathlineto{\pgfqpoint{3.775778in}{5.078735in}}%
\pgfpathlineto{\pgfqpoint{3.808869in}{5.096992in}}%
\pgfpathlineto{\pgfqpoint{3.841959in}{5.113825in}}%
\pgfpathlineto{\pgfqpoint{3.875049in}{5.129252in}}%
\pgfpathlineto{\pgfqpoint{3.908139in}{5.143295in}}%
\pgfpathlineto{\pgfqpoint{3.941229in}{5.155972in}}%
\pgfpathlineto{\pgfqpoint{3.974320in}{5.167305in}}%
\pgfpathlineto{\pgfqpoint{4.007410in}{5.177313in}}%
\pgfpathlineto{\pgfqpoint{4.040500in}{5.186017in}}%
\pgfpathlineto{\pgfqpoint{4.073590in}{5.193437in}}%
\pgfpathlineto{\pgfqpoint{4.106680in}{5.199593in}}%
\pgfpathlineto{\pgfqpoint{4.139771in}{5.204504in}}%
\pgfpathlineto{\pgfqpoint{4.172861in}{5.208192in}}%
\pgfpathlineto{\pgfqpoint{4.205951in}{5.210676in}}%
\pgfpathlineto{\pgfqpoint{4.239041in}{5.211977in}}%
\pgfpathlineto{\pgfqpoint{4.272131in}{5.212114in}}%
\pgfpathlineto{\pgfqpoint{4.305221in}{5.211109in}}%
\pgfpathlineto{\pgfqpoint{4.338312in}{5.208980in}}%
\pgfpathlineto{\pgfqpoint{4.371402in}{5.205748in}}%
\pgfpathlineto{\pgfqpoint{4.404492in}{5.201433in}}%
\pgfpathlineto{\pgfqpoint{4.437582in}{5.196056in}}%
\pgfpathlineto{\pgfqpoint{4.470672in}{5.189636in}}%
\pgfpathlineto{\pgfqpoint{4.503763in}{5.182194in}}%
\pgfpathlineto{\pgfqpoint{4.536853in}{5.173749in}}%
\pgfpathlineto{\pgfqpoint{4.569943in}{5.164323in}}%
\pgfpathlineto{\pgfqpoint{4.603033in}{5.153934in}}%
\pgfpathlineto{\pgfqpoint{4.636123in}{5.142604in}}%
\pgfpathlineto{\pgfqpoint{4.669214in}{5.130352in}}%
\pgfpathlineto{\pgfqpoint{4.702304in}{5.117199in}}%
\pgfpathlineto{\pgfqpoint{4.735394in}{5.103164in}}%
\pgfpathlineto{\pgfqpoint{4.768484in}{5.088268in}}%
\pgfpathlineto{\pgfqpoint{4.801574in}{5.072531in}}%
\pgfpathlineto{\pgfqpoint{4.851210in}{5.047393in}}%
\pgfpathlineto{\pgfqpoint{4.900845in}{5.020476in}}%
\pgfpathlineto{\pgfqpoint{4.950480in}{4.991847in}}%
\pgfpathlineto{\pgfqpoint{5.000116in}{4.961576in}}%
\pgfpathlineto{\pgfqpoint{5.049751in}{4.929729in}}%
\pgfpathlineto{\pgfqpoint{5.099386in}{4.896376in}}%
\pgfpathlineto{\pgfqpoint{5.149022in}{4.861583in}}%
\pgfpathlineto{\pgfqpoint{5.198657in}{4.825419in}}%
\pgfpathlineto{\pgfqpoint{5.248292in}{4.787952in}}%
\pgfpathlineto{\pgfqpoint{5.297927in}{4.749250in}}%
\pgfpathlineto{\pgfqpoint{5.347563in}{4.709381in}}%
\pgfpathlineto{\pgfqpoint{5.413743in}{4.654524in}}%
\pgfpathlineto{\pgfqpoint{5.479924in}{4.597874in}}%
\pgfpathlineto{\pgfqpoint{5.546104in}{4.539592in}}%
\pgfpathlineto{\pgfqpoint{5.612284in}{4.479840in}}%
\pgfpathlineto{\pgfqpoint{5.678465in}{4.418778in}}%
\pgfpathlineto{\pgfqpoint{5.761190in}{4.340855in}}%
\pgfpathlineto{\pgfqpoint{5.843916in}{4.261452in}}%
\pgfpathlineto{\pgfqpoint{5.943186in}{4.164660in}}%
\pgfpathlineto{\pgfqpoint{6.075547in}{4.033934in}}%
\pgfpathlineto{\pgfqpoint{6.373359in}{3.739115in}}%
\pgfpathlineto{\pgfqpoint{6.472629in}{3.642564in}}%
\pgfpathlineto{\pgfqpoint{6.555355in}{3.563432in}}%
\pgfpathlineto{\pgfqpoint{6.638080in}{3.485843in}}%
\pgfpathlineto{\pgfqpoint{6.704261in}{3.425093in}}%
\pgfpathlineto{\pgfqpoint{6.770441in}{3.365694in}}%
\pgfpathlineto{\pgfqpoint{6.836622in}{3.307805in}}%
\pgfpathlineto{\pgfqpoint{6.902802in}{3.251588in}}%
\pgfpathlineto{\pgfqpoint{6.968982in}{3.197204in}}%
\pgfpathlineto{\pgfqpoint{7.018618in}{3.157716in}}%
\pgfpathlineto{\pgfqpoint{7.068253in}{3.119418in}}%
\pgfpathlineto{\pgfqpoint{7.117888in}{3.082378in}}%
\pgfpathlineto{\pgfqpoint{7.167524in}{3.046664in}}%
\pgfpathlineto{\pgfqpoint{7.217159in}{3.012343in}}%
\pgfpathlineto{\pgfqpoint{7.266794in}{2.979484in}}%
\pgfpathlineto{\pgfqpoint{7.316429in}{2.948155in}}%
\pgfpathlineto{\pgfqpoint{7.366065in}{2.918424in}}%
\pgfpathlineto{\pgfqpoint{7.415700in}{2.890358in}}%
\pgfpathlineto{\pgfqpoint{7.465335in}{2.864026in}}%
\pgfpathlineto{\pgfqpoint{7.514971in}{2.839496in}}%
\pgfpathlineto{\pgfqpoint{7.548061in}{2.824177in}}%
\pgfpathlineto{\pgfqpoint{7.581151in}{2.809710in}}%
\pgfpathlineto{\pgfqpoint{7.614241in}{2.796113in}}%
\pgfpathlineto{\pgfqpoint{7.647331in}{2.783408in}}%
\pgfpathlineto{\pgfqpoint{7.680422in}{2.771614in}}%
\pgfpathlineto{\pgfqpoint{7.713512in}{2.760753in}}%
\pgfpathlineto{\pgfqpoint{7.746602in}{2.750843in}}%
\pgfpathlineto{\pgfqpoint{7.779692in}{2.741905in}}%
\pgfpathlineto{\pgfqpoint{7.812782in}{2.733959in}}%
\pgfpathlineto{\pgfqpoint{7.845873in}{2.727025in}}%
\pgfpathlineto{\pgfqpoint{7.878963in}{2.721124in}}%
\pgfpathlineto{\pgfqpoint{7.912053in}{2.716276in}}%
\pgfpathlineto{\pgfqpoint{7.945143in}{2.712500in}}%
\pgfpathlineto{\pgfqpoint{7.978233in}{2.709817in}}%
\pgfpathlineto{\pgfqpoint{8.011324in}{2.708247in}}%
\pgfpathlineto{\pgfqpoint{8.044414in}{2.707810in}}%
\pgfpathlineto{\pgfqpoint{8.077504in}{2.708527in}}%
\pgfpathlineto{\pgfqpoint{8.110594in}{2.710416in}}%
\pgfpathlineto{\pgfqpoint{8.143684in}{2.713500in}}%
\pgfpathlineto{\pgfqpoint{8.176775in}{2.717797in}}%
\pgfpathlineto{\pgfqpoint{8.209865in}{2.723328in}}%
\pgfpathlineto{\pgfqpoint{8.242955in}{2.730114in}}%
\pgfpathlineto{\pgfqpoint{8.276045in}{2.738173in}}%
\pgfpathlineto{\pgfqpoint{8.309135in}{2.747526in}}%
\pgfpathlineto{\pgfqpoint{8.342226in}{2.758195in}}%
\pgfpathlineto{\pgfqpoint{8.375316in}{2.770197in}}%
\pgfpathlineto{\pgfqpoint{8.408406in}{2.783555in}}%
\pgfpathlineto{\pgfqpoint{8.441496in}{2.798287in}}%
\pgfpathlineto{\pgfqpoint{8.474586in}{2.814414in}}%
\pgfpathlineto{\pgfqpoint{8.507677in}{2.831957in}}%
\pgfpathlineto{\pgfqpoint{8.540767in}{2.850935in}}%
\pgfpathlineto{\pgfqpoint{8.573857in}{2.871368in}}%
\pgfpathlineto{\pgfqpoint{8.606947in}{2.893277in}}%
\pgfpathlineto{\pgfqpoint{8.640037in}{2.916682in}}%
\pgfpathlineto{\pgfqpoint{8.673128in}{2.941603in}}%
\pgfpathlineto{\pgfqpoint{8.706218in}{2.968059in}}%
\pgfpathlineto{\pgfqpoint{8.739308in}{2.996072in}}%
\pgfpathlineto{\pgfqpoint{8.772398in}{3.025662in}}%
\pgfpathlineto{\pgfqpoint{8.805488in}{3.056847in}}%
\pgfpathlineto{\pgfqpoint{8.838579in}{3.089650in}}%
\pgfpathlineto{\pgfqpoint{8.871669in}{3.124089in}}%
\pgfpathlineto{\pgfqpoint{8.904759in}{3.160185in}}%
\pgfpathlineto{\pgfqpoint{8.937849in}{3.197958in}}%
\pgfpathlineto{\pgfqpoint{8.970939in}{3.237429in}}%
\pgfpathlineto{\pgfqpoint{9.004030in}{3.278616in}}%
\pgfpathlineto{\pgfqpoint{9.037120in}{3.321542in}}%
\pgfpathlineto{\pgfqpoint{9.070210in}{3.366224in}}%
\pgfpathlineto{\pgfqpoint{9.103300in}{3.412685in}}%
\pgfpathlineto{\pgfqpoint{9.136390in}{3.460944in}}%
\pgfpathlineto{\pgfqpoint{9.169481in}{3.511020in}}%
\pgfpathlineto{\pgfqpoint{9.202571in}{3.562935in}}%
\pgfpathlineto{\pgfqpoint{9.235661in}{3.616709in}}%
\pgfpathlineto{\pgfqpoint{9.268751in}{3.672360in}}%
\pgfpathlineto{\pgfqpoint{9.301841in}{3.729911in}}%
\pgfpathlineto{\pgfqpoint{9.334932in}{3.789380in}}%
\pgfpathlineto{\pgfqpoint{9.368022in}{3.850788in}}%
\pgfpathlineto{\pgfqpoint{9.401112in}{3.914156in}}%
\pgfpathlineto{\pgfqpoint{9.434202in}{3.979502in}}%
\pgfpathlineto{\pgfqpoint{9.467292in}{4.046848in}}%
\pgfpathlineto{\pgfqpoint{9.500382in}{4.116214in}}%
\pgfpathlineto{\pgfqpoint{9.533473in}{4.187619in}}%
\pgfpathlineto{\pgfqpoint{9.566563in}{4.261084in}}%
\pgfpathlineto{\pgfqpoint{9.599653in}{4.336629in}}%
\pgfpathlineto{\pgfqpoint{9.632743in}{4.414274in}}%
\pgfpathlineto{\pgfqpoint{9.665833in}{4.494039in}}%
\pgfpathlineto{\pgfqpoint{9.698924in}{4.575945in}}%
\pgfpathlineto{\pgfqpoint{9.732014in}{4.660011in}}%
\pgfpathlineto{\pgfqpoint{9.765104in}{4.746258in}}%
\pgfpathlineto{\pgfqpoint{9.798194in}{4.834706in}}%
\pgfpathlineto{\pgfqpoint{9.831284in}{4.925375in}}%
\pgfpathlineto{\pgfqpoint{9.864375in}{5.018285in}}%
\pgfpathlineto{\pgfqpoint{9.897465in}{5.113456in}}%
\pgfpathlineto{\pgfqpoint{9.930555in}{5.210909in}}%
\pgfpathlineto{\pgfqpoint{9.980190in}{5.361409in}}%
\pgfpathlineto{\pgfqpoint{10.029826in}{5.517157in}}%
\pgfpathlineto{\pgfqpoint{10.079461in}{5.678218in}}%
\pgfpathlineto{\pgfqpoint{10.129096in}{5.844662in}}%
\pgfpathlineto{\pgfqpoint{10.178732in}{6.016557in}}%
\pgfpathlineto{\pgfqpoint{10.228367in}{6.193970in}}%
\pgfpathlineto{\pgfqpoint{10.278002in}{6.376970in}}%
\pgfpathlineto{\pgfqpoint{10.327637in}{6.565624in}}%
\pgfpathlineto{\pgfqpoint{10.377273in}{6.760000in}}%
\pgfpathlineto{\pgfqpoint{10.377273in}{6.760000in}}%
\pgfusepath{stroke}%
\end{pgfscope}%
\begin{pgfscope}%
\pgfpathrectangle{\pgfqpoint{1.500000in}{0.880000in}}{\pgfqpoint{9.300000in}{6.160000in}}%
\pgfusepath{clip}%
\pgfsetroundcap%
\pgfsetroundjoin%
\pgfsetlinewidth{2.007500pt}%
\definecolor{currentstroke}{rgb}{0.501961,0.000000,0.501961}%
\pgfsetstrokecolor{currentstroke}%
\pgfsetdash{}{0pt}%
\pgfpathmoveto{\pgfqpoint{1.922727in}{6.760000in}}%
\pgfpathlineto{\pgfqpoint{1.955817in}{6.544659in}}%
\pgfpathlineto{\pgfqpoint{1.988908in}{6.336898in}}%
\pgfpathlineto{\pgfqpoint{2.021998in}{6.136577in}}%
\pgfpathlineto{\pgfqpoint{2.055088in}{5.943560in}}%
\pgfpathlineto{\pgfqpoint{2.088178in}{5.757708in}}%
\pgfpathlineto{\pgfqpoint{2.121268in}{5.578885in}}%
\pgfpathlineto{\pgfqpoint{2.154359in}{5.406956in}}%
\pgfpathlineto{\pgfqpoint{2.187449in}{5.241788in}}%
\pgfpathlineto{\pgfqpoint{2.220539in}{5.083248in}}%
\pgfpathlineto{\pgfqpoint{2.253629in}{4.931205in}}%
\pgfpathlineto{\pgfqpoint{2.286719in}{4.785527in}}%
\pgfpathlineto{\pgfqpoint{2.319810in}{4.646085in}}%
\pgfpathlineto{\pgfqpoint{2.352900in}{4.512751in}}%
\pgfpathlineto{\pgfqpoint{2.385990in}{4.385398in}}%
\pgfpathlineto{\pgfqpoint{2.419080in}{4.263899in}}%
\pgfpathlineto{\pgfqpoint{2.452170in}{4.148130in}}%
\pgfpathlineto{\pgfqpoint{2.485261in}{4.037967in}}%
\pgfpathlineto{\pgfqpoint{2.518351in}{3.933287in}}%
\pgfpathlineto{\pgfqpoint{2.551441in}{3.833968in}}%
\pgfpathlineto{\pgfqpoint{2.584531in}{3.739889in}}%
\pgfpathlineto{\pgfqpoint{2.617621in}{3.650932in}}%
\pgfpathlineto{\pgfqpoint{2.650712in}{3.566977in}}%
\pgfpathlineto{\pgfqpoint{2.683802in}{3.487908in}}%
\pgfpathlineto{\pgfqpoint{2.716892in}{3.413608in}}%
\pgfpathlineto{\pgfqpoint{2.749982in}{3.343963in}}%
\pgfpathlineto{\pgfqpoint{2.783072in}{3.278857in}}%
\pgfpathlineto{\pgfqpoint{2.816163in}{3.218179in}}%
\pgfpathlineto{\pgfqpoint{2.849253in}{3.161817in}}%
\pgfpathlineto{\pgfqpoint{2.882343in}{3.109659in}}%
\pgfpathlineto{\pgfqpoint{2.898888in}{3.085123in}}%
\pgfpathlineto{\pgfqpoint{2.915433in}{3.061597in}}%
\pgfpathlineto{\pgfqpoint{2.931978in}{3.039068in}}%
\pgfpathlineto{\pgfqpoint{2.948523in}{3.017522in}}%
\pgfpathlineto{\pgfqpoint{2.965068in}{2.996946in}}%
\pgfpathlineto{\pgfqpoint{2.981614in}{2.977326in}}%
\pgfpathlineto{\pgfqpoint{2.998159in}{2.958650in}}%
\pgfpathlineto{\pgfqpoint{3.014704in}{2.940904in}}%
\pgfpathlineto{\pgfqpoint{3.031249in}{2.924075in}}%
\pgfpathlineto{\pgfqpoint{3.047794in}{2.908150in}}%
\pgfpathlineto{\pgfqpoint{3.064339in}{2.893116in}}%
\pgfpathlineto{\pgfqpoint{3.080884in}{2.878960in}}%
\pgfpathlineto{\pgfqpoint{3.097429in}{2.865669in}}%
\pgfpathlineto{\pgfqpoint{3.113974in}{2.853231in}}%
\pgfpathlineto{\pgfqpoint{3.130519in}{2.841633in}}%
\pgfpathlineto{\pgfqpoint{3.147065in}{2.830862in}}%
\pgfpathlineto{\pgfqpoint{3.163610in}{2.820905in}}%
\pgfpathlineto{\pgfqpoint{3.180155in}{2.811751in}}%
\pgfpathlineto{\pgfqpoint{3.196700in}{2.803386in}}%
\pgfpathlineto{\pgfqpoint{3.213245in}{2.795799in}}%
\pgfpathlineto{\pgfqpoint{3.229790in}{2.788977in}}%
\pgfpathlineto{\pgfqpoint{3.246335in}{2.782908in}}%
\pgfpathlineto{\pgfqpoint{3.262880in}{2.777580in}}%
\pgfpathlineto{\pgfqpoint{3.279425in}{2.772980in}}%
\pgfpathlineto{\pgfqpoint{3.295970in}{2.769098in}}%
\pgfpathlineto{\pgfqpoint{3.312516in}{2.765919in}}%
\pgfpathlineto{\pgfqpoint{3.329061in}{2.763434in}}%
\pgfpathlineto{\pgfqpoint{3.345606in}{2.761630in}}%
\pgfpathlineto{\pgfqpoint{3.362151in}{2.760495in}}%
\pgfpathlineto{\pgfqpoint{3.378696in}{2.760018in}}%
\pgfpathlineto{\pgfqpoint{3.395241in}{2.760187in}}%
\pgfpathlineto{\pgfqpoint{3.411786in}{2.760990in}}%
\pgfpathlineto{\pgfqpoint{3.428331in}{2.762417in}}%
\pgfpathlineto{\pgfqpoint{3.444876in}{2.764456in}}%
\pgfpathlineto{\pgfqpoint{3.461421in}{2.767095in}}%
\pgfpathlineto{\pgfqpoint{3.477967in}{2.770323in}}%
\pgfpathlineto{\pgfqpoint{3.511057in}{2.778502in}}%
\pgfpathlineto{\pgfqpoint{3.544147in}{2.788904in}}%
\pgfpathlineto{\pgfqpoint{3.577237in}{2.801443in}}%
\pgfpathlineto{\pgfqpoint{3.610327in}{2.816032in}}%
\pgfpathlineto{\pgfqpoint{3.643418in}{2.832585in}}%
\pgfpathlineto{\pgfqpoint{3.676508in}{2.851018in}}%
\pgfpathlineto{\pgfqpoint{3.709598in}{2.871248in}}%
\pgfpathlineto{\pgfqpoint{3.742688in}{2.893194in}}%
\pgfpathlineto{\pgfqpoint{3.775778in}{2.916774in}}%
\pgfpathlineto{\pgfqpoint{3.808869in}{2.941908in}}%
\pgfpathlineto{\pgfqpoint{3.841959in}{2.968519in}}%
\pgfpathlineto{\pgfqpoint{3.875049in}{2.996528in}}%
\pgfpathlineto{\pgfqpoint{3.908139in}{3.025859in}}%
\pgfpathlineto{\pgfqpoint{3.941229in}{3.056437in}}%
\pgfpathlineto{\pgfqpoint{3.974320in}{3.088187in}}%
\pgfpathlineto{\pgfqpoint{4.007410in}{3.121037in}}%
\pgfpathlineto{\pgfqpoint{4.057045in}{3.172215in}}%
\pgfpathlineto{\pgfqpoint{4.106680in}{3.225466in}}%
\pgfpathlineto{\pgfqpoint{4.156316in}{3.280558in}}%
\pgfpathlineto{\pgfqpoint{4.205951in}{3.337263in}}%
\pgfpathlineto{\pgfqpoint{4.272131in}{3.414996in}}%
\pgfpathlineto{\pgfqpoint{4.338312in}{3.494694in}}%
\pgfpathlineto{\pgfqpoint{4.421037in}{3.596330in}}%
\pgfpathlineto{\pgfqpoint{4.536853in}{3.740739in}}%
\pgfpathlineto{\pgfqpoint{4.702304in}{3.947033in}}%
\pgfpathlineto{\pgfqpoint{4.785029in}{4.048413in}}%
\pgfpathlineto{\pgfqpoint{4.851210in}{4.127990in}}%
\pgfpathlineto{\pgfqpoint{4.917390in}{4.205822in}}%
\pgfpathlineto{\pgfqpoint{4.983571in}{4.281572in}}%
\pgfpathlineto{\pgfqpoint{5.033206in}{4.336824in}}%
\pgfpathlineto{\pgfqpoint{5.082841in}{4.390597in}}%
\pgfpathlineto{\pgfqpoint{5.132476in}{4.442766in}}%
\pgfpathlineto{\pgfqpoint{5.182112in}{4.493215in}}%
\pgfpathlineto{\pgfqpoint{5.231747in}{4.541831in}}%
\pgfpathlineto{\pgfqpoint{5.281382in}{4.588509in}}%
\pgfpathlineto{\pgfqpoint{5.331018in}{4.633148in}}%
\pgfpathlineto{\pgfqpoint{5.380653in}{4.675652in}}%
\pgfpathlineto{\pgfqpoint{5.413743in}{4.702759in}}%
\pgfpathlineto{\pgfqpoint{5.446833in}{4.728852in}}%
\pgfpathlineto{\pgfqpoint{5.479924in}{4.753908in}}%
\pgfpathlineto{\pgfqpoint{5.513014in}{4.777903in}}%
\pgfpathlineto{\pgfqpoint{5.546104in}{4.800816in}}%
\pgfpathlineto{\pgfqpoint{5.579194in}{4.822627in}}%
\pgfpathlineto{\pgfqpoint{5.612284in}{4.843315in}}%
\pgfpathlineto{\pgfqpoint{5.645374in}{4.862862in}}%
\pgfpathlineto{\pgfqpoint{5.678465in}{4.881250in}}%
\pgfpathlineto{\pgfqpoint{5.711555in}{4.898464in}}%
\pgfpathlineto{\pgfqpoint{5.744645in}{4.914489in}}%
\pgfpathlineto{\pgfqpoint{5.777735in}{4.929309in}}%
\pgfpathlineto{\pgfqpoint{5.810825in}{4.942913in}}%
\pgfpathlineto{\pgfqpoint{5.843916in}{4.955288in}}%
\pgfpathlineto{\pgfqpoint{5.877006in}{4.966423in}}%
\pgfpathlineto{\pgfqpoint{5.910096in}{4.976309in}}%
\pgfpathlineto{\pgfqpoint{5.943186in}{4.984938in}}%
\pgfpathlineto{\pgfqpoint{5.976276in}{4.992302in}}%
\pgfpathlineto{\pgfqpoint{6.009367in}{4.998394in}}%
\pgfpathlineto{\pgfqpoint{6.042457in}{5.003209in}}%
\pgfpathlineto{\pgfqpoint{6.075547in}{5.006744in}}%
\pgfpathlineto{\pgfqpoint{6.108637in}{5.008995in}}%
\pgfpathlineto{\pgfqpoint{6.141727in}{5.009960in}}%
\pgfpathlineto{\pgfqpoint{6.174818in}{5.009638in}}%
\pgfpathlineto{\pgfqpoint{6.207908in}{5.008030in}}%
\pgfpathlineto{\pgfqpoint{6.240998in}{5.005137in}}%
\pgfpathlineto{\pgfqpoint{6.274088in}{5.000962in}}%
\pgfpathlineto{\pgfqpoint{6.307178in}{4.995507in}}%
\pgfpathlineto{\pgfqpoint{6.340269in}{4.988779in}}%
\pgfpathlineto{\pgfqpoint{6.373359in}{4.980781in}}%
\pgfpathlineto{\pgfqpoint{6.406449in}{4.971523in}}%
\pgfpathlineto{\pgfqpoint{6.439539in}{4.961011in}}%
\pgfpathlineto{\pgfqpoint{6.472629in}{4.949254in}}%
\pgfpathlineto{\pgfqpoint{6.505720in}{4.936264in}}%
\pgfpathlineto{\pgfqpoint{6.538810in}{4.922050in}}%
\pgfpathlineto{\pgfqpoint{6.571900in}{4.906626in}}%
\pgfpathlineto{\pgfqpoint{6.604990in}{4.890005in}}%
\pgfpathlineto{\pgfqpoint{6.638080in}{4.872202in}}%
\pgfpathlineto{\pgfqpoint{6.671171in}{4.853232in}}%
\pgfpathlineto{\pgfqpoint{6.704261in}{4.833112in}}%
\pgfpathlineto{\pgfqpoint{6.737351in}{4.811861in}}%
\pgfpathlineto{\pgfqpoint{6.770441in}{4.789496in}}%
\pgfpathlineto{\pgfqpoint{6.803531in}{4.766039in}}%
\pgfpathlineto{\pgfqpoint{6.836622in}{4.741511in}}%
\pgfpathlineto{\pgfqpoint{6.869712in}{4.715934in}}%
\pgfpathlineto{\pgfqpoint{6.902802in}{4.689331in}}%
\pgfpathlineto{\pgfqpoint{6.935892in}{4.661727in}}%
\pgfpathlineto{\pgfqpoint{6.985527in}{4.618501in}}%
\pgfpathlineto{\pgfqpoint{7.035163in}{4.573171in}}%
\pgfpathlineto{\pgfqpoint{7.084798in}{4.525836in}}%
\pgfpathlineto{\pgfqpoint{7.134433in}{4.476597in}}%
\pgfpathlineto{\pgfqpoint{7.184069in}{4.425562in}}%
\pgfpathlineto{\pgfqpoint{7.233704in}{4.372845in}}%
\pgfpathlineto{\pgfqpoint{7.283339in}{4.318565in}}%
\pgfpathlineto{\pgfqpoint{7.332975in}{4.262847in}}%
\pgfpathlineto{\pgfqpoint{7.399155in}{4.186546in}}%
\pgfpathlineto{\pgfqpoint{7.465335in}{4.108246in}}%
\pgfpathlineto{\pgfqpoint{7.531516in}{4.028288in}}%
\pgfpathlineto{\pgfqpoint{7.614241in}{3.926560in}}%
\pgfpathlineto{\pgfqpoint{7.730057in}{3.782157in}}%
\pgfpathlineto{\pgfqpoint{7.895508in}{3.575864in}}%
\pgfpathlineto{\pgfqpoint{7.978233in}{3.474612in}}%
\pgfpathlineto{\pgfqpoint{8.044414in}{3.395359in}}%
\pgfpathlineto{\pgfqpoint{8.094049in}{3.337263in}}%
\pgfpathlineto{\pgfqpoint{8.143684in}{3.280558in}}%
\pgfpathlineto{\pgfqpoint{8.193320in}{3.225466in}}%
\pgfpathlineto{\pgfqpoint{8.242955in}{3.172215in}}%
\pgfpathlineto{\pgfqpoint{8.292590in}{3.121037in}}%
\pgfpathlineto{\pgfqpoint{8.325680in}{3.088187in}}%
\pgfpathlineto{\pgfqpoint{8.358771in}{3.056437in}}%
\pgfpathlineto{\pgfqpoint{8.391861in}{3.025859in}}%
\pgfpathlineto{\pgfqpoint{8.424951in}{2.996528in}}%
\pgfpathlineto{\pgfqpoint{8.458041in}{2.968519in}}%
\pgfpathlineto{\pgfqpoint{8.491131in}{2.941908in}}%
\pgfpathlineto{\pgfqpoint{8.524222in}{2.916774in}}%
\pgfpathlineto{\pgfqpoint{8.557312in}{2.893194in}}%
\pgfpathlineto{\pgfqpoint{8.590402in}{2.871248in}}%
\pgfpathlineto{\pgfqpoint{8.623492in}{2.851018in}}%
\pgfpathlineto{\pgfqpoint{8.656582in}{2.832585in}}%
\pgfpathlineto{\pgfqpoint{8.689673in}{2.816032in}}%
\pgfpathlineto{\pgfqpoint{8.722763in}{2.801443in}}%
\pgfpathlineto{\pgfqpoint{8.755853in}{2.788904in}}%
\pgfpathlineto{\pgfqpoint{8.788943in}{2.778502in}}%
\pgfpathlineto{\pgfqpoint{8.822033in}{2.770323in}}%
\pgfpathlineto{\pgfqpoint{8.838579in}{2.767095in}}%
\pgfpathlineto{\pgfqpoint{8.855124in}{2.764456in}}%
\pgfpathlineto{\pgfqpoint{8.871669in}{2.762417in}}%
\pgfpathlineto{\pgfqpoint{8.888214in}{2.760990in}}%
\pgfpathlineto{\pgfqpoint{8.904759in}{2.760187in}}%
\pgfpathlineto{\pgfqpoint{8.921304in}{2.760018in}}%
\pgfpathlineto{\pgfqpoint{8.937849in}{2.760495in}}%
\pgfpathlineto{\pgfqpoint{8.954394in}{2.761630in}}%
\pgfpathlineto{\pgfqpoint{8.970939in}{2.763434in}}%
\pgfpathlineto{\pgfqpoint{8.987484in}{2.765919in}}%
\pgfpathlineto{\pgfqpoint{9.004030in}{2.769098in}}%
\pgfpathlineto{\pgfqpoint{9.020575in}{2.772980in}}%
\pgfpathlineto{\pgfqpoint{9.037120in}{2.777580in}}%
\pgfpathlineto{\pgfqpoint{9.053665in}{2.782908in}}%
\pgfpathlineto{\pgfqpoint{9.070210in}{2.788977in}}%
\pgfpathlineto{\pgfqpoint{9.086755in}{2.795799in}}%
\pgfpathlineto{\pgfqpoint{9.103300in}{2.803386in}}%
\pgfpathlineto{\pgfqpoint{9.119845in}{2.811751in}}%
\pgfpathlineto{\pgfqpoint{9.136390in}{2.820905in}}%
\pgfpathlineto{\pgfqpoint{9.152935in}{2.830862in}}%
\pgfpathlineto{\pgfqpoint{9.169481in}{2.841633in}}%
\pgfpathlineto{\pgfqpoint{9.186026in}{2.853231in}}%
\pgfpathlineto{\pgfqpoint{9.202571in}{2.865669in}}%
\pgfpathlineto{\pgfqpoint{9.219116in}{2.878960in}}%
\pgfpathlineto{\pgfqpoint{9.235661in}{2.893116in}}%
\pgfpathlineto{\pgfqpoint{9.252206in}{2.908150in}}%
\pgfpathlineto{\pgfqpoint{9.268751in}{2.924075in}}%
\pgfpathlineto{\pgfqpoint{9.285296in}{2.940904in}}%
\pgfpathlineto{\pgfqpoint{9.301841in}{2.958650in}}%
\pgfpathlineto{\pgfqpoint{9.318386in}{2.977326in}}%
\pgfpathlineto{\pgfqpoint{9.334932in}{2.996946in}}%
\pgfpathlineto{\pgfqpoint{9.351477in}{3.017522in}}%
\pgfpathlineto{\pgfqpoint{9.368022in}{3.039068in}}%
\pgfpathlineto{\pgfqpoint{9.384567in}{3.061597in}}%
\pgfpathlineto{\pgfqpoint{9.401112in}{3.085123in}}%
\pgfpathlineto{\pgfqpoint{9.417657in}{3.109659in}}%
\pgfpathlineto{\pgfqpoint{9.434202in}{3.135219in}}%
\pgfpathlineto{\pgfqpoint{9.467292in}{3.189465in}}%
\pgfpathlineto{\pgfqpoint{9.500382in}{3.247972in}}%
\pgfpathlineto{\pgfqpoint{9.533473in}{3.310849in}}%
\pgfpathlineto{\pgfqpoint{9.566563in}{3.378211in}}%
\pgfpathlineto{\pgfqpoint{9.599653in}{3.450169in}}%
\pgfpathlineto{\pgfqpoint{9.632743in}{3.526839in}}%
\pgfpathlineto{\pgfqpoint{9.665833in}{3.608337in}}%
\pgfpathlineto{\pgfqpoint{9.698924in}{3.694778in}}%
\pgfpathlineto{\pgfqpoint{9.732014in}{3.786281in}}%
\pgfpathlineto{\pgfqpoint{9.765104in}{3.882965in}}%
\pgfpathlineto{\pgfqpoint{9.798194in}{3.984949in}}%
\pgfpathlineto{\pgfqpoint{9.831284in}{4.092356in}}%
\pgfpathlineto{\pgfqpoint{9.864375in}{4.205306in}}%
\pgfpathlineto{\pgfqpoint{9.897465in}{4.323924in}}%
\pgfpathlineto{\pgfqpoint{9.930555in}{4.448335in}}%
\pgfpathlineto{\pgfqpoint{9.963645in}{4.578662in}}%
\pgfpathlineto{\pgfqpoint{9.996735in}{4.715034in}}%
\pgfpathlineto{\pgfqpoint{10.029826in}{4.857578in}}%
\pgfpathlineto{\pgfqpoint{10.062916in}{5.006423in}}%
\pgfpathlineto{\pgfqpoint{10.096006in}{5.161698in}}%
\pgfpathlineto{\pgfqpoint{10.129096in}{5.323536in}}%
\pgfpathlineto{\pgfqpoint{10.162186in}{5.492067in}}%
\pgfpathlineto{\pgfqpoint{10.195277in}{5.667426in}}%
\pgfpathlineto{\pgfqpoint{10.228367in}{5.849746in}}%
\pgfpathlineto{\pgfqpoint{10.261457in}{6.039164in}}%
\pgfpathlineto{\pgfqpoint{10.294547in}{6.235816in}}%
\pgfpathlineto{\pgfqpoint{10.327637in}{6.439839in}}%
\pgfpathlineto{\pgfqpoint{10.360728in}{6.651373in}}%
\pgfpathlineto{\pgfqpoint{10.377273in}{6.760000in}}%
\pgfpathlineto{\pgfqpoint{10.377273in}{6.760000in}}%
\pgfusepath{stroke}%
\end{pgfscope}%
\begin{pgfscope}%
\pgfsetrectcap%
\pgfsetmiterjoin%
\pgfsetlinewidth{0.803000pt}%
\definecolor{currentstroke}{rgb}{0.800000,0.800000,0.800000}%
\pgfsetstrokecolor{currentstroke}%
\pgfsetdash{}{0pt}%
\pgfpathmoveto{\pgfqpoint{1.500000in}{0.880000in}}%
\pgfpathlineto{\pgfqpoint{1.500000in}{7.040000in}}%
\pgfusepath{stroke}%
\end{pgfscope}%
\begin{pgfscope}%
\pgfsetrectcap%
\pgfsetmiterjoin%
\pgfsetlinewidth{0.803000pt}%
\definecolor{currentstroke}{rgb}{0.800000,0.800000,0.800000}%
\pgfsetstrokecolor{currentstroke}%
\pgfsetdash{}{0pt}%
\pgfpathmoveto{\pgfqpoint{10.800000in}{0.880000in}}%
\pgfpathlineto{\pgfqpoint{10.800000in}{7.040000in}}%
\pgfusepath{stroke}%
\end{pgfscope}%
\begin{pgfscope}%
\pgfsetrectcap%
\pgfsetmiterjoin%
\pgfsetlinewidth{0.803000pt}%
\definecolor{currentstroke}{rgb}{0.800000,0.800000,0.800000}%
\pgfsetstrokecolor{currentstroke}%
\pgfsetdash{}{0pt}%
\pgfpathmoveto{\pgfqpoint{1.500000in}{0.880000in}}%
\pgfpathlineto{\pgfqpoint{10.800000in}{0.880000in}}%
\pgfusepath{stroke}%
\end{pgfscope}%
\begin{pgfscope}%
\pgfsetrectcap%
\pgfsetmiterjoin%
\pgfsetlinewidth{0.803000pt}%
\definecolor{currentstroke}{rgb}{0.800000,0.800000,0.800000}%
\pgfsetstrokecolor{currentstroke}%
\pgfsetdash{}{0pt}%
\pgfpathmoveto{\pgfqpoint{1.500000in}{7.040000in}}%
\pgfpathlineto{\pgfqpoint{10.800000in}{7.040000in}}%
\pgfusepath{stroke}%
\end{pgfscope}%
\begin{pgfscope}%
\definecolor{textcolor}{rgb}{0.150000,0.150000,0.150000}%
\pgfsetstrokecolor{textcolor}%
\pgfsetfillcolor{textcolor}%
\pgftext[x=6.150000in,y=7.123333in,,base]{\color{textcolor}{\sffamily\fontsize{16.000000}{19.200000}\bfseries\selectfont\catcode`\^=\active\def^{\ifmmode\sp\else\^{}\fi}\catcode`\%=\active\def%{\%}First Few Legendre Polynomials}}%
\end{pgfscope}%
\begin{pgfscope}%
\pgfsetbuttcap%
\pgfsetmiterjoin%
\definecolor{currentfill}{rgb}{1.000000,1.000000,1.000000}%
\pgfsetfillcolor{currentfill}%
\pgfsetfillopacity{0.800000}%
\pgfsetlinewidth{1.003750pt}%
\definecolor{currentstroke}{rgb}{0.800000,0.800000,0.800000}%
\pgfsetstrokecolor{currentstroke}%
\pgfsetstrokeopacity{0.800000}%
\pgfsetdash{}{0pt}%
\pgfpathmoveto{\pgfqpoint{9.759174in}{0.963333in}}%
\pgfpathlineto{\pgfqpoint{10.683333in}{0.963333in}}%
\pgfpathquadraticcurveto{\pgfqpoint{10.716667in}{0.963333in}}{\pgfqpoint{10.716667in}{0.996667in}}%
\pgfpathlineto{\pgfqpoint{10.716667in}{2.230000in}}%
\pgfpathquadraticcurveto{\pgfqpoint{10.716667in}{2.263333in}}{\pgfqpoint{10.683333in}{2.263333in}}%
\pgfpathlineto{\pgfqpoint{9.759174in}{2.263333in}}%
\pgfpathquadraticcurveto{\pgfqpoint{9.725841in}{2.263333in}}{\pgfqpoint{9.725841in}{2.230000in}}%
\pgfpathlineto{\pgfqpoint{9.725841in}{0.996667in}}%
\pgfpathquadraticcurveto{\pgfqpoint{9.725841in}{0.963333in}}{\pgfqpoint{9.759174in}{0.963333in}}%
\pgfpathlineto{\pgfqpoint{9.759174in}{0.963333in}}%
\pgfpathclose%
\pgfusepath{stroke,fill}%
\end{pgfscope}%
\begin{pgfscope}%
\pgfsetroundcap%
\pgfsetroundjoin%
\pgfsetlinewidth{2.007500pt}%
\definecolor{currentstroke}{rgb}{0.000000,0.000000,1.000000}%
\pgfsetstrokecolor{currentstroke}%
\pgfsetdash{}{0pt}%
\pgfpathmoveto{\pgfqpoint{9.792508in}{2.130000in}}%
\pgfpathlineto{\pgfqpoint{9.959174in}{2.130000in}}%
\pgfpathlineto{\pgfqpoint{10.125841in}{2.130000in}}%
\pgfusepath{stroke}%
\end{pgfscope}%
\begin{pgfscope}%
\definecolor{textcolor}{rgb}{0.150000,0.150000,0.150000}%
\pgfsetstrokecolor{textcolor}%
\pgfsetfillcolor{textcolor}%
\pgftext[x=10.259174in,y=2.071667in,left,base]{\color{textcolor}{\sffamily\fontsize{12.000000}{14.400000}\selectfont\catcode`\^=\active\def^{\ifmmode\sp\else\^{}\fi}\catcode`\%=\active\def%{\%}$P_0(x)$}}%
\end{pgfscope}%
\begin{pgfscope}%
\pgfsetroundcap%
\pgfsetroundjoin%
\pgfsetlinewidth{2.007500pt}%
\definecolor{currentstroke}{rgb}{1.000000,0.000000,0.000000}%
\pgfsetstrokecolor{currentstroke}%
\pgfsetdash{}{0pt}%
\pgfpathmoveto{\pgfqpoint{9.792508in}{1.880000in}}%
\pgfpathlineto{\pgfqpoint{9.959174in}{1.880000in}}%
\pgfpathlineto{\pgfqpoint{10.125841in}{1.880000in}}%
\pgfusepath{stroke}%
\end{pgfscope}%
\begin{pgfscope}%
\definecolor{textcolor}{rgb}{0.150000,0.150000,0.150000}%
\pgfsetstrokecolor{textcolor}%
\pgfsetfillcolor{textcolor}%
\pgftext[x=10.259174in,y=1.821667in,left,base]{\color{textcolor}{\sffamily\fontsize{12.000000}{14.400000}\selectfont\catcode`\^=\active\def^{\ifmmode\sp\else\^{}\fi}\catcode`\%=\active\def%{\%}$P_1(x)$}}%
\end{pgfscope}%
\begin{pgfscope}%
\pgfsetroundcap%
\pgfsetroundjoin%
\pgfsetlinewidth{2.007500pt}%
\definecolor{currentstroke}{rgb}{0.000000,0.501961,0.000000}%
\pgfsetstrokecolor{currentstroke}%
\pgfsetdash{}{0pt}%
\pgfpathmoveto{\pgfqpoint{9.792508in}{1.630000in}}%
\pgfpathlineto{\pgfqpoint{9.959174in}{1.630000in}}%
\pgfpathlineto{\pgfqpoint{10.125841in}{1.630000in}}%
\pgfusepath{stroke}%
\end{pgfscope}%
\begin{pgfscope}%
\definecolor{textcolor}{rgb}{0.150000,0.150000,0.150000}%
\pgfsetstrokecolor{textcolor}%
\pgfsetfillcolor{textcolor}%
\pgftext[x=10.259174in,y=1.571667in,left,base]{\color{textcolor}{\sffamily\fontsize{12.000000}{14.400000}\selectfont\catcode`\^=\active\def^{\ifmmode\sp\else\^{}\fi}\catcode`\%=\active\def%{\%}$P_2(x)$}}%
\end{pgfscope}%
\begin{pgfscope}%
\pgfsetroundcap%
\pgfsetroundjoin%
\pgfsetlinewidth{2.007500pt}%
\definecolor{currentstroke}{rgb}{0.000000,0.000000,0.000000}%
\pgfsetstrokecolor{currentstroke}%
\pgfsetdash{}{0pt}%
\pgfpathmoveto{\pgfqpoint{9.792508in}{1.380000in}}%
\pgfpathlineto{\pgfqpoint{9.959174in}{1.380000in}}%
\pgfpathlineto{\pgfqpoint{10.125841in}{1.380000in}}%
\pgfusepath{stroke}%
\end{pgfscope}%
\begin{pgfscope}%
\definecolor{textcolor}{rgb}{0.150000,0.150000,0.150000}%
\pgfsetstrokecolor{textcolor}%
\pgfsetfillcolor{textcolor}%
\pgftext[x=10.259174in,y=1.321667in,left,base]{\color{textcolor}{\sffamily\fontsize{12.000000}{14.400000}\selectfont\catcode`\^=\active\def^{\ifmmode\sp\else\^{}\fi}\catcode`\%=\active\def%{\%}$P_3(x)$}}%
\end{pgfscope}%
\begin{pgfscope}%
\pgfsetroundcap%
\pgfsetroundjoin%
\pgfsetlinewidth{2.007500pt}%
\definecolor{currentstroke}{rgb}{0.501961,0.000000,0.501961}%
\pgfsetstrokecolor{currentstroke}%
\pgfsetdash{}{0pt}%
\pgfpathmoveto{\pgfqpoint{9.792508in}{1.130000in}}%
\pgfpathlineto{\pgfqpoint{9.959174in}{1.130000in}}%
\pgfpathlineto{\pgfqpoint{10.125841in}{1.130000in}}%
\pgfusepath{stroke}%
\end{pgfscope}%
\begin{pgfscope}%
\definecolor{textcolor}{rgb}{0.150000,0.150000,0.150000}%
\pgfsetstrokecolor{textcolor}%
\pgfsetfillcolor{textcolor}%
\pgftext[x=10.259174in,y=1.071667in,left,base]{\color{textcolor}{\sffamily\fontsize{12.000000}{14.400000}\selectfont\catcode`\^=\active\def^{\ifmmode\sp\else\^{}\fi}\catcode`\%=\active\def%{\%}$P_4(x)$}}%
\end{pgfscope}%
\end{pgfpicture}%
\makeatother%
\endgroup%
}
            \caption{Graph of the first few Legendre polynomials}
            \label{img:Legendre_polynomials}
        \end{figure}
        \noindent The Legendre polynomials appear in the angular part of the solution to hydrogenic wavefunctions as discussed in Sec.~\ref{sec:The_Angular_Part}. They are also incorporated into the spherical harmonic function discussed in Sec.~\ref{sec:Spherical_Harmonics}.

        \section{Laguerre Polynomials} \label{sec:Laguerre_Polynomial}
        The Laguerre ODE is 
        \begin{align}
            xy^{\prime \prime}(x) + (1 - x)y^\prime(x) + ny(x) = 0 
        \end{align}
        \begin{figure}[h]
            \centering
            \resizebox{0.8\linewidth}{!}{%% Creator: Matplotlib, PGF backend
%%
%% To include the figure in your LaTeX document, write
%%   \input{<filename>.pgf}
%%
%% Make sure the required packages are loaded in your preamble
%%   \usepackage{pgf}
%%
%% Also ensure that all the required font packages are loaded; for instance,
%% the lmodern package is sometimes necessary when using math font.
%%   \usepackage{lmodern}
%%
%% Figures using additional raster images can only be included by \input if
%% they are in the same directory as the main LaTeX file. For loading figures
%% from other directories you can use the `import` package
%%   \usepackage{import}
%%
%% and then include the figures with
%%   \import{<path to file>}{<filename>.pgf}
%%
%% Matplotlib used the following preamble
%%   \def\mathdefault#1{#1}
%%   \everymath=\expandafter{\the\everymath\displaystyle}
%%   
%%   \ifdefined\pdftexversion\else  % non-pdftex case.
%%     \usepackage{fontspec}
%%     \setmainfont{DejaVuSerif.ttf}[Path=\detokenize{/Users/evanpetrimoulx/.pyenv/versions/3.12.6/lib/python3.12/site-packages/matplotlib/mpl-data/fonts/ttf/}]
%%     \setsansfont{Arial.ttf}[Path=\detokenize{/System/Library/Fonts/Supplemental/}]
%%     \setmonofont{DejaVuSansMono.ttf}[Path=\detokenize{/Users/evanpetrimoulx/.pyenv/versions/3.12.6/lib/python3.12/site-packages/matplotlib/mpl-data/fonts/ttf/}]
%%   \fi
%%   \makeatletter\@ifpackageloaded{underscore}{}{\usepackage[strings]{underscore}}\makeatother
%%
\begingroup%
\makeatletter%
\begin{pgfpicture}%
\pgfpathrectangle{\pgfpointorigin}{\pgfqpoint{12.000000in}{8.000000in}}%
\pgfusepath{use as bounding box, clip}%
\begin{pgfscope}%
\pgfsetbuttcap%
\pgfsetmiterjoin%
\definecolor{currentfill}{rgb}{1.000000,1.000000,1.000000}%
\pgfsetfillcolor{currentfill}%
\pgfsetlinewidth{0.000000pt}%
\definecolor{currentstroke}{rgb}{1.000000,1.000000,1.000000}%
\pgfsetstrokecolor{currentstroke}%
\pgfsetdash{}{0pt}%
\pgfpathmoveto{\pgfqpoint{0.000000in}{0.000000in}}%
\pgfpathlineto{\pgfqpoint{12.000000in}{0.000000in}}%
\pgfpathlineto{\pgfqpoint{12.000000in}{8.000000in}}%
\pgfpathlineto{\pgfqpoint{0.000000in}{8.000000in}}%
\pgfpathlineto{\pgfqpoint{0.000000in}{0.000000in}}%
\pgfpathclose%
\pgfusepath{fill}%
\end{pgfscope}%
\begin{pgfscope}%
\pgfsetbuttcap%
\pgfsetmiterjoin%
\definecolor{currentfill}{rgb}{1.000000,1.000000,1.000000}%
\pgfsetfillcolor{currentfill}%
\pgfsetlinewidth{0.000000pt}%
\definecolor{currentstroke}{rgb}{0.000000,0.000000,0.000000}%
\pgfsetstrokecolor{currentstroke}%
\pgfsetstrokeopacity{0.000000}%
\pgfsetdash{}{0pt}%
\pgfpathmoveto{\pgfqpoint{1.500000in}{0.880000in}}%
\pgfpathlineto{\pgfqpoint{10.800000in}{0.880000in}}%
\pgfpathlineto{\pgfqpoint{10.800000in}{7.040000in}}%
\pgfpathlineto{\pgfqpoint{1.500000in}{7.040000in}}%
\pgfpathlineto{\pgfqpoint{1.500000in}{0.880000in}}%
\pgfpathclose%
\pgfusepath{fill}%
\end{pgfscope}%
\begin{pgfscope}%
\pgfpathrectangle{\pgfqpoint{1.500000in}{0.880000in}}{\pgfqpoint{9.300000in}{6.160000in}}%
\pgfusepath{clip}%
\pgfsetbuttcap%
\pgfsetroundjoin%
\pgfsetlinewidth{0.602250pt}%
\definecolor{currentstroke}{rgb}{0.800000,0.800000,0.800000}%
\pgfsetstrokecolor{currentstroke}%
\pgfsetdash{{2.220000pt}{0.960000pt}}{0.000000pt}%
\pgfpathmoveto{\pgfqpoint{2.120000in}{0.880000in}}%
\pgfpathlineto{\pgfqpoint{2.120000in}{7.040000in}}%
\pgfusepath{stroke}%
\end{pgfscope}%
\begin{pgfscope}%
\definecolor{textcolor}{rgb}{0.150000,0.150000,0.150000}%
\pgfsetstrokecolor{textcolor}%
\pgfsetfillcolor{textcolor}%
\pgftext[x=2.120000in,y=0.782778in,,top]{\color{textcolor}{\sffamily\fontsize{10.000000}{12.000000}\selectfont\catcode`\^=\active\def^{\ifmmode\sp\else\^{}\fi}\catcode`\%=\active\def%{\%}\ensuremath{-}4}}%
\end{pgfscope}%
\begin{pgfscope}%
\pgfpathrectangle{\pgfqpoint{1.500000in}{0.880000in}}{\pgfqpoint{9.300000in}{6.160000in}}%
\pgfusepath{clip}%
\pgfsetbuttcap%
\pgfsetroundjoin%
\pgfsetlinewidth{0.602250pt}%
\definecolor{currentstroke}{rgb}{0.800000,0.800000,0.800000}%
\pgfsetstrokecolor{currentstroke}%
\pgfsetdash{{2.220000pt}{0.960000pt}}{0.000000pt}%
\pgfpathmoveto{\pgfqpoint{3.360000in}{0.880000in}}%
\pgfpathlineto{\pgfqpoint{3.360000in}{7.040000in}}%
\pgfusepath{stroke}%
\end{pgfscope}%
\begin{pgfscope}%
\definecolor{textcolor}{rgb}{0.150000,0.150000,0.150000}%
\pgfsetstrokecolor{textcolor}%
\pgfsetfillcolor{textcolor}%
\pgftext[x=3.360000in,y=0.782778in,,top]{\color{textcolor}{\sffamily\fontsize{10.000000}{12.000000}\selectfont\catcode`\^=\active\def^{\ifmmode\sp\else\^{}\fi}\catcode`\%=\active\def%{\%}\ensuremath{-}2}}%
\end{pgfscope}%
\begin{pgfscope}%
\pgfpathrectangle{\pgfqpoint{1.500000in}{0.880000in}}{\pgfqpoint{9.300000in}{6.160000in}}%
\pgfusepath{clip}%
\pgfsetbuttcap%
\pgfsetroundjoin%
\pgfsetlinewidth{0.602250pt}%
\definecolor{currentstroke}{rgb}{0.800000,0.800000,0.800000}%
\pgfsetstrokecolor{currentstroke}%
\pgfsetdash{{2.220000pt}{0.960000pt}}{0.000000pt}%
\pgfpathmoveto{\pgfqpoint{4.600000in}{0.880000in}}%
\pgfpathlineto{\pgfqpoint{4.600000in}{7.040000in}}%
\pgfusepath{stroke}%
\end{pgfscope}%
\begin{pgfscope}%
\definecolor{textcolor}{rgb}{0.150000,0.150000,0.150000}%
\pgfsetstrokecolor{textcolor}%
\pgfsetfillcolor{textcolor}%
\pgftext[x=4.600000in,y=0.782778in,,top]{\color{textcolor}{\sffamily\fontsize{10.000000}{12.000000}\selectfont\catcode`\^=\active\def^{\ifmmode\sp\else\^{}\fi}\catcode`\%=\active\def%{\%}0}}%
\end{pgfscope}%
\begin{pgfscope}%
\pgfpathrectangle{\pgfqpoint{1.500000in}{0.880000in}}{\pgfqpoint{9.300000in}{6.160000in}}%
\pgfusepath{clip}%
\pgfsetbuttcap%
\pgfsetroundjoin%
\pgfsetlinewidth{0.602250pt}%
\definecolor{currentstroke}{rgb}{0.800000,0.800000,0.800000}%
\pgfsetstrokecolor{currentstroke}%
\pgfsetdash{{2.220000pt}{0.960000pt}}{0.000000pt}%
\pgfpathmoveto{\pgfqpoint{5.840000in}{0.880000in}}%
\pgfpathlineto{\pgfqpoint{5.840000in}{7.040000in}}%
\pgfusepath{stroke}%
\end{pgfscope}%
\begin{pgfscope}%
\definecolor{textcolor}{rgb}{0.150000,0.150000,0.150000}%
\pgfsetstrokecolor{textcolor}%
\pgfsetfillcolor{textcolor}%
\pgftext[x=5.840000in,y=0.782778in,,top]{\color{textcolor}{\sffamily\fontsize{10.000000}{12.000000}\selectfont\catcode`\^=\active\def^{\ifmmode\sp\else\^{}\fi}\catcode`\%=\active\def%{\%}2}}%
\end{pgfscope}%
\begin{pgfscope}%
\pgfpathrectangle{\pgfqpoint{1.500000in}{0.880000in}}{\pgfqpoint{9.300000in}{6.160000in}}%
\pgfusepath{clip}%
\pgfsetbuttcap%
\pgfsetroundjoin%
\pgfsetlinewidth{0.602250pt}%
\definecolor{currentstroke}{rgb}{0.800000,0.800000,0.800000}%
\pgfsetstrokecolor{currentstroke}%
\pgfsetdash{{2.220000pt}{0.960000pt}}{0.000000pt}%
\pgfpathmoveto{\pgfqpoint{7.080000in}{0.880000in}}%
\pgfpathlineto{\pgfqpoint{7.080000in}{7.040000in}}%
\pgfusepath{stroke}%
\end{pgfscope}%
\begin{pgfscope}%
\definecolor{textcolor}{rgb}{0.150000,0.150000,0.150000}%
\pgfsetstrokecolor{textcolor}%
\pgfsetfillcolor{textcolor}%
\pgftext[x=7.080000in,y=0.782778in,,top]{\color{textcolor}{\sffamily\fontsize{10.000000}{12.000000}\selectfont\catcode`\^=\active\def^{\ifmmode\sp\else\^{}\fi}\catcode`\%=\active\def%{\%}4}}%
\end{pgfscope}%
\begin{pgfscope}%
\pgfpathrectangle{\pgfqpoint{1.500000in}{0.880000in}}{\pgfqpoint{9.300000in}{6.160000in}}%
\pgfusepath{clip}%
\pgfsetbuttcap%
\pgfsetroundjoin%
\pgfsetlinewidth{0.602250pt}%
\definecolor{currentstroke}{rgb}{0.800000,0.800000,0.800000}%
\pgfsetstrokecolor{currentstroke}%
\pgfsetdash{{2.220000pt}{0.960000pt}}{0.000000pt}%
\pgfpathmoveto{\pgfqpoint{8.320000in}{0.880000in}}%
\pgfpathlineto{\pgfqpoint{8.320000in}{7.040000in}}%
\pgfusepath{stroke}%
\end{pgfscope}%
\begin{pgfscope}%
\definecolor{textcolor}{rgb}{0.150000,0.150000,0.150000}%
\pgfsetstrokecolor{textcolor}%
\pgfsetfillcolor{textcolor}%
\pgftext[x=8.320000in,y=0.782778in,,top]{\color{textcolor}{\sffamily\fontsize{10.000000}{12.000000}\selectfont\catcode`\^=\active\def^{\ifmmode\sp\else\^{}\fi}\catcode`\%=\active\def%{\%}6}}%
\end{pgfscope}%
\begin{pgfscope}%
\pgfpathrectangle{\pgfqpoint{1.500000in}{0.880000in}}{\pgfqpoint{9.300000in}{6.160000in}}%
\pgfusepath{clip}%
\pgfsetbuttcap%
\pgfsetroundjoin%
\pgfsetlinewidth{0.602250pt}%
\definecolor{currentstroke}{rgb}{0.800000,0.800000,0.800000}%
\pgfsetstrokecolor{currentstroke}%
\pgfsetdash{{2.220000pt}{0.960000pt}}{0.000000pt}%
\pgfpathmoveto{\pgfqpoint{9.560000in}{0.880000in}}%
\pgfpathlineto{\pgfqpoint{9.560000in}{7.040000in}}%
\pgfusepath{stroke}%
\end{pgfscope}%
\begin{pgfscope}%
\definecolor{textcolor}{rgb}{0.150000,0.150000,0.150000}%
\pgfsetstrokecolor{textcolor}%
\pgfsetfillcolor{textcolor}%
\pgftext[x=9.560000in,y=0.782778in,,top]{\color{textcolor}{\sffamily\fontsize{10.000000}{12.000000}\selectfont\catcode`\^=\active\def^{\ifmmode\sp\else\^{}\fi}\catcode`\%=\active\def%{\%}8}}%
\end{pgfscope}%
\begin{pgfscope}%
\pgfpathrectangle{\pgfqpoint{1.500000in}{0.880000in}}{\pgfqpoint{9.300000in}{6.160000in}}%
\pgfusepath{clip}%
\pgfsetbuttcap%
\pgfsetroundjoin%
\pgfsetlinewidth{0.602250pt}%
\definecolor{currentstroke}{rgb}{0.800000,0.800000,0.800000}%
\pgfsetstrokecolor{currentstroke}%
\pgfsetdash{{2.220000pt}{0.960000pt}}{0.000000pt}%
\pgfpathmoveto{\pgfqpoint{10.800000in}{0.880000in}}%
\pgfpathlineto{\pgfqpoint{10.800000in}{7.040000in}}%
\pgfusepath{stroke}%
\end{pgfscope}%
\begin{pgfscope}%
\definecolor{textcolor}{rgb}{0.150000,0.150000,0.150000}%
\pgfsetstrokecolor{textcolor}%
\pgfsetfillcolor{textcolor}%
\pgftext[x=10.800000in,y=0.782778in,,top]{\color{textcolor}{\sffamily\fontsize{10.000000}{12.000000}\selectfont\catcode`\^=\active\def^{\ifmmode\sp\else\^{}\fi}\catcode`\%=\active\def%{\%}10}}%
\end{pgfscope}%
\begin{pgfscope}%
\definecolor{textcolor}{rgb}{0.150000,0.150000,0.150000}%
\pgfsetstrokecolor{textcolor}%
\pgfsetfillcolor{textcolor}%
\pgftext[x=6.150000in,y=0.600202in,,top]{\color{textcolor}{\sffamily\fontsize{14.000000}{16.800000}\selectfont\catcode`\^=\active\def^{\ifmmode\sp\else\^{}\fi}\catcode`\%=\active\def%{\%}x}}%
\end{pgfscope}%
\begin{pgfscope}%
\pgfpathrectangle{\pgfqpoint{1.500000in}{0.880000in}}{\pgfqpoint{9.300000in}{6.160000in}}%
\pgfusepath{clip}%
\pgfsetbuttcap%
\pgfsetroundjoin%
\pgfsetlinewidth{0.602250pt}%
\definecolor{currentstroke}{rgb}{0.800000,0.800000,0.800000}%
\pgfsetstrokecolor{currentstroke}%
\pgfsetdash{{2.220000pt}{0.960000pt}}{0.000000pt}%
\pgfpathmoveto{\pgfqpoint{1.500000in}{0.880000in}}%
\pgfpathlineto{\pgfqpoint{10.800000in}{0.880000in}}%
\pgfusepath{stroke}%
\end{pgfscope}%
\begin{pgfscope}%
\definecolor{textcolor}{rgb}{0.150000,0.150000,0.150000}%
\pgfsetstrokecolor{textcolor}%
\pgfsetfillcolor{textcolor}%
\pgftext[x=1.140266in, y=0.830290in, left, base]{\color{textcolor}{\sffamily\fontsize{10.000000}{12.000000}\selectfont\catcode`\^=\active\def^{\ifmmode\sp\else\^{}\fi}\catcode`\%=\active\def%{\%}\ensuremath{-}20}}%
\end{pgfscope}%
\begin{pgfscope}%
\pgfpathrectangle{\pgfqpoint{1.500000in}{0.880000in}}{\pgfqpoint{9.300000in}{6.160000in}}%
\pgfusepath{clip}%
\pgfsetbuttcap%
\pgfsetroundjoin%
\pgfsetlinewidth{0.602250pt}%
\definecolor{currentstroke}{rgb}{0.800000,0.800000,0.800000}%
\pgfsetstrokecolor{currentstroke}%
\pgfsetdash{{2.220000pt}{0.960000pt}}{0.000000pt}%
\pgfpathmoveto{\pgfqpoint{1.500000in}{1.650000in}}%
\pgfpathlineto{\pgfqpoint{10.800000in}{1.650000in}}%
\pgfusepath{stroke}%
\end{pgfscope}%
\begin{pgfscope}%
\definecolor{textcolor}{rgb}{0.150000,0.150000,0.150000}%
\pgfsetstrokecolor{textcolor}%
\pgfsetfillcolor{textcolor}%
\pgftext[x=1.140266in, y=1.600290in, left, base]{\color{textcolor}{\sffamily\fontsize{10.000000}{12.000000}\selectfont\catcode`\^=\active\def^{\ifmmode\sp\else\^{}\fi}\catcode`\%=\active\def%{\%}\ensuremath{-}15}}%
\end{pgfscope}%
\begin{pgfscope}%
\pgfpathrectangle{\pgfqpoint{1.500000in}{0.880000in}}{\pgfqpoint{9.300000in}{6.160000in}}%
\pgfusepath{clip}%
\pgfsetbuttcap%
\pgfsetroundjoin%
\pgfsetlinewidth{0.602250pt}%
\definecolor{currentstroke}{rgb}{0.800000,0.800000,0.800000}%
\pgfsetstrokecolor{currentstroke}%
\pgfsetdash{{2.220000pt}{0.960000pt}}{0.000000pt}%
\pgfpathmoveto{\pgfqpoint{1.500000in}{2.420000in}}%
\pgfpathlineto{\pgfqpoint{10.800000in}{2.420000in}}%
\pgfusepath{stroke}%
\end{pgfscope}%
\begin{pgfscope}%
\definecolor{textcolor}{rgb}{0.150000,0.150000,0.150000}%
\pgfsetstrokecolor{textcolor}%
\pgfsetfillcolor{textcolor}%
\pgftext[x=1.140266in, y=2.370290in, left, base]{\color{textcolor}{\sffamily\fontsize{10.000000}{12.000000}\selectfont\catcode`\^=\active\def^{\ifmmode\sp\else\^{}\fi}\catcode`\%=\active\def%{\%}\ensuremath{-}10}}%
\end{pgfscope}%
\begin{pgfscope}%
\pgfpathrectangle{\pgfqpoint{1.500000in}{0.880000in}}{\pgfqpoint{9.300000in}{6.160000in}}%
\pgfusepath{clip}%
\pgfsetbuttcap%
\pgfsetroundjoin%
\pgfsetlinewidth{0.602250pt}%
\definecolor{currentstroke}{rgb}{0.800000,0.800000,0.800000}%
\pgfsetstrokecolor{currentstroke}%
\pgfsetdash{{2.220000pt}{0.960000pt}}{0.000000pt}%
\pgfpathmoveto{\pgfqpoint{1.500000in}{3.190000in}}%
\pgfpathlineto{\pgfqpoint{10.800000in}{3.190000in}}%
\pgfusepath{stroke}%
\end{pgfscope}%
\begin{pgfscope}%
\definecolor{textcolor}{rgb}{0.150000,0.150000,0.150000}%
\pgfsetstrokecolor{textcolor}%
\pgfsetfillcolor{textcolor}%
\pgftext[x=1.217509in, y=3.140290in, left, base]{\color{textcolor}{\sffamily\fontsize{10.000000}{12.000000}\selectfont\catcode`\^=\active\def^{\ifmmode\sp\else\^{}\fi}\catcode`\%=\active\def%{\%}\ensuremath{-}5}}%
\end{pgfscope}%
\begin{pgfscope}%
\pgfpathrectangle{\pgfqpoint{1.500000in}{0.880000in}}{\pgfqpoint{9.300000in}{6.160000in}}%
\pgfusepath{clip}%
\pgfsetbuttcap%
\pgfsetroundjoin%
\pgfsetlinewidth{0.602250pt}%
\definecolor{currentstroke}{rgb}{0.800000,0.800000,0.800000}%
\pgfsetstrokecolor{currentstroke}%
\pgfsetdash{{2.220000pt}{0.960000pt}}{0.000000pt}%
\pgfpathmoveto{\pgfqpoint{1.500000in}{3.960000in}}%
\pgfpathlineto{\pgfqpoint{10.800000in}{3.960000in}}%
\pgfusepath{stroke}%
\end{pgfscope}%
\begin{pgfscope}%
\definecolor{textcolor}{rgb}{0.150000,0.150000,0.150000}%
\pgfsetstrokecolor{textcolor}%
\pgfsetfillcolor{textcolor}%
\pgftext[x=1.325534in, y=3.910290in, left, base]{\color{textcolor}{\sffamily\fontsize{10.000000}{12.000000}\selectfont\catcode`\^=\active\def^{\ifmmode\sp\else\^{}\fi}\catcode`\%=\active\def%{\%}0}}%
\end{pgfscope}%
\begin{pgfscope}%
\pgfpathrectangle{\pgfqpoint{1.500000in}{0.880000in}}{\pgfqpoint{9.300000in}{6.160000in}}%
\pgfusepath{clip}%
\pgfsetbuttcap%
\pgfsetroundjoin%
\pgfsetlinewidth{0.602250pt}%
\definecolor{currentstroke}{rgb}{0.800000,0.800000,0.800000}%
\pgfsetstrokecolor{currentstroke}%
\pgfsetdash{{2.220000pt}{0.960000pt}}{0.000000pt}%
\pgfpathmoveto{\pgfqpoint{1.500000in}{4.730000in}}%
\pgfpathlineto{\pgfqpoint{10.800000in}{4.730000in}}%
\pgfusepath{stroke}%
\end{pgfscope}%
\begin{pgfscope}%
\definecolor{textcolor}{rgb}{0.150000,0.150000,0.150000}%
\pgfsetstrokecolor{textcolor}%
\pgfsetfillcolor{textcolor}%
\pgftext[x=1.325534in, y=4.680290in, left, base]{\color{textcolor}{\sffamily\fontsize{10.000000}{12.000000}\selectfont\catcode`\^=\active\def^{\ifmmode\sp\else\^{}\fi}\catcode`\%=\active\def%{\%}5}}%
\end{pgfscope}%
\begin{pgfscope}%
\pgfpathrectangle{\pgfqpoint{1.500000in}{0.880000in}}{\pgfqpoint{9.300000in}{6.160000in}}%
\pgfusepath{clip}%
\pgfsetbuttcap%
\pgfsetroundjoin%
\pgfsetlinewidth{0.602250pt}%
\definecolor{currentstroke}{rgb}{0.800000,0.800000,0.800000}%
\pgfsetstrokecolor{currentstroke}%
\pgfsetdash{{2.220000pt}{0.960000pt}}{0.000000pt}%
\pgfpathmoveto{\pgfqpoint{1.500000in}{5.500000in}}%
\pgfpathlineto{\pgfqpoint{10.800000in}{5.500000in}}%
\pgfusepath{stroke}%
\end{pgfscope}%
\begin{pgfscope}%
\definecolor{textcolor}{rgb}{0.150000,0.150000,0.150000}%
\pgfsetstrokecolor{textcolor}%
\pgfsetfillcolor{textcolor}%
\pgftext[x=1.248291in, y=5.450290in, left, base]{\color{textcolor}{\sffamily\fontsize{10.000000}{12.000000}\selectfont\catcode`\^=\active\def^{\ifmmode\sp\else\^{}\fi}\catcode`\%=\active\def%{\%}10}}%
\end{pgfscope}%
\begin{pgfscope}%
\pgfpathrectangle{\pgfqpoint{1.500000in}{0.880000in}}{\pgfqpoint{9.300000in}{6.160000in}}%
\pgfusepath{clip}%
\pgfsetbuttcap%
\pgfsetroundjoin%
\pgfsetlinewidth{0.602250pt}%
\definecolor{currentstroke}{rgb}{0.800000,0.800000,0.800000}%
\pgfsetstrokecolor{currentstroke}%
\pgfsetdash{{2.220000pt}{0.960000pt}}{0.000000pt}%
\pgfpathmoveto{\pgfqpoint{1.500000in}{6.270000in}}%
\pgfpathlineto{\pgfqpoint{10.800000in}{6.270000in}}%
\pgfusepath{stroke}%
\end{pgfscope}%
\begin{pgfscope}%
\definecolor{textcolor}{rgb}{0.150000,0.150000,0.150000}%
\pgfsetstrokecolor{textcolor}%
\pgfsetfillcolor{textcolor}%
\pgftext[x=1.248291in, y=6.220290in, left, base]{\color{textcolor}{\sffamily\fontsize{10.000000}{12.000000}\selectfont\catcode`\^=\active\def^{\ifmmode\sp\else\^{}\fi}\catcode`\%=\active\def%{\%}15}}%
\end{pgfscope}%
\begin{pgfscope}%
\pgfpathrectangle{\pgfqpoint{1.500000in}{0.880000in}}{\pgfqpoint{9.300000in}{6.160000in}}%
\pgfusepath{clip}%
\pgfsetbuttcap%
\pgfsetroundjoin%
\pgfsetlinewidth{0.602250pt}%
\definecolor{currentstroke}{rgb}{0.800000,0.800000,0.800000}%
\pgfsetstrokecolor{currentstroke}%
\pgfsetdash{{2.220000pt}{0.960000pt}}{0.000000pt}%
\pgfpathmoveto{\pgfqpoint{1.500000in}{7.040000in}}%
\pgfpathlineto{\pgfqpoint{10.800000in}{7.040000in}}%
\pgfusepath{stroke}%
\end{pgfscope}%
\begin{pgfscope}%
\definecolor{textcolor}{rgb}{0.150000,0.150000,0.150000}%
\pgfsetstrokecolor{textcolor}%
\pgfsetfillcolor{textcolor}%
\pgftext[x=1.248291in, y=6.990290in, left, base]{\color{textcolor}{\sffamily\fontsize{10.000000}{12.000000}\selectfont\catcode`\^=\active\def^{\ifmmode\sp\else\^{}\fi}\catcode`\%=\active\def%{\%}20}}%
\end{pgfscope}%
\begin{pgfscope}%
\definecolor{textcolor}{rgb}{0.150000,0.150000,0.150000}%
\pgfsetstrokecolor{textcolor}%
\pgfsetfillcolor{textcolor}%
\pgftext[x=1.084710in,y=3.960000in,,bottom,rotate=90.000000]{\color{textcolor}{\sffamily\fontsize{14.000000}{16.800000}\selectfont\catcode`\^=\active\def^{\ifmmode\sp\else\^{}\fi}\catcode`\%=\active\def%{\%}$L_n(x)$}}%
\end{pgfscope}%
\begin{pgfscope}%
\pgfpathrectangle{\pgfqpoint{1.500000in}{0.880000in}}{\pgfqpoint{9.300000in}{6.160000in}}%
\pgfusepath{clip}%
\pgfsetroundcap%
\pgfsetroundjoin%
\pgfsetlinewidth{2.007500pt}%
\definecolor{currentstroke}{rgb}{0.000000,0.000000,1.000000}%
\pgfsetstrokecolor{currentstroke}%
\pgfsetdash{}{0pt}%
\pgfpathmoveto{\pgfqpoint{1.500000in}{4.114000in}}%
\pgfpathlineto{\pgfqpoint{10.805000in}{4.114000in}}%
\pgfpathlineto{\pgfqpoint{10.805000in}{4.114000in}}%
\pgfusepath{stroke}%
\end{pgfscope}%
\begin{pgfscope}%
\pgfpathrectangle{\pgfqpoint{1.500000in}{0.880000in}}{\pgfqpoint{9.300000in}{6.160000in}}%
\pgfusepath{clip}%
\pgfsetroundcap%
\pgfsetroundjoin%
\pgfsetlinewidth{2.007500pt}%
\definecolor{currentstroke}{rgb}{1.000000,0.000000,0.000000}%
\pgfsetstrokecolor{currentstroke}%
\pgfsetdash{}{0pt}%
\pgfpathmoveto{\pgfqpoint{1.500000in}{4.884000in}}%
\pgfpathlineto{\pgfqpoint{10.805000in}{2.572758in}}%
\pgfpathlineto{\pgfqpoint{10.805000in}{2.572758in}}%
\pgfusepath{stroke}%
\end{pgfscope}%
\begin{pgfscope}%
\pgfpathrectangle{\pgfqpoint{1.500000in}{0.880000in}}{\pgfqpoint{9.300000in}{6.160000in}}%
\pgfusepath{clip}%
\pgfsetroundcap%
\pgfsetroundjoin%
\pgfsetlinewidth{2.007500pt}%
\definecolor{currentstroke}{rgb}{0.000000,0.501961,0.000000}%
\pgfsetstrokecolor{currentstroke}%
\pgfsetdash{}{0pt}%
\pgfpathmoveto{\pgfqpoint{1.818845in}{7.045000in}}%
\pgfpathlineto{\pgfqpoint{1.888258in}{6.934127in}}%
\pgfpathlineto{\pgfqpoint{1.961057in}{6.819937in}}%
\pgfpathlineto{\pgfqpoint{2.033855in}{6.707870in}}%
\pgfpathlineto{\pgfqpoint{2.106654in}{6.597926in}}%
\pgfpathlineto{\pgfqpoint{2.179452in}{6.490105in}}%
\pgfpathlineto{\pgfqpoint{2.252250in}{6.384408in}}%
\pgfpathlineto{\pgfqpoint{2.325049in}{6.280833in}}%
\pgfpathlineto{\pgfqpoint{2.397847in}{6.179382in}}%
\pgfpathlineto{\pgfqpoint{2.470646in}{6.080054in}}%
\pgfpathlineto{\pgfqpoint{2.543444in}{5.982848in}}%
\pgfpathlineto{\pgfqpoint{2.616243in}{5.887767in}}%
\pgfpathlineto{\pgfqpoint{2.689041in}{5.794808in}}%
\pgfpathlineto{\pgfqpoint{2.761840in}{5.703972in}}%
\pgfpathlineto{\pgfqpoint{2.834638in}{5.615260in}}%
\pgfpathlineto{\pgfqpoint{2.907436in}{5.528671in}}%
\pgfpathlineto{\pgfqpoint{2.980235in}{5.444204in}}%
\pgfpathlineto{\pgfqpoint{3.053033in}{5.361861in}}%
\pgfpathlineto{\pgfqpoint{3.125832in}{5.281642in}}%
\pgfpathlineto{\pgfqpoint{3.198630in}{5.203545in}}%
\pgfpathlineto{\pgfqpoint{3.271429in}{5.127571in}}%
\pgfpathlineto{\pgfqpoint{3.344227in}{5.053721in}}%
\pgfpathlineto{\pgfqpoint{3.417025in}{4.981994in}}%
\pgfpathlineto{\pgfqpoint{3.489824in}{4.912390in}}%
\pgfpathlineto{\pgfqpoint{3.562622in}{4.844909in}}%
\pgfpathlineto{\pgfqpoint{3.635421in}{4.779551in}}%
\pgfpathlineto{\pgfqpoint{3.708219in}{4.716317in}}%
\pgfpathlineto{\pgfqpoint{3.781018in}{4.655205in}}%
\pgfpathlineto{\pgfqpoint{3.853816in}{4.596217in}}%
\pgfpathlineto{\pgfqpoint{3.926614in}{4.539352in}}%
\pgfpathlineto{\pgfqpoint{3.975147in}{4.502621in}}%
\pgfpathlineto{\pgfqpoint{4.023679in}{4.466834in}}%
\pgfpathlineto{\pgfqpoint{4.072211in}{4.431991in}}%
\pgfpathlineto{\pgfqpoint{4.120744in}{4.398091in}}%
\pgfpathlineto{\pgfqpoint{4.169276in}{4.365135in}}%
\pgfpathlineto{\pgfqpoint{4.217808in}{4.333123in}}%
\pgfpathlineto{\pgfqpoint{4.266341in}{4.302054in}}%
\pgfpathlineto{\pgfqpoint{4.314873in}{4.271929in}}%
\pgfpathlineto{\pgfqpoint{4.363405in}{4.242747in}}%
\pgfpathlineto{\pgfqpoint{4.411937in}{4.214509in}}%
\pgfpathlineto{\pgfqpoint{4.460470in}{4.187215in}}%
\pgfpathlineto{\pgfqpoint{4.509002in}{4.160864in}}%
\pgfpathlineto{\pgfqpoint{4.557534in}{4.135457in}}%
\pgfpathlineto{\pgfqpoint{4.606067in}{4.110994in}}%
\pgfpathlineto{\pgfqpoint{4.654599in}{4.087474in}}%
\pgfpathlineto{\pgfqpoint{4.703131in}{4.064898in}}%
\pgfpathlineto{\pgfqpoint{4.751663in}{4.043265in}}%
\pgfpathlineto{\pgfqpoint{4.800196in}{4.022576in}}%
\pgfpathlineto{\pgfqpoint{4.848728in}{4.002831in}}%
\pgfpathlineto{\pgfqpoint{4.897260in}{3.984029in}}%
\pgfpathlineto{\pgfqpoint{4.945793in}{3.966171in}}%
\pgfpathlineto{\pgfqpoint{4.994325in}{3.949257in}}%
\pgfpathlineto{\pgfqpoint{5.042857in}{3.933286in}}%
\pgfpathlineto{\pgfqpoint{5.091389in}{3.918259in}}%
\pgfpathlineto{\pgfqpoint{5.139922in}{3.904175in}}%
\pgfpathlineto{\pgfqpoint{5.188454in}{3.891035in}}%
\pgfpathlineto{\pgfqpoint{5.236986in}{3.878839in}}%
\pgfpathlineto{\pgfqpoint{5.285519in}{3.867586in}}%
\pgfpathlineto{\pgfqpoint{5.334051in}{3.857277in}}%
\pgfpathlineto{\pgfqpoint{5.382583in}{3.847911in}}%
\pgfpathlineto{\pgfqpoint{5.431115in}{3.839490in}}%
\pgfpathlineto{\pgfqpoint{5.479648in}{3.832011in}}%
\pgfpathlineto{\pgfqpoint{5.528180in}{3.825477in}}%
\pgfpathlineto{\pgfqpoint{5.576712in}{3.819886in}}%
\pgfpathlineto{\pgfqpoint{5.625245in}{3.815238in}}%
\pgfpathlineto{\pgfqpoint{5.673777in}{3.811535in}}%
\pgfpathlineto{\pgfqpoint{5.722309in}{3.808775in}}%
\pgfpathlineto{\pgfqpoint{5.770841in}{3.806958in}}%
\pgfpathlineto{\pgfqpoint{5.819374in}{3.806085in}}%
\pgfpathlineto{\pgfqpoint{5.867906in}{3.806156in}}%
\pgfpathlineto{\pgfqpoint{5.916438in}{3.807170in}}%
\pgfpathlineto{\pgfqpoint{5.964971in}{3.809128in}}%
\pgfpathlineto{\pgfqpoint{6.013503in}{3.812030in}}%
\pgfpathlineto{\pgfqpoint{6.062035in}{3.815875in}}%
\pgfpathlineto{\pgfqpoint{6.110568in}{3.820664in}}%
\pgfpathlineto{\pgfqpoint{6.159100in}{3.826397in}}%
\pgfpathlineto{\pgfqpoint{6.207632in}{3.833073in}}%
\pgfpathlineto{\pgfqpoint{6.256164in}{3.840693in}}%
\pgfpathlineto{\pgfqpoint{6.304697in}{3.849256in}}%
\pgfpathlineto{\pgfqpoint{6.353229in}{3.858763in}}%
\pgfpathlineto{\pgfqpoint{6.401761in}{3.869214in}}%
\pgfpathlineto{\pgfqpoint{6.450294in}{3.880608in}}%
\pgfpathlineto{\pgfqpoint{6.498826in}{3.892946in}}%
\pgfpathlineto{\pgfqpoint{6.547358in}{3.906227in}}%
\pgfpathlineto{\pgfqpoint{6.595890in}{3.920452in}}%
\pgfpathlineto{\pgfqpoint{6.644423in}{3.935621in}}%
\pgfpathlineto{\pgfqpoint{6.692955in}{3.951734in}}%
\pgfpathlineto{\pgfqpoint{6.741487in}{3.968790in}}%
\pgfpathlineto{\pgfqpoint{6.790020in}{3.986789in}}%
\pgfpathlineto{\pgfqpoint{6.838552in}{4.005732in}}%
\pgfpathlineto{\pgfqpoint{6.887084in}{4.025619in}}%
\pgfpathlineto{\pgfqpoint{6.935616in}{4.046450in}}%
\pgfpathlineto{\pgfqpoint{6.984149in}{4.068224in}}%
\pgfpathlineto{\pgfqpoint{7.032681in}{4.090942in}}%
\pgfpathlineto{\pgfqpoint{7.081213in}{4.114603in}}%
\pgfpathlineto{\pgfqpoint{7.129746in}{4.139208in}}%
\pgfpathlineto{\pgfqpoint{7.178278in}{4.164757in}}%
\pgfpathlineto{\pgfqpoint{7.226810in}{4.191249in}}%
\pgfpathlineto{\pgfqpoint{7.275342in}{4.218685in}}%
\pgfpathlineto{\pgfqpoint{7.323875in}{4.247064in}}%
\pgfpathlineto{\pgfqpoint{7.372407in}{4.276387in}}%
\pgfpathlineto{\pgfqpoint{7.420939in}{4.306654in}}%
\pgfpathlineto{\pgfqpoint{7.469472in}{4.337864in}}%
\pgfpathlineto{\pgfqpoint{7.518004in}{4.370018in}}%
\pgfpathlineto{\pgfqpoint{7.566536in}{4.403116in}}%
\pgfpathlineto{\pgfqpoint{7.615068in}{4.437157in}}%
\pgfpathlineto{\pgfqpoint{7.663601in}{4.472142in}}%
\pgfpathlineto{\pgfqpoint{7.712133in}{4.508071in}}%
\pgfpathlineto{\pgfqpoint{7.760665in}{4.544943in}}%
\pgfpathlineto{\pgfqpoint{7.809198in}{4.582758in}}%
\pgfpathlineto{\pgfqpoint{7.881996in}{4.641251in}}%
\pgfpathlineto{\pgfqpoint{7.954795in}{4.701867in}}%
\pgfpathlineto{\pgfqpoint{8.027593in}{4.764607in}}%
\pgfpathlineto{\pgfqpoint{8.100391in}{4.829469in}}%
\pgfpathlineto{\pgfqpoint{8.173190in}{4.896454in}}%
\pgfpathlineto{\pgfqpoint{8.245988in}{4.965563in}}%
\pgfpathlineto{\pgfqpoint{8.318787in}{5.036795in}}%
\pgfpathlineto{\pgfqpoint{8.391585in}{5.110150in}}%
\pgfpathlineto{\pgfqpoint{8.464384in}{5.185628in}}%
\pgfpathlineto{\pgfqpoint{8.537182in}{5.263229in}}%
\pgfpathlineto{\pgfqpoint{8.609980in}{5.342954in}}%
\pgfpathlineto{\pgfqpoint{8.682779in}{5.424801in}}%
\pgfpathlineto{\pgfqpoint{8.755577in}{5.508772in}}%
\pgfpathlineto{\pgfqpoint{8.828376in}{5.594866in}}%
\pgfpathlineto{\pgfqpoint{8.901174in}{5.683083in}}%
\pgfpathlineto{\pgfqpoint{8.973973in}{5.773423in}}%
\pgfpathlineto{\pgfqpoint{9.046771in}{5.865886in}}%
\pgfpathlineto{\pgfqpoint{9.119569in}{5.960473in}}%
\pgfpathlineto{\pgfqpoint{9.192368in}{6.057182in}}%
\pgfpathlineto{\pgfqpoint{9.265166in}{6.156015in}}%
\pgfpathlineto{\pgfqpoint{9.337965in}{6.256971in}}%
\pgfpathlineto{\pgfqpoint{9.410763in}{6.360050in}}%
\pgfpathlineto{\pgfqpoint{9.483562in}{6.465253in}}%
\pgfpathlineto{\pgfqpoint{9.556360in}{6.572578in}}%
\pgfpathlineto{\pgfqpoint{9.629159in}{6.682027in}}%
\pgfpathlineto{\pgfqpoint{9.701957in}{6.793598in}}%
\pgfpathlineto{\pgfqpoint{9.774755in}{6.907293in}}%
\pgfpathlineto{\pgfqpoint{9.847554in}{7.023111in}}%
\pgfpathlineto{\pgfqpoint{9.861146in}{7.045000in}}%
\pgfpathlineto{\pgfqpoint{9.861146in}{7.045000in}}%
\pgfusepath{stroke}%
\end{pgfscope}%
\begin{pgfscope}%
\pgfpathrectangle{\pgfqpoint{1.500000in}{0.880000in}}{\pgfqpoint{9.300000in}{6.160000in}}%
\pgfusepath{clip}%
\pgfsetroundcap%
\pgfsetroundjoin%
\pgfsetlinewidth{2.007500pt}%
\definecolor{currentstroke}{rgb}{0.000000,0.000000,0.000000}%
\pgfsetstrokecolor{currentstroke}%
\pgfsetdash{}{0pt}%
\pgfpathmoveto{\pgfqpoint{3.070502in}{7.045000in}}%
\pgfpathlineto{\pgfqpoint{3.101566in}{6.942194in}}%
\pgfpathlineto{\pgfqpoint{3.150098in}{6.785964in}}%
\pgfpathlineto{\pgfqpoint{3.198630in}{6.634771in}}%
\pgfpathlineto{\pgfqpoint{3.247162in}{6.488542in}}%
\pgfpathlineto{\pgfqpoint{3.295695in}{6.347202in}}%
\pgfpathlineto{\pgfqpoint{3.344227in}{6.210679in}}%
\pgfpathlineto{\pgfqpoint{3.392759in}{6.078898in}}%
\pgfpathlineto{\pgfqpoint{3.441292in}{5.951785in}}%
\pgfpathlineto{\pgfqpoint{3.489824in}{5.829266in}}%
\pgfpathlineto{\pgfqpoint{3.538356in}{5.711268in}}%
\pgfpathlineto{\pgfqpoint{3.586888in}{5.597716in}}%
\pgfpathlineto{\pgfqpoint{3.635421in}{5.488538in}}%
\pgfpathlineto{\pgfqpoint{3.683953in}{5.383658in}}%
\pgfpathlineto{\pgfqpoint{3.732485in}{5.283003in}}%
\pgfpathlineto{\pgfqpoint{3.781018in}{5.186500in}}%
\pgfpathlineto{\pgfqpoint{3.829550in}{5.094074in}}%
\pgfpathlineto{\pgfqpoint{3.878082in}{5.005651in}}%
\pgfpathlineto{\pgfqpoint{3.926614in}{4.921158in}}%
\pgfpathlineto{\pgfqpoint{3.975147in}{4.840521in}}%
\pgfpathlineto{\pgfqpoint{4.023679in}{4.763666in}}%
\pgfpathlineto{\pgfqpoint{4.072211in}{4.690518in}}%
\pgfpathlineto{\pgfqpoint{4.120744in}{4.621005in}}%
\pgfpathlineto{\pgfqpoint{4.169276in}{4.555052in}}%
\pgfpathlineto{\pgfqpoint{4.217808in}{4.492586in}}%
\pgfpathlineto{\pgfqpoint{4.266341in}{4.433532in}}%
\pgfpathlineto{\pgfqpoint{4.314873in}{4.377817in}}%
\pgfpathlineto{\pgfqpoint{4.363405in}{4.325366in}}%
\pgfpathlineto{\pgfqpoint{4.411937in}{4.276107in}}%
\pgfpathlineto{\pgfqpoint{4.460470in}{4.229965in}}%
\pgfpathlineto{\pgfqpoint{4.484736in}{4.208039in}}%
\pgfpathlineto{\pgfqpoint{4.509002in}{4.186866in}}%
\pgfpathlineto{\pgfqpoint{4.533268in}{4.166434in}}%
\pgfpathlineto{\pgfqpoint{4.557534in}{4.146736in}}%
\pgfpathlineto{\pgfqpoint{4.581800in}{4.127761in}}%
\pgfpathlineto{\pgfqpoint{4.606067in}{4.109502in}}%
\pgfpathlineto{\pgfqpoint{4.630333in}{4.091947in}}%
\pgfpathlineto{\pgfqpoint{4.654599in}{4.075089in}}%
\pgfpathlineto{\pgfqpoint{4.678865in}{4.058918in}}%
\pgfpathlineto{\pgfqpoint{4.703131in}{4.043424in}}%
\pgfpathlineto{\pgfqpoint{4.727397in}{4.028599in}}%
\pgfpathlineto{\pgfqpoint{4.751663in}{4.014433in}}%
\pgfpathlineto{\pgfqpoint{4.775930in}{4.000917in}}%
\pgfpathlineto{\pgfqpoint{4.800196in}{3.988042in}}%
\pgfpathlineto{\pgfqpoint{4.824462in}{3.975799in}}%
\pgfpathlineto{\pgfqpoint{4.872994in}{3.953170in}}%
\pgfpathlineto{\pgfqpoint{4.921526in}{3.932956in}}%
\pgfpathlineto{\pgfqpoint{4.970059in}{3.915083in}}%
\pgfpathlineto{\pgfqpoint{5.018591in}{3.899478in}}%
\pgfpathlineto{\pgfqpoint{5.067123in}{3.886068in}}%
\pgfpathlineto{\pgfqpoint{5.115656in}{3.874776in}}%
\pgfpathlineto{\pgfqpoint{5.164188in}{3.865532in}}%
\pgfpathlineto{\pgfqpoint{5.212720in}{3.858259in}}%
\pgfpathlineto{\pgfqpoint{5.261252in}{3.852884in}}%
\pgfpathlineto{\pgfqpoint{5.309785in}{3.849334in}}%
\pgfpathlineto{\pgfqpoint{5.358317in}{3.847535in}}%
\pgfpathlineto{\pgfqpoint{5.406849in}{3.847412in}}%
\pgfpathlineto{\pgfqpoint{5.455382in}{3.848892in}}%
\pgfpathlineto{\pgfqpoint{5.503914in}{3.851902in}}%
\pgfpathlineto{\pgfqpoint{5.552446in}{3.856366in}}%
\pgfpathlineto{\pgfqpoint{5.600978in}{3.862212in}}%
\pgfpathlineto{\pgfqpoint{5.649511in}{3.869365in}}%
\pgfpathlineto{\pgfqpoint{5.698043in}{3.877751in}}%
\pgfpathlineto{\pgfqpoint{5.746575in}{3.887297in}}%
\pgfpathlineto{\pgfqpoint{5.795108in}{3.897929in}}%
\pgfpathlineto{\pgfqpoint{5.843640in}{3.909573in}}%
\pgfpathlineto{\pgfqpoint{5.892172in}{3.922156in}}%
\pgfpathlineto{\pgfqpoint{5.940705in}{3.935602in}}%
\pgfpathlineto{\pgfqpoint{6.013503in}{3.957230in}}%
\pgfpathlineto{\pgfqpoint{6.086301in}{3.980387in}}%
\pgfpathlineto{\pgfqpoint{6.159100in}{4.004824in}}%
\pgfpathlineto{\pgfqpoint{6.231898in}{4.030292in}}%
\pgfpathlineto{\pgfqpoint{6.328963in}{4.065420in}}%
\pgfpathlineto{\pgfqpoint{6.498826in}{4.128459in}}%
\pgfpathlineto{\pgfqpoint{6.644423in}{4.182037in}}%
\pgfpathlineto{\pgfqpoint{6.741487in}{4.216475in}}%
\pgfpathlineto{\pgfqpoint{6.814286in}{4.241211in}}%
\pgfpathlineto{\pgfqpoint{6.887084in}{4.264734in}}%
\pgfpathlineto{\pgfqpoint{6.959883in}{4.286794in}}%
\pgfpathlineto{\pgfqpoint{7.008415in}{4.300566in}}%
\pgfpathlineto{\pgfqpoint{7.056947in}{4.313502in}}%
\pgfpathlineto{\pgfqpoint{7.105479in}{4.325530in}}%
\pgfpathlineto{\pgfqpoint{7.154012in}{4.336576in}}%
\pgfpathlineto{\pgfqpoint{7.202544in}{4.346565in}}%
\pgfpathlineto{\pgfqpoint{7.251076in}{4.355425in}}%
\pgfpathlineto{\pgfqpoint{7.299609in}{4.363080in}}%
\pgfpathlineto{\pgfqpoint{7.348141in}{4.369457in}}%
\pgfpathlineto{\pgfqpoint{7.396673in}{4.374483in}}%
\pgfpathlineto{\pgfqpoint{7.445205in}{4.378083in}}%
\pgfpathlineto{\pgfqpoint{7.493738in}{4.380184in}}%
\pgfpathlineto{\pgfqpoint{7.542270in}{4.380711in}}%
\pgfpathlineto{\pgfqpoint{7.590802in}{4.379591in}}%
\pgfpathlineto{\pgfqpoint{7.639335in}{4.376751in}}%
\pgfpathlineto{\pgfqpoint{7.687867in}{4.372115in}}%
\pgfpathlineto{\pgfqpoint{7.736399in}{4.365610in}}%
\pgfpathlineto{\pgfqpoint{7.784932in}{4.357163in}}%
\pgfpathlineto{\pgfqpoint{7.833464in}{4.346699in}}%
\pgfpathlineto{\pgfqpoint{7.881996in}{4.334145in}}%
\pgfpathlineto{\pgfqpoint{7.930528in}{4.319427in}}%
\pgfpathlineto{\pgfqpoint{7.979061in}{4.302470in}}%
\pgfpathlineto{\pgfqpoint{8.027593in}{4.283202in}}%
\pgfpathlineto{\pgfqpoint{8.051859in}{4.272677in}}%
\pgfpathlineto{\pgfqpoint{8.076125in}{4.261547in}}%
\pgfpathlineto{\pgfqpoint{8.100391in}{4.249803in}}%
\pgfpathlineto{\pgfqpoint{8.124658in}{4.237433in}}%
\pgfpathlineto{\pgfqpoint{8.148924in}{4.224431in}}%
\pgfpathlineto{\pgfqpoint{8.173190in}{4.210786in}}%
\pgfpathlineto{\pgfqpoint{8.197456in}{4.196489in}}%
\pgfpathlineto{\pgfqpoint{8.221722in}{4.181531in}}%
\pgfpathlineto{\pgfqpoint{8.245988in}{4.165903in}}%
\pgfpathlineto{\pgfqpoint{8.270254in}{4.149595in}}%
\pgfpathlineto{\pgfqpoint{8.294521in}{4.132598in}}%
\pgfpathlineto{\pgfqpoint{8.318787in}{4.114903in}}%
\pgfpathlineto{\pgfqpoint{8.343053in}{4.096501in}}%
\pgfpathlineto{\pgfqpoint{8.367319in}{4.077383in}}%
\pgfpathlineto{\pgfqpoint{8.391585in}{4.057539in}}%
\pgfpathlineto{\pgfqpoint{8.415851in}{4.036959in}}%
\pgfpathlineto{\pgfqpoint{8.440117in}{4.015636in}}%
\pgfpathlineto{\pgfqpoint{8.464384in}{3.993559in}}%
\pgfpathlineto{\pgfqpoint{8.488650in}{3.970720in}}%
\pgfpathlineto{\pgfqpoint{8.512916in}{3.947109in}}%
\pgfpathlineto{\pgfqpoint{8.537182in}{3.922716in}}%
\pgfpathlineto{\pgfqpoint{8.585714in}{3.871551in}}%
\pgfpathlineto{\pgfqpoint{8.634247in}{3.817151in}}%
\pgfpathlineto{\pgfqpoint{8.682779in}{3.759441in}}%
\pgfpathlineto{\pgfqpoint{8.731311in}{3.698348in}}%
\pgfpathlineto{\pgfqpoint{8.779843in}{3.633799in}}%
\pgfpathlineto{\pgfqpoint{8.828376in}{3.565719in}}%
\pgfpathlineto{\pgfqpoint{8.876908in}{3.494034in}}%
\pgfpathlineto{\pgfqpoint{8.925440in}{3.418671in}}%
\pgfpathlineto{\pgfqpoint{8.973973in}{3.339555in}}%
\pgfpathlineto{\pgfqpoint{9.022505in}{3.256613in}}%
\pgfpathlineto{\pgfqpoint{9.071037in}{3.169771in}}%
\pgfpathlineto{\pgfqpoint{9.119569in}{3.078956in}}%
\pgfpathlineto{\pgfqpoint{9.168102in}{2.984092in}}%
\pgfpathlineto{\pgfqpoint{9.216634in}{2.885107in}}%
\pgfpathlineto{\pgfqpoint{9.265166in}{2.781926in}}%
\pgfpathlineto{\pgfqpoint{9.313699in}{2.674476in}}%
\pgfpathlineto{\pgfqpoint{9.362231in}{2.562682in}}%
\pgfpathlineto{\pgfqpoint{9.410763in}{2.446472in}}%
\pgfpathlineto{\pgfqpoint{9.459295in}{2.325770in}}%
\pgfpathlineto{\pgfqpoint{9.507828in}{2.200504in}}%
\pgfpathlineto{\pgfqpoint{9.556360in}{2.070599in}}%
\pgfpathlineto{\pgfqpoint{9.604892in}{1.935981in}}%
\pgfpathlineto{\pgfqpoint{9.653425in}{1.796577in}}%
\pgfpathlineto{\pgfqpoint{9.701957in}{1.652312in}}%
\pgfpathlineto{\pgfqpoint{9.750489in}{1.503114in}}%
\pgfpathlineto{\pgfqpoint{9.799022in}{1.348907in}}%
\pgfpathlineto{\pgfqpoint{9.847554in}{1.189619in}}%
\pgfpathlineto{\pgfqpoint{9.896086in}{1.025174in}}%
\pgfpathlineto{\pgfqpoint{9.939084in}{0.875000in}}%
\pgfpathlineto{\pgfqpoint{9.939084in}{0.875000in}}%
\pgfusepath{stroke}%
\end{pgfscope}%
\begin{pgfscope}%
\pgfpathrectangle{\pgfqpoint{1.500000in}{0.880000in}}{\pgfqpoint{9.300000in}{6.160000in}}%
\pgfusepath{clip}%
\pgfsetroundcap%
\pgfsetroundjoin%
\pgfsetlinewidth{2.007500pt}%
\definecolor{currentstroke}{rgb}{0.501961,0.000000,0.501961}%
\pgfsetstrokecolor{currentstroke}%
\pgfsetdash{}{0pt}%
\pgfpathmoveto{\pgfqpoint{3.530642in}{7.045000in}}%
\pgfpathlineto{\pgfqpoint{3.538356in}{7.006650in}}%
\pgfpathlineto{\pgfqpoint{3.562622in}{6.888938in}}%
\pgfpathlineto{\pgfqpoint{3.586888in}{6.774122in}}%
\pgfpathlineto{\pgfqpoint{3.611155in}{6.662166in}}%
\pgfpathlineto{\pgfqpoint{3.635421in}{6.553037in}}%
\pgfpathlineto{\pgfqpoint{3.659687in}{6.446700in}}%
\pgfpathlineto{\pgfqpoint{3.683953in}{6.343120in}}%
\pgfpathlineto{\pgfqpoint{3.708219in}{6.242262in}}%
\pgfpathlineto{\pgfqpoint{3.732485in}{6.144093in}}%
\pgfpathlineto{\pgfqpoint{3.756751in}{6.048576in}}%
\pgfpathlineto{\pgfqpoint{3.781018in}{5.955678in}}%
\pgfpathlineto{\pgfqpoint{3.805284in}{5.865365in}}%
\pgfpathlineto{\pgfqpoint{3.829550in}{5.777600in}}%
\pgfpathlineto{\pgfqpoint{3.853816in}{5.692350in}}%
\pgfpathlineto{\pgfqpoint{3.878082in}{5.609581in}}%
\pgfpathlineto{\pgfqpoint{3.902348in}{5.529257in}}%
\pgfpathlineto{\pgfqpoint{3.926614in}{5.451343in}}%
\pgfpathlineto{\pgfqpoint{3.950881in}{5.375806in}}%
\pgfpathlineto{\pgfqpoint{3.975147in}{5.302611in}}%
\pgfpathlineto{\pgfqpoint{3.999413in}{5.231723in}}%
\pgfpathlineto{\pgfqpoint{4.023679in}{5.163107in}}%
\pgfpathlineto{\pgfqpoint{4.047945in}{5.096729in}}%
\pgfpathlineto{\pgfqpoint{4.072211in}{5.032554in}}%
\pgfpathlineto{\pgfqpoint{4.096477in}{4.970547in}}%
\pgfpathlineto{\pgfqpoint{4.120744in}{4.910674in}}%
\pgfpathlineto{\pgfqpoint{4.145010in}{4.852901in}}%
\pgfpathlineto{\pgfqpoint{4.169276in}{4.797192in}}%
\pgfpathlineto{\pgfqpoint{4.193542in}{4.743514in}}%
\pgfpathlineto{\pgfqpoint{4.217808in}{4.691830in}}%
\pgfpathlineto{\pgfqpoint{4.242074in}{4.642108in}}%
\pgfpathlineto{\pgfqpoint{4.266341in}{4.594311in}}%
\pgfpathlineto{\pgfqpoint{4.290607in}{4.548406in}}%
\pgfpathlineto{\pgfqpoint{4.314873in}{4.504358in}}%
\pgfpathlineto{\pgfqpoint{4.339139in}{4.462133in}}%
\pgfpathlineto{\pgfqpoint{4.363405in}{4.421695in}}%
\pgfpathlineto{\pgfqpoint{4.387671in}{4.383009in}}%
\pgfpathlineto{\pgfqpoint{4.411937in}{4.346043in}}%
\pgfpathlineto{\pgfqpoint{4.436204in}{4.310760in}}%
\pgfpathlineto{\pgfqpoint{4.460470in}{4.277126in}}%
\pgfpathlineto{\pgfqpoint{4.484736in}{4.245107in}}%
\pgfpathlineto{\pgfqpoint{4.509002in}{4.214668in}}%
\pgfpathlineto{\pgfqpoint{4.533268in}{4.185773in}}%
\pgfpathlineto{\pgfqpoint{4.557534in}{4.158390in}}%
\pgfpathlineto{\pgfqpoint{4.581800in}{4.132483in}}%
\pgfpathlineto{\pgfqpoint{4.606067in}{4.108017in}}%
\pgfpathlineto{\pgfqpoint{4.630333in}{4.084958in}}%
\pgfpathlineto{\pgfqpoint{4.654599in}{4.063271in}}%
\pgfpathlineto{\pgfqpoint{4.678865in}{4.042921in}}%
\pgfpathlineto{\pgfqpoint{4.703131in}{4.023874in}}%
\pgfpathlineto{\pgfqpoint{4.727397in}{4.006096in}}%
\pgfpathlineto{\pgfqpoint{4.751663in}{3.989551in}}%
\pgfpathlineto{\pgfqpoint{4.775930in}{3.974206in}}%
\pgfpathlineto{\pgfqpoint{4.800196in}{3.960025in}}%
\pgfpathlineto{\pgfqpoint{4.824462in}{3.946973in}}%
\pgfpathlineto{\pgfqpoint{4.848728in}{3.935017in}}%
\pgfpathlineto{\pgfqpoint{4.872994in}{3.924121in}}%
\pgfpathlineto{\pgfqpoint{4.897260in}{3.914251in}}%
\pgfpathlineto{\pgfqpoint{4.921526in}{3.905373in}}%
\pgfpathlineto{\pgfqpoint{4.945793in}{3.897451in}}%
\pgfpathlineto{\pgfqpoint{4.970059in}{3.890451in}}%
\pgfpathlineto{\pgfqpoint{4.994325in}{3.884339in}}%
\pgfpathlineto{\pgfqpoint{5.018591in}{3.879079in}}%
\pgfpathlineto{\pgfqpoint{5.042857in}{3.874638in}}%
\pgfpathlineto{\pgfqpoint{5.067123in}{3.870980in}}%
\pgfpathlineto{\pgfqpoint{5.091389in}{3.868071in}}%
\pgfpathlineto{\pgfqpoint{5.115656in}{3.865876in}}%
\pgfpathlineto{\pgfqpoint{5.139922in}{3.864361in}}%
\pgfpathlineto{\pgfqpoint{5.164188in}{3.863491in}}%
\pgfpathlineto{\pgfqpoint{5.188454in}{3.863232in}}%
\pgfpathlineto{\pgfqpoint{5.236986in}{3.864405in}}%
\pgfpathlineto{\pgfqpoint{5.285519in}{3.867605in}}%
\pgfpathlineto{\pgfqpoint{5.334051in}{3.872555in}}%
\pgfpathlineto{\pgfqpoint{5.382583in}{3.878976in}}%
\pgfpathlineto{\pgfqpoint{5.431115in}{3.886592in}}%
\pgfpathlineto{\pgfqpoint{5.503914in}{3.899652in}}%
\pgfpathlineto{\pgfqpoint{5.625245in}{3.923471in}}%
\pgfpathlineto{\pgfqpoint{5.698043in}{3.937470in}}%
\pgfpathlineto{\pgfqpoint{5.746575in}{3.946104in}}%
\pgfpathlineto{\pgfqpoint{5.795108in}{3.953856in}}%
\pgfpathlineto{\pgfqpoint{5.843640in}{3.960448in}}%
\pgfpathlineto{\pgfqpoint{5.892172in}{3.965604in}}%
\pgfpathlineto{\pgfqpoint{5.940705in}{3.969047in}}%
\pgfpathlineto{\pgfqpoint{5.989237in}{3.970500in}}%
\pgfpathlineto{\pgfqpoint{6.013503in}{3.970394in}}%
\pgfpathlineto{\pgfqpoint{6.037769in}{3.969686in}}%
\pgfpathlineto{\pgfqpoint{6.062035in}{3.968342in}}%
\pgfpathlineto{\pgfqpoint{6.086301in}{3.966327in}}%
\pgfpathlineto{\pgfqpoint{6.110568in}{3.963607in}}%
\pgfpathlineto{\pgfqpoint{6.134834in}{3.960147in}}%
\pgfpathlineto{\pgfqpoint{6.159100in}{3.955913in}}%
\pgfpathlineto{\pgfqpoint{6.183366in}{3.950869in}}%
\pgfpathlineto{\pgfqpoint{6.207632in}{3.944982in}}%
\pgfpathlineto{\pgfqpoint{6.231898in}{3.938216in}}%
\pgfpathlineto{\pgfqpoint{6.256164in}{3.930537in}}%
\pgfpathlineto{\pgfqpoint{6.280431in}{3.921911in}}%
\pgfpathlineto{\pgfqpoint{6.304697in}{3.912302in}}%
\pgfpathlineto{\pgfqpoint{6.328963in}{3.901677in}}%
\pgfpathlineto{\pgfqpoint{6.353229in}{3.889999in}}%
\pgfpathlineto{\pgfqpoint{6.377495in}{3.877236in}}%
\pgfpathlineto{\pgfqpoint{6.401761in}{3.863352in}}%
\pgfpathlineto{\pgfqpoint{6.426027in}{3.848312in}}%
\pgfpathlineto{\pgfqpoint{6.450294in}{3.832083in}}%
\pgfpathlineto{\pgfqpoint{6.474560in}{3.814629in}}%
\pgfpathlineto{\pgfqpoint{6.498826in}{3.795915in}}%
\pgfpathlineto{\pgfqpoint{6.523092in}{3.775908in}}%
\pgfpathlineto{\pgfqpoint{6.547358in}{3.754572in}}%
\pgfpathlineto{\pgfqpoint{6.571624in}{3.731873in}}%
\pgfpathlineto{\pgfqpoint{6.595890in}{3.707776in}}%
\pgfpathlineto{\pgfqpoint{6.620157in}{3.682246in}}%
\pgfpathlineto{\pgfqpoint{6.644423in}{3.655250in}}%
\pgfpathlineto{\pgfqpoint{6.668689in}{3.626752in}}%
\pgfpathlineto{\pgfqpoint{6.692955in}{3.596718in}}%
\pgfpathlineto{\pgfqpoint{6.717221in}{3.565113in}}%
\pgfpathlineto{\pgfqpoint{6.741487in}{3.531902in}}%
\pgfpathlineto{\pgfqpoint{6.765753in}{3.497051in}}%
\pgfpathlineto{\pgfqpoint{6.790020in}{3.460526in}}%
\pgfpathlineto{\pgfqpoint{6.814286in}{3.422291in}}%
\pgfpathlineto{\pgfqpoint{6.838552in}{3.382312in}}%
\pgfpathlineto{\pgfqpoint{6.862818in}{3.340554in}}%
\pgfpathlineto{\pgfqpoint{6.887084in}{3.296983in}}%
\pgfpathlineto{\pgfqpoint{6.911350in}{3.251565in}}%
\pgfpathlineto{\pgfqpoint{6.935616in}{3.204263in}}%
\pgfpathlineto{\pgfqpoint{6.959883in}{3.155045in}}%
\pgfpathlineto{\pgfqpoint{6.984149in}{3.103875in}}%
\pgfpathlineto{\pgfqpoint{7.008415in}{3.050718in}}%
\pgfpathlineto{\pgfqpoint{7.032681in}{2.995540in}}%
\pgfpathlineto{\pgfqpoint{7.056947in}{2.938307in}}%
\pgfpathlineto{\pgfqpoint{7.081213in}{2.878984in}}%
\pgfpathlineto{\pgfqpoint{7.105479in}{2.817535in}}%
\pgfpathlineto{\pgfqpoint{7.129746in}{2.753927in}}%
\pgfpathlineto{\pgfqpoint{7.154012in}{2.688125in}}%
\pgfpathlineto{\pgfqpoint{7.178278in}{2.620095in}}%
\pgfpathlineto{\pgfqpoint{7.202544in}{2.549800in}}%
\pgfpathlineto{\pgfqpoint{7.226810in}{2.477208in}}%
\pgfpathlineto{\pgfqpoint{7.251076in}{2.402283in}}%
\pgfpathlineto{\pgfqpoint{7.275342in}{2.324991in}}%
\pgfpathlineto{\pgfqpoint{7.299609in}{2.245297in}}%
\pgfpathlineto{\pgfqpoint{7.323875in}{2.163167in}}%
\pgfpathlineto{\pgfqpoint{7.348141in}{2.078565in}}%
\pgfpathlineto{\pgfqpoint{7.372407in}{1.991458in}}%
\pgfpathlineto{\pgfqpoint{7.396673in}{1.901810in}}%
\pgfpathlineto{\pgfqpoint{7.420939in}{1.809588in}}%
\pgfpathlineto{\pgfqpoint{7.445205in}{1.714755in}}%
\pgfpathlineto{\pgfqpoint{7.469472in}{1.617279in}}%
\pgfpathlineto{\pgfqpoint{7.493738in}{1.517123in}}%
\pgfpathlineto{\pgfqpoint{7.518004in}{1.414254in}}%
\pgfpathlineto{\pgfqpoint{7.542270in}{1.308637in}}%
\pgfpathlineto{\pgfqpoint{7.566536in}{1.200238in}}%
\pgfpathlineto{\pgfqpoint{7.590802in}{1.089020in}}%
\pgfpathlineto{\pgfqpoint{7.635807in}{0.875000in}}%
\pgfpathlineto{\pgfqpoint{7.635807in}{0.875000in}}%
\pgfusepath{stroke}%
\end{pgfscope}%
\begin{pgfscope}%
\pgfpathrectangle{\pgfqpoint{1.500000in}{0.880000in}}{\pgfqpoint{9.300000in}{6.160000in}}%
\pgfusepath{clip}%
\pgfsetbuttcap%
\pgfsetroundjoin%
\pgfsetlinewidth{1.003750pt}%
\definecolor{currentstroke}{rgb}{0.000000,0.000000,0.000000}%
\pgfsetstrokecolor{currentstroke}%
\pgfsetdash{{3.700000pt}{1.600000pt}}{0.000000pt}%
\pgfpathmoveto{\pgfqpoint{1.500000in}{3.960000in}}%
\pgfpathlineto{\pgfqpoint{10.800000in}{3.960000in}}%
\pgfusepath{stroke}%
\end{pgfscope}%
\begin{pgfscope}%
\pgfpathrectangle{\pgfqpoint{1.500000in}{0.880000in}}{\pgfqpoint{9.300000in}{6.160000in}}%
\pgfusepath{clip}%
\pgfsetbuttcap%
\pgfsetroundjoin%
\pgfsetlinewidth{1.003750pt}%
\definecolor{currentstroke}{rgb}{0.000000,0.000000,0.000000}%
\pgfsetstrokecolor{currentstroke}%
\pgfsetdash{{3.700000pt}{1.600000pt}}{0.000000pt}%
\pgfpathmoveto{\pgfqpoint{4.600000in}{0.880000in}}%
\pgfpathlineto{\pgfqpoint{4.600000in}{7.040000in}}%
\pgfusepath{stroke}%
\end{pgfscope}%
\begin{pgfscope}%
\pgfsetrectcap%
\pgfsetmiterjoin%
\pgfsetlinewidth{0.803000pt}%
\definecolor{currentstroke}{rgb}{0.800000,0.800000,0.800000}%
\pgfsetstrokecolor{currentstroke}%
\pgfsetdash{}{0pt}%
\pgfpathmoveto{\pgfqpoint{1.500000in}{0.880000in}}%
\pgfpathlineto{\pgfqpoint{1.500000in}{7.040000in}}%
\pgfusepath{stroke}%
\end{pgfscope}%
\begin{pgfscope}%
\pgfsetrectcap%
\pgfsetmiterjoin%
\pgfsetlinewidth{0.803000pt}%
\definecolor{currentstroke}{rgb}{0.800000,0.800000,0.800000}%
\pgfsetstrokecolor{currentstroke}%
\pgfsetdash{}{0pt}%
\pgfpathmoveto{\pgfqpoint{10.800000in}{0.880000in}}%
\pgfpathlineto{\pgfqpoint{10.800000in}{7.040000in}}%
\pgfusepath{stroke}%
\end{pgfscope}%
\begin{pgfscope}%
\pgfsetrectcap%
\pgfsetmiterjoin%
\pgfsetlinewidth{0.803000pt}%
\definecolor{currentstroke}{rgb}{0.800000,0.800000,0.800000}%
\pgfsetstrokecolor{currentstroke}%
\pgfsetdash{}{0pt}%
\pgfpathmoveto{\pgfqpoint{1.500000in}{0.880000in}}%
\pgfpathlineto{\pgfqpoint{10.800000in}{0.880000in}}%
\pgfusepath{stroke}%
\end{pgfscope}%
\begin{pgfscope}%
\pgfsetrectcap%
\pgfsetmiterjoin%
\pgfsetlinewidth{0.803000pt}%
\definecolor{currentstroke}{rgb}{0.800000,0.800000,0.800000}%
\pgfsetstrokecolor{currentstroke}%
\pgfsetdash{}{0pt}%
\pgfpathmoveto{\pgfqpoint{1.500000in}{7.040000in}}%
\pgfpathlineto{\pgfqpoint{10.800000in}{7.040000in}}%
\pgfusepath{stroke}%
\end{pgfscope}%
\begin{pgfscope}%
\definecolor{textcolor}{rgb}{0.150000,0.150000,0.150000}%
\pgfsetstrokecolor{textcolor}%
\pgfsetfillcolor{textcolor}%
\pgftext[x=6.150000in,y=7.123333in,,base]{\color{textcolor}{\sffamily\fontsize{16.000000}{19.200000}\bfseries\selectfont\catcode`\^=\active\def^{\ifmmode\sp\else\^{}\fi}\catcode`\%=\active\def%{\%}First Few Laguerre Polynomials}}%
\end{pgfscope}%
\begin{pgfscope}%
\pgfsetbuttcap%
\pgfsetmiterjoin%
\definecolor{currentfill}{rgb}{1.000000,1.000000,1.000000}%
\pgfsetfillcolor{currentfill}%
\pgfsetfillopacity{0.800000}%
\pgfsetlinewidth{1.003750pt}%
\definecolor{currentstroke}{rgb}{0.800000,0.800000,0.800000}%
\pgfsetstrokecolor{currentstroke}%
\pgfsetstrokeopacity{0.800000}%
\pgfsetdash{}{0pt}%
\pgfpathmoveto{\pgfqpoint{1.616667in}{0.963333in}}%
\pgfpathlineto{\pgfqpoint{2.546670in}{0.963333in}}%
\pgfpathquadraticcurveto{\pgfqpoint{2.580004in}{0.963333in}}{\pgfqpoint{2.580004in}{0.996667in}}%
\pgfpathlineto{\pgfqpoint{2.580004in}{2.230000in}}%
\pgfpathquadraticcurveto{\pgfqpoint{2.580004in}{2.263333in}}{\pgfqpoint{2.546670in}{2.263333in}}%
\pgfpathlineto{\pgfqpoint{1.616667in}{2.263333in}}%
\pgfpathquadraticcurveto{\pgfqpoint{1.583333in}{2.263333in}}{\pgfqpoint{1.583333in}{2.230000in}}%
\pgfpathlineto{\pgfqpoint{1.583333in}{0.996667in}}%
\pgfpathquadraticcurveto{\pgfqpoint{1.583333in}{0.963333in}}{\pgfqpoint{1.616667in}{0.963333in}}%
\pgfpathlineto{\pgfqpoint{1.616667in}{0.963333in}}%
\pgfpathclose%
\pgfusepath{stroke,fill}%
\end{pgfscope}%
\begin{pgfscope}%
\pgfsetroundcap%
\pgfsetroundjoin%
\pgfsetlinewidth{2.007500pt}%
\definecolor{currentstroke}{rgb}{0.000000,0.000000,1.000000}%
\pgfsetstrokecolor{currentstroke}%
\pgfsetdash{}{0pt}%
\pgfpathmoveto{\pgfqpoint{1.650000in}{2.130000in}}%
\pgfpathlineto{\pgfqpoint{1.816667in}{2.130000in}}%
\pgfpathlineto{\pgfqpoint{1.983333in}{2.130000in}}%
\pgfusepath{stroke}%
\end{pgfscope}%
\begin{pgfscope}%
\definecolor{textcolor}{rgb}{0.150000,0.150000,0.150000}%
\pgfsetstrokecolor{textcolor}%
\pgfsetfillcolor{textcolor}%
\pgftext[x=2.116667in,y=2.071667in,left,base]{\color{textcolor}{\sffamily\fontsize{12.000000}{14.400000}\selectfont\catcode`\^=\active\def^{\ifmmode\sp\else\^{}\fi}\catcode`\%=\active\def%{\%}$L_0(x)$}}%
\end{pgfscope}%
\begin{pgfscope}%
\pgfsetroundcap%
\pgfsetroundjoin%
\pgfsetlinewidth{2.007500pt}%
\definecolor{currentstroke}{rgb}{1.000000,0.000000,0.000000}%
\pgfsetstrokecolor{currentstroke}%
\pgfsetdash{}{0pt}%
\pgfpathmoveto{\pgfqpoint{1.650000in}{1.880000in}}%
\pgfpathlineto{\pgfqpoint{1.816667in}{1.880000in}}%
\pgfpathlineto{\pgfqpoint{1.983333in}{1.880000in}}%
\pgfusepath{stroke}%
\end{pgfscope}%
\begin{pgfscope}%
\definecolor{textcolor}{rgb}{0.150000,0.150000,0.150000}%
\pgfsetstrokecolor{textcolor}%
\pgfsetfillcolor{textcolor}%
\pgftext[x=2.116667in,y=1.821667in,left,base]{\color{textcolor}{\sffamily\fontsize{12.000000}{14.400000}\selectfont\catcode`\^=\active\def^{\ifmmode\sp\else\^{}\fi}\catcode`\%=\active\def%{\%}$L_1(x)$}}%
\end{pgfscope}%
\begin{pgfscope}%
\pgfsetroundcap%
\pgfsetroundjoin%
\pgfsetlinewidth{2.007500pt}%
\definecolor{currentstroke}{rgb}{0.000000,0.501961,0.000000}%
\pgfsetstrokecolor{currentstroke}%
\pgfsetdash{}{0pt}%
\pgfpathmoveto{\pgfqpoint{1.650000in}{1.630000in}}%
\pgfpathlineto{\pgfqpoint{1.816667in}{1.630000in}}%
\pgfpathlineto{\pgfqpoint{1.983333in}{1.630000in}}%
\pgfusepath{stroke}%
\end{pgfscope}%
\begin{pgfscope}%
\definecolor{textcolor}{rgb}{0.150000,0.150000,0.150000}%
\pgfsetstrokecolor{textcolor}%
\pgfsetfillcolor{textcolor}%
\pgftext[x=2.116667in,y=1.571667in,left,base]{\color{textcolor}{\sffamily\fontsize{12.000000}{14.400000}\selectfont\catcode`\^=\active\def^{\ifmmode\sp\else\^{}\fi}\catcode`\%=\active\def%{\%}$L_2(x)$}}%
\end{pgfscope}%
\begin{pgfscope}%
\pgfsetroundcap%
\pgfsetroundjoin%
\pgfsetlinewidth{2.007500pt}%
\definecolor{currentstroke}{rgb}{0.000000,0.000000,0.000000}%
\pgfsetstrokecolor{currentstroke}%
\pgfsetdash{}{0pt}%
\pgfpathmoveto{\pgfqpoint{1.650000in}{1.380000in}}%
\pgfpathlineto{\pgfqpoint{1.816667in}{1.380000in}}%
\pgfpathlineto{\pgfqpoint{1.983333in}{1.380000in}}%
\pgfusepath{stroke}%
\end{pgfscope}%
\begin{pgfscope}%
\definecolor{textcolor}{rgb}{0.150000,0.150000,0.150000}%
\pgfsetstrokecolor{textcolor}%
\pgfsetfillcolor{textcolor}%
\pgftext[x=2.116667in,y=1.321667in,left,base]{\color{textcolor}{\sffamily\fontsize{12.000000}{14.400000}\selectfont\catcode`\^=\active\def^{\ifmmode\sp\else\^{}\fi}\catcode`\%=\active\def%{\%}$L_3(x)$}}%
\end{pgfscope}%
\begin{pgfscope}%
\pgfsetroundcap%
\pgfsetroundjoin%
\pgfsetlinewidth{2.007500pt}%
\definecolor{currentstroke}{rgb}{0.501961,0.000000,0.501961}%
\pgfsetstrokecolor{currentstroke}%
\pgfsetdash{}{0pt}%
\pgfpathmoveto{\pgfqpoint{1.650000in}{1.130000in}}%
\pgfpathlineto{\pgfqpoint{1.816667in}{1.130000in}}%
\pgfpathlineto{\pgfqpoint{1.983333in}{1.130000in}}%
\pgfusepath{stroke}%
\end{pgfscope}%
\begin{pgfscope}%
\definecolor{textcolor}{rgb}{0.150000,0.150000,0.150000}%
\pgfsetstrokecolor{textcolor}%
\pgfsetfillcolor{textcolor}%
\pgftext[x=2.116667in,y=1.071667in,left,base]{\color{textcolor}{\sffamily\fontsize{12.000000}{14.400000}\selectfont\catcode`\^=\active\def^{\ifmmode\sp\else\^{}\fi}\catcode`\%=\active\def%{\%}$L_4(x)$}}%
\end{pgfscope}%
\end{pgfpicture}%
\makeatother%
\endgroup%
}
            \caption{Graph of the first few Laguerre polynomials}
            \label{img:Laguerre_polynomials}
        \end{figure}
        \noindent which has unique solutions which are known as Laguerre polynomials. These polynomials can be generated given the following formula 
        \begin{align}
            L_n(x) = \frac{e^x}{n!} \left( \frac{d}{dx} \right)^n (x^ne^{-x})\;.
        \end{align}
        \noindent Where the generating function $g(x, t)$ is given by \cite{Arfken_Weber_Arfken_Weber_2008}
        \begin{align}
            g(x, t) = \frac{e^x}{2\pi i} \oint_C \frac{e^{-z}}{z - x - tz} dz = \frac{e^{\frac{-xt}{(1 - t)}}}{1 - t} = \sum_{n = 0}^\infty L_n(x) t^n
        \end{align}
        \begin{table}
            \centering 
            \begin{tabular}{l}
                Laguerre polynomials\\
                \hline 
                $L_0(x) = 1$\\
                $L_1(x) = -x + 1$\\
                $L_2(x) = \frac{x^2 - 4x + 2}{2!}$\\
                $L_3(x) = \frac{x^3 + 9x^2 - 18x + 6}{3!}$\\
                $L_4(x) = \frac{x^4 - 16x^3 + 72x^2 - 96x + 24}{4!}$\\
                \hline
            \end{tabular}
            \caption{Table containing the first few Laguerre polynomials}
            \label{tab:Laguerre_polynomial_table}
        \end{table}
        \noindent Similarly to the Legendre polynomials and spherical harmonic functions, the Laguerre polynomials also form a complete set of orthogonal basis functions.
        \begin{align}
            \int_0^\infty L_n(x) L_m(x) e^{-x} dx = \delta_{m n}
        \end{align}
        \noindent The Laguerre polynomials are used to described the radial part of the solution to hydrogenic wavefunctions, as discussed in Sec.~\ref{sec:The_Radial_Part}. 
        \section{Spherical Harmonics} \label{sec:Spherical_Harmonics}
        Spherical harmonics appear most commonly when solving Laplace's equations in spherical coordinates. It is composed of the Legendre polynomials with an additional phase about the azimuthal angle $\phi$ \cite{Riley_Hobson_Bence_2006}. Laplace's equation is 
        \begin{align}
            \nabla^2 f(r, \theta, \phi) = 0
        \end{align}
        \noindent The solution to this equation is separable, so the angular part of the solution can be written as a product of two functions 
        \begin{align}
            \Theta (\theta) \Phi(\phi) = P^m_l(\cos(\theta)) e^{im\phi}\;.
        \end{align}
        \noindent The product of these two angular functions is called the spherical harmonic function, it is denoted by 
        \begin{align}
            Y^m_l (\theta, \phi) = P^m_l(\cos(\theta)) e^{im\phi}\;.
        \end{align}
        \noindent Where $P^l_m(\cos \theta)$ is the associated Legendre polynomial, which is related to the standard Legendre polynomial by 
        \begin{align}
            P^m_l(x) = (-1)^m (1-x^2)^{\frac{m}{2}} \frac{d^m}{dx^m} \left(P_l(x) \right)
        \end{align}
        The spherical harmonic functions for a complete set of orthogonal basis functions 
        \begin{align}
            \int_0^{2\pi} \int_0^\pi Y^m_l(\theta, \phi) Y^{m^\prime}_{l^\prime}(\theta, \phi) d\theta d\phi = \delta_{ll^\prime}\delta_{mm^\prime}
        \end{align}
        \begin{table}[h]
            \centering 
            \resizebox{\columnwidth}{!}{
                \begin{tabular}{l l l}
                    Spherical harmonic functions&\\
                    \hline 
                    $Y^0_0 (\theta, \phi) = \sqrt{\frac{1}{4\pi}}$ & & \\
                    $Y^0_1 (\theta, \phi_) = \sqrt{\frac{3}{4\pi}} \cos \theta$ & $Y^{\pm1}_{1} (\theta, \phi) = \mp \sqrt{\frac{3}{8\pi}} \sin \theta e^{\pm i \phi}$ &\\
                    $Y^0_2 (\theta, \phi) = \sqrt{\frac{5}{16\pi}} (3 \cos^2\theta - 1)$ & $Y^{\pm 1}_2 (\theta, \phi) = \mp \sqrt{\frac{15}{8\pi}} \sin\theta \cos\theta e^{\pm i\phi}$ & $Y^{\pm 2}_2(\theta, \phi) = \sqrt{\frac{15}{32\pi}} \sin^2 \theta e^{\pm2i\phi}$\\
                    \hline
                \end{tabular}
            }
            \caption{Table of the first few spherical harmonic functions}
            \label{tab:spherical_harmonic}
        \end{table}
        \noindent The spherical harmonic functions play a key role in the solution to the angular part of the hydrogenic wavefunctions in the derivation of the quadratic Zeeman effect and the relativistic magnetic dipole moment operator.

        \section{Confluent Hypergeometric Function} \label{sec:Confluent_Hypergeometric_Function}
        The confluent hypergeometric function is a special form of the standard hypergeometric function. The confluent hypergeometric functions are produced from the solution to the differential equation 
        \begin{align}
            xy^{\prime \prime} + (c - x)y^\prime - ay = 0
        \end{align}
        \noindent where there are two possible solutions 
        \begin{align}
            y_1(x) &= 1 + \frac{a}{c} \frac{x}{1!} + \frac{a(a+1)}{c(c+1)}\frac{x^2}{2!} + \dots \equiv M(a, c ;x)\\
            y_2(x) &= x^{1-c}M(a-c + 1, 2 - c; x)\;.
        \end{align}
        \noindent The function $M(a, c;x)$ is called the confluent hypergeometric function \cite{Riley_Hobson_Bence_2006}. It can also be described by the following integral 
        \begin{align}
            M(a, c; x) = \frac{\Gamma(c)}{\Gamma(a)\Gamma(c - a)} \int_0^1 e^{tx} t^{a-1} (1 - t)^{c - a - 1} dt,
        \end{align}
        \noindent which converges so long that $c > a > 0$. The confluent hypergeometric function is a more generalized version of many of the special functions discussed in this chapter, and depending on the choice of $a$ and $c$, are capable of producing the other special functions. For example, the Laguerre polynomials can be expressed in terms of the confluent hypergeometric function as 
        \begin{align}
            M(a, b; x) = \frac{\Gamma(1 - a) \Gamma(b)}{\Gamma(b - a)} L_a^{b-1}(x)\;. \label{eq:confluent-hypergeometric}
        \end{align}
        \noindent $L_a^{b-1}(x)$ is known as the associated Laguerre polynomial, and is related to the standard Laguerre polynomial discussed in Sec.~\ref{sec:Laguerre_Polynomial} by 
        \begin{align}
            L_n^k(x) = (-1)^k \frac{d^k}{dx^k} L_{n+k}(x)\;.
        \end{align}
        \noindent In the relation given in equation \eqref{eq:confluent-hypergeometric}, $k = {b-1}$, and $n = a$. The confluent hypergeometric function is used to represent the radial solutions to the hydrogenic wavefunctions discussed in sections \ref{sec:quadratic_zeeman} and \ref{sec:magnetic_dipole_operator}.

\chapter{Methods of solving for higher-order perturbations programmatically} \label{sec:Program_perturbation}
    This section serves as a reference to the codes implemented to solve higher order perturbations for the quadratic Zeeman effect and the relativistic magnetic dipole moment operator. The codes can be found on the authors GitHub page at \href{https://github.com/epetrimoulx/Quadratic-Zeeman-Effect}{epetrimoulx/Higher-Order-Zeeman-Effect}. The code is designed such that the order of the perturbation being considered is decided by the user. The first and second order perturbation equations look like 
    \begin{align}
        \left( H^0 - E^0 \right) \vert \psi^{1}\rangle &= \left( V - E^{1}\right) \vert \psi^{0} \rangle \\
        \left( H^0 - E^0 \right) \vert \psi^{2}\rangle &= V \vert \psi^{1}\rangle + E^{1} \vert \psi^{1} \rangle + E^{2} \vert \psi^{0} \rangle
    \end{align}
    \noindent The second order perturbation equation is dependant on the answer for $\vert \psi^{(1)} \rangle$ from the first order perturbation equation. However, if the second order solution is expressed in terms of the first order solution
    \begin{align}
        \left( H^0 - E^0 \right) \vert \psi^{2}\rangle &= \left(V - E^{1}\right) (H^0 - E^0)^{-1} E^{(1)} |\psi^{(0)}\rangle + E^{2} |\psi^{0}\rangle\;.
    \end{align}
    \noindent This can be treated as just solving the first order equation again but with a modified inhomogeneous term on the RHS of the equation. 
    \begin{align}
        \left( H^0 - E^0 \right) \vert \psi^{n} \rangle &= F(V, \psi^{n-1}) + E^{n} \vert \psi^{0} \rangle\;.
    \end{align}
    \noindent This ``folding'' of the perturbation equation at higher order onto itself using the previous solutions from the lower order equations allows for the iterative calculation of higher order perturbations. The main issue with this strategy is that the equations become too long for any human to do by hand, but computationally, this can be calculated to $n^\text{th}$ order easily so long that the user carefully accounts for higher order contributing states such as the $d$-states which arise in second order perturbations of $r^2$. 

