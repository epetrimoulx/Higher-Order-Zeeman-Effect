\chapter{Derivation of the Recursion Relations}
\chapter{Hermitian Properties of the Hamiltonian} \label{sec:Hermitian_Left}
\chapter{Special Functions}
    \section{Overview}
        This section of the appendix discusses the special functions used throughout this thesis, providing their definitions and properties in greater detail than discussed in the main body of this paper. The introduction to these functions will be brief, and are placed here for convenience of the reader. Sec~.\ref{sec:Gamma_Function} introduces the gamma function and its properties, and its use in providing a closed solution to many of the integrals present in the matrix elements discussed in Sec~.\ref{sec:Integration_Techniques} is highlighted. Sections \ref{sec:Laguerre_Polynomial} and \ref{sec:Spherical_Harmonics} introduce the Laguerre polynomials and the spherical harmonic function respectively, which are key to the solution of hydrogenic wavefunctions used in this thesis. Additionally, the confluent hypergeometric function is discussed in Sec~.\ref{sec:Confluent_Hypergeometric_Function}, where an alternative formulation of the radial wavefunctions for hydrogen is given. Finally, the chapter will conclude with the description of Legendre polynomials, which can be used in a similar manner to the spherical harmonics due to their property of forming a complete set of orthogonal functions.
    \section{The gamma function} \label{sec:Gamma_Function}
        The gamma function $\Gamma(z)$ is an extension of the factorial function into the complex plane. 
        \begin{align}
            \Gamma(z) \in \mathbb{C} 
        \end{align}
        For the case where the input parameter $z \in \mathbb{Z}^+$, the function is equal to 
        \begin{align}
            \Gamma(n) = (n - 1)!\;.
        \end{align}

        \noindent This is a relation of key importance in Sec.~\ref{sec:The_Radial_Part}, where the radial integral is replaced with a factorial function for positive integers of $j$ in the sum.\\

        The gamma function also has an integral definition, which is known as the Euler integral

        \begin{align}
            \Gamma(z) = \int_0^\infty e^{-t} t^{z-1} dt, \hspace{1cm} \mathfrak{R}e(z) > 0\;.
        \end{align}

        \noindent This integral is defined as long as the real part of $z$ is greater than zero. This integral closely resembles the radial integrals present in Sec.~\ref{sec:The_Radial_Part}, which are of the form

        \begin{align}
            I_{\text{radial}} = \int_0^\infty r^j e^{-\alpha r} dr
        \end{align}

        \noindent To fit the gamma function, perform a change of variables $t = \alpha r$, $dt = \alpha dr$.

        \begin{align}
            I_{\text{radial}} &= \int_0^\infty \left( \frac{t}{\alpha} \right)^j e^{-t} \frac{1}{\alpha} dt
        \end{align}
        %
        \begin{align}
            I_{\text{radial}} &= \frac{1}{\alpha^{j+1}}\int_0^\infty {t}^j e^{-t} dt
        \end{align}

        \noindent So the solution to the radial integral is shown to be 

        \begin{align}
            I_{\text{radial}} = \frac{\Gamma(j+1)}{\alpha^{j+1}}
        \end{align}

        \noindent if $j \in \mathbb{Z}^+$, the relationship between the gamma function and the factorial function can be used. The values of $j$ are the summation indices for each power of $r$\footnote{See sections \ref{sec:quadratic_zeeman} and \ref{sec:magnetic_dipole_operator} for the summation expressions for each.}. Since the summation indices are constrained to a set of integers from 0 to infinity, it can be said that 

        \begin{align}
            \forall j, j \in Z^{0+}
        \end{align}

        \noindent Additionally, the $0^{\text{th}}$ power of $r$ (corresponding to $j = 0$) does not appear in the recursion relations of the expanded sum in \ref{sec:magnetic_dipole_operator}, and is zero in \ref{sec:quadratic_zeeman}. This further restricts the set of $j$ indices in both problems to 

        \begin{align}
            \forall j, j\in \mathbb{N}
        \end{align}

        \noindent Since the set of natural numbers only contains positive integers, the euler integral only evaluates values for positive integer $j$. The gamma function by its original definition is then replaced by the factorial function, giving the solution to the radial integrals used in sections \ref{sec:The_Radial_Part}, \ref{sec:quadratic_zeeman} and \ref{sec:magnetic_dipole_operator}.

        





    \section{Laguerre Polynomials} \label{sec:Laguerre_Polynomial}
    \section{Spherical Harmonics} \label{sec:Spherical_Harmonics}
    \section{Confluent Hypergeometric Function} \label{sec:Confluent_Hypergeometric_Function}
    \section{Legendre Polynomials} \label{sec:Legendre_Polynomial}
\chapter{Methods of solving for higher-order perturbations programmatically} \label{sec:Program_perturbation}