\chapter{Conclusion and Future Work}\label{sec:Conclusion}
    \section{Overview}
        The final chapter of this thesis begins with concluding remarks in Sec.~\ref{sec:Synthesis}, where the significance and impact of the higher order Zeeman effect is discussed. A summary of the work is also provided, restating the solution process. The impact of the result is stated and compared to the ongoing experiment at the Max Planck institute in Germany, where the correction is compared to their current results' uncertainty. Sec.~\ref{sec:FutureWork} then dicusses the future applications of this work, as well as any extensions to the work presented thus far. After the future work has been presented, the chapter concludes with the practical impact of the result of the higher order Zeeman effect.
    \section{Synthesis of Conclusion}\label{sec:Synthesis}
        This thesis first began by introducing the Zeeman effect, building the total effect by introducing the known contributions from the canonical momentum, spin interaction, nuclear interaction, and the relativistic magnetic dipole moment operator. It was shown that the canonical momentum produced both a term linearly dependant on magnetic field strength and dependant on magnetic quantum number, and a term that is dependant on the square of the magnetic field strength, not dependant on the magnetic quantum number. The linear term was successfully shown to contribute to the overall linear Zeeman effect, and it was discussed that the quadratic Zeeman effect does not contribute to further splitting of the energy levels. The spin interaction was then shown to contribute another term linearly dependant on the magnetic field strength with a dependance on magnetic quantum number, which was combined with the contribution from the canonical momentum to yield the standard linear electronic Zeeman effect. Next, the relativistic magnetic dipole moment operator was introduced which displayed another dependance on magnetic field strength arising from the relativistic corrections to the atom. The combination of all the aforementioned terms yielded the total linear electronic Zeeman effect. After the full electronic contribution was stated, the nuclear Zeeman effect was introduced, which is a factor of $\frac{m}{M}$ smaller than the electronic effects due to the difference in mass of the electron and the proton. The synthesis of all of these effects gives the current understanding of the total Zeeman effect. Using double perturbation theory, both the quadratic Zeeman effect and the relativistic magnetic dipole moment operator were perturbed and expressed as two individual power series expansions. The second order double perturbation equation then yielded a cross term between the two effects, which was expected to be of order $B^3$, and contain a dependance of the magnetic quantum number. The perturbation equations were then solved, and their solutions were verified with the Dalgarno interchange theorem, successfully producing a higher order Zeeman effect which contributes to the splitting of spectral lines.\\

        
    \section{Future Work}\label{sec:FutureWork}
    \section{Theoretical and Practical Impact}\label{sec:PracticalImpact}