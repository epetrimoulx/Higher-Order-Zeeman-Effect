\chapter{Conclusion and Future Work}\label{sec:Conclusion}
    \section{Overview}
        The final chapter of this thesis begins with concluding remarks in Sec.~\ref{sec:Synthesis}, where the significance and practical impact of the higher order Zeeman effect is discussed. A summary of the work is also provided, restating the solution process. The impact of the result is stated and compared to the ongoing experiment at the Max Planck institute in Germany, where the correction is compared to their current results' uncertainty. Sec.~\ref{sec:FutureWork} then dicusses the future applications of this work, as well as any extensions to the work presented thus far, and concludes the thesis.
    \section{Synthesis of Conclusion}\label{sec:Synthesis}
        This thesis first began by introducing the Zeeman effect, building the total effect by introducing the known contributions from the canonical momentum, spin interaction, nuclear interaction, and the relativistic magnetic dipole moment operator. It was shown that the using the canonical momentum in the non-relativistic Hamiltonian produced both a term linearly dependant on magnetic field strength and dependant on magnetic quantum number, and a term that is dependant on the square of the magnetic field strength, not dependant on the magnetic quantum number. The linear term was successfully shown to contribute to the overall linear Zeeman effect, and it was discussed that the quadratic Zeeman effect does not contribute to further splitting of the energy levels. The spin interaction was then shown to contribute another term linearly dependant on the magnetic field strength with a dependance on magnetic quantum number, which was combined with the contribution from the canonical momentum to yield the standard linear electronic Zeeman effect. Next, the relativistic magnetic dipole moment operator was introduced which displayed another dependance on magnetic field strength arising from the relativistic corrections to the atom. The combination of all the aforementioned terms yielded the total linear electronic Zeeman effect. After the full electronic contribution was stated, the nuclear Zeeman effect was introduced, which is a factor of $\frac{M}{m}$ smaller than the electronic effects due to the difference in mass of the electron and the proton. The synthesis of all of these effects gives the current understanding of the total Zeeman effect. Using double perturbation theory, both the quadratic Zeeman effect and the relativistic magnetic dipole moment operator were perturbed and expressed as two individual power series expansions. The second order double perturbation equation then yielded a cross term between the two effects, which was expected to be of order $B^3$, and contain a dependance of the magnetic quantum number. The perturbation equations were then solved using the method of Frobenius and their solutions were verified with the Dalgarno interchange theorem. A higher order Zeeman effect which further contributes to the splitting of spectral lines was then successfully synthesized for a ground state hydrogenic wavefunction.\\

        The relativistic correction to $^3$He$^+$ is found to be a contributing effect at $1$T at the micro-Hertz level, but becomes increasingly more impactful at higher magnetic field strength. By the $B = 5.7T$ level of strength, its contribution already approaches the milli-Hertz level, which is just below the detectability of cutting edge magnetometers such as the one present in the German experiment. Approaching higher magnetic field strength, the effect becomes increasingly important in high energy physics and the study of astrophysical bodies. The largest magnetic field ever constructed on Earth was $45.5T$ \cite{Hahn_Kim_Kim_Hu_Painter_Dixon_Kim_Bhattarai_Noguchi_Jaroszynski_et_al._2019}, in which the cubic Zeeman effect contributes at the level of Hz $(C_{\text{rel}}^{(2)} = 0.238536051(39) \text{ Hz})$, which is large enough to impact the uncertainty of the highest precision magnetometers. Increasing field strength further, astrophysical objects such as neutron stars can reach magnetic field strengths exceeding $10^9-10^{11}T$ \cite{Reisenegger_2001}. Here, the higher order Zeeman effects certainly are required for accurate analysis, where a neutron star with a magnetic field strength of $10^{11}T$ has a correction of $2.53232729(39) \times 10^{27}$ Hz.
        
    \section{Future Work}\label{sec:FutureWork}
        There are many areas of extension that are possible within this work. The first natural continuation to this result is to calculate the effects to higher order. Accounting for a second order perturbation for both the quadratic Zeeman effect and the relativistic magnetic dipole moment operator would provide a higher accuracy to the overall cubic Zeeman effect, contributing small, but important effects. This calculation would include higher order transitions including $s$ and $d$ states which arise naturally by including the total expression for the quadratic Zeeman effect, which was written in terms of Legendre polynomials $P_2$ and $P_0$. The first order perturbation equation only contains the $P_0$ term, since in a ground state hydrogenic wavefunction, all $s$ states are spherically symmetric. Additionally, future work can be done by analyzing $^3$He, since the result from $^3$He$^+$ can be treated as a first-order approximation to that of $^3$He, with electron-electron interactions introduced as perturbative corrections. The link between the result from $^3$He$^+$ and the first order approximation for $^3$He is described mathematically by the following perturbation equation
        \begin{align}
            \hat{H}(^3\text{He}) = \hat{H}(^3\text{He}^+) + \lambda \hat{H}^\prime(e-e)\;.
        \end{align}
        \noindent This opens the door for the analysis of high-precision Zeeman effects in not only single electron systems, but in 2-electron systems as well. This provides the ability to further study the behaviour of spectral line splitting in many body atomic systems.

    