\begin{abstract}
    S\hspace{-3pt}ince its discovery, the Zeeman effect has played a large role in the field of atomic physics and magnetometry, which is the study of the intensity of magnetic fields across space and time. It is also the physical basis for nuclear magnetic resonance (NMR) and magnetic resonance imaging (MRI). The use of the Zeeman effect as a magnetometer is based on the assumption of a linear relationship between magnetic field strength $B$ and transition frequency. This work calculates nonlinear corrections of order $B^3$ due to relativistic effects by the use of double perturbation theory. The results are applied to the helion ion $^3$He$^+$ for which high precision experiments are in progress at the Max Planck Institute in Heidelberg, Germany. The results will be used in experimental research involving the construction of a high-precision magnetometer.
\end{abstract}