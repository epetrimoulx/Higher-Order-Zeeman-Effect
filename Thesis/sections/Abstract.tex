\begin{abstract}
    S\hspace{-3pt}ince its discovery, the Zeeman effect has played a large role in the field of atomic physics and magnetometry, which is the study of the intensity of magnetic field across space and time. There have been several calculations to include the relativistic corrections, field inhomogeneities, and quadratic effects in hydrogenic systems. However, little is known about its behavior in multi-electron atoms such as $^3$He, which is of key interest in magnetometry and the muon magnetic moment anomaly $(\mu_g - 2)$, for which there is a 5.0 $\sigma$ discrepancy with the standard model prediction.

    Using perturbation theory, analytic corrections to the Zeeman effect are calculated for $^3$He$^+$, including nonlinearities. These calculations are validated via the Dalgarno interchange theorem up to first order, and provide corrections of order $B^3$. This work will be used in experimental research involving the construction of an absolute magnetometer based on $^3$He nuclear magnetic resonance at the University of Michigan in conjunction with Klaus Blaum's Max Plank Heidelberg group on a self-calibrated measurement of the helion magnetic moment.
\end{abstract}